\documentclass[12pt,]{article}
%\usepackage{lmodern}  Melissa removed to deal with font rendering issue
\usepackage{amssymb,amsmath}
\usepackage{ifxetex,ifluatex}
\usepackage{fixltx2e} % provides \textsubscript

%Melissa removed the following section to deal with font rendering issue
%\ifnum 0\ifxetex 1\fi\ifluatex 1\fi=0 % if pdftex
%  \usepackage[T1]{fontenc}
%  \usepackage[utf8]{inputenc}
%%\else % if luatex or xelatex
%  \ifxetex
%    \usepackage{mathspec}
%  \else
%    \usepackage{fontspec}
%  \fi
%  \defaultfontfeatures{Ligatures=TeX,Scale=MatchLowercase}
%  \newcommand{\euro}{€}
%%%%%%\fi

% use upquote if available, for straight quotes in verbatim environments
\IfFileExists{upquote.sty}{\usepackage{upquote}}{}
% use microtype if available
\IfFileExists{microtype.sty}{%
\usepackage{microtype}
\UseMicrotypeSet[protrusion]{basicmath} % disable protrusion for tt fonts
}{}
\usepackage[margin=1in]{geometry}
\usepackage{hyperref}
\PassOptionsToPackage{usenames,dvipsnames}{color} % color is loaded by hyperref
\hypersetup{unicode=true,
            pdftitle={Status of Pacific ocean perch (Sebastes alutus) along the US west coast in 2017},
            pdfborder={0 0 0},
            breaklinks=true}
\urlstyle{same}  % don't use monospace font for urls
\usepackage{graphicx,grffile}
\makeatletter
\def\maxwidth{\ifdim\Gin@nat@width>\linewidth\linewidth\else\Gin@nat@width\fi}
\def\maxheight{\ifdim\Gin@nat@height>\textheight\textheight\else\Gin@nat@height\fi}
\makeatother
% Scale images if necessary, so that they will not overflow the page
% margins by default, and it is still possible to overwrite the defaults
% using explicit options in \includegraphics[width, height, ...]{}
\setkeys{Gin}{width=\maxwidth,height=\maxheight,keepaspectratio}
\setlength{\parindent}{0pt}
\setlength{\parskip}{6pt plus 2pt minus 1pt}
\setlength{\emergencystretch}{3em}  % prevent overfull lines
\providecommand{\tightlist}{%
  \setlength{\itemsep}{0pt}\setlength{\parskip}{0pt}}
\setcounter{secnumdepth}{5}

%%% Use protect on footnotes to avoid problems with footnotes in titles
\let\rmarkdownfootnote\footnote%
\def\footnote{\protect\rmarkdownfootnote}

%%% Change title format to be more compact
\usepackage{titling}

% Create subtitle command for use in maketitle
\newcommand{\subtitle}[1]{
  \posttitle{
    \begin{center}\large#1\end{center}
    }
}

\setlength{\droptitle}{-2em}
  \title{Status of Pacific ocean perch (\emph{Sebastes alutus}) along the US west
coast in 2017}
  \pretitle{\vspace{\droptitle}\centering\huge}
  \posttitle{\par}
  \author{}
  \preauthor{}\postauthor{}
  \date{}
  \predate{}\postdate{}


% This file contains all of the LaTeX packages you may need to compile the document
% Documentation for each package can be found onlines
\usepackage{tabularx}                                             % table environment providing flexibility
\usepackage{caption}                                              % for creating captions  
\usepackage{longtable}                                            % allows tables to span multiple pages
\usepackage{tabu}
\usepackage{rotating}                                             % allows for sideways tables
%\usepackage{float}                                                % floating environments; may not need in rmarkdown
\usepackage{placeins}                                             % keeps floats from moving
\usepackage{floatrow}                                             % package to put table captions at the top
\floatsetup[table]{capposition = top}                             % line to put captions at the top of pander tables
\usepackage{indentfirst}                                          % indents first paragraph of a section
\usepackage{mdwtab}                                               % continued float multi-page figure
\usepackage{enumerate}                                            % create lists
\usepackage{hyperref}                                             % highlight cross references
\hypersetup{colorlinks=true, urlcolor=blue, linktoc=page, linkcolor=blue, citecolor=blue} %define referencing colors
%\usepackage{makebox}                                             % make boxes around text
\usepackage[usenames,dvipsnames]{xcolor}                          % color name options
%\usepackage[space]{grffile}                                      % spaces in file name path
\usepackage{soul}                                                 % highlight text
\usepackage{enumitem}                                             % numbered lists
%\usepackage{lineno}                                               % Line numbers; comment out for final
\usepackage{upquote}                                              % produce grave accent in latex
\usepackage{verbatim}                                             % produces verbatim results
\usepackage{fancyvrb}                                             % verbatim in a box
%\usepackage{draftwatermark}                                      % places Draft watermark in background; comment out for final
\usepackage{textcomp}                                             % fixes error with packages interfering
\usepackage{lscape}                                               % rotate pages - to allow for landscape longtables
%\pdfinterwordspaceon                                             % fix loss of inter word spacing
\usepackage{cmap}                                                 % fix mapping characters to unicode
\RequirePackage[linewidth = 1]{pdfcomment}                        % pdf comments
\RequirePackage[l2tabu, orthodox]{nag}                            % checks packages related to the accessibility?
%\usepackage[inline]{showlabels}                                   % show table and figure labels; comment out for final
%\RequirePackage[tagged]{accessibilityMeta}


%\linenumbers                                                      % specify use of line numbers


\definecolor{light-gray}{gray}{.85}                               % define light-gray as a color
%\usepackage[tagged]{accessibility-meta}

 
%\showlabels[\color{mred}]{label}

% Redefines (sub)paragraphs to behave more like sections
\ifx\paragraph\undefined\else
\let\oldparagraph\paragraph
\renewcommand{\paragraph}[1]{\oldparagraph{#1}\mbox{}}
\fi
\ifx\subparagraph\undefined\else
\let\oldsubparagraph\subparagraph
\renewcommand{\subparagraph}[1]{\oldsubparagraph{#1}\mbox{}}
\fi

\begin{document}
\maketitle


\begin{center}
\thispagestyle{empty}


\vspace{.5cm}

\includegraphics{Sebastes_alutus}~\\[0.5cm]
%\pdftooltip{\includegraphics{Sebastes_alutus}}{This is a fish.}



Chantel R. Wetzel\textsuperscript{1}\\
Lee Cronin-Fine\textsuperscript{2}\\
Kelli F. Johnson\textsuperscript{1,2}\\

\vspace{.5cm}

\small
\textsuperscript{1}Northwest Fisheries Science Center, U.S. Department of Commerce, National Oceanic and Atmospheric Administration, National Marine Fisheries Service, 2725 Montlake Boulevard East, Seattle, Washington 98112\\

\vspace{.3cm}

\textsuperscript{2}University of Washington, School of Aquatic and Fishery Sciences\\





\vspace{1cm}

\vfill
DRAFT SAFE\\
Disclaimer: This information is distributed solely for the purpose of pre-dissemination
peer review under applicable information quality guidelines. It has not been formally
disseminated by NOAA Fisheries. It does not represent and should not be construed to
represent any agency determination or policy. 



\vspace{.3cm}
%Bottom of the page
%{\large \today}

\newpage

\vspace{3cm}

Please cite as:\\

Wetzel, C.R., Cronin-Fine, L., and Johnson, K.F. 2017. Status of Pacific ocean perch (\textit{Sebastes alutus}) along the US west coast in 2017. Pacific Fishery Managment Council, 7700 Ambassador Place NE, Suite 200, Portland, OR 97220. 

\vspace{3cm}

\maketitle






\pagenumbering{roman}
\setcounter{page}{1}
\end{center}

{
\setcounter{tocdepth}{4}
\tableofcontents
}
\setlength{\parskip}{5mm plus1mm minus1mm} \pagebreak

\setcounter{page}{1} \renewcommand{\thefigure}{\alph{figure}}
\renewcommand{\thetable}{\alph{table}}

\section*{Executive Summary}\label{executive-summary}
\addcontentsline{toc}{section}{Executive Summary}

\subsection*{Stock}\label{stock}
\addcontentsline{toc}{subsection}{Stock}

This assessment reports the status of the Pacific ocean perch rockfish
(\emph{Sebastes alutus}) off the US west coast from Northern California
to the Canadian border using data through 2016. Pacific ocean perch are
most abundant in the Gulf of Alaska and have been observed off of Japan,
in the Bering Sea, and south to Baja California, though they are sparse
south of Oregon and rare in southern California. Although neither
catches nor other data from north of the US-Canada border were included
in this assessment, the connectivity of these populations and the
contribution to the biomass possibly through adult migration and/or
larval dispersion is not certain. To date, no significant genetic
differences have been found in the range covered by this assessment.

\subsection*{Landings}\label{landings}
\addcontentsline{toc}{subsection}{Landings}

Harvest of Pacific ocean perch first exceeded 1 mt off the US west coast
in 1918. Catches ramped up in the 1940s with large removals in
Washington waters. During the 1950s the removals primary occurred in
Oregon waters with catches from Washington declining following the
1940s. The largest removals, occurring between 1966-1968, were largely a
result of harvest by foreign vessels. The fishery proceeded with more
moderate removals ranging between 1165 to 2619 metric tons (mt) per year
between 1969 and 1980. Removals generally declined from 1981 to 1994 to
between 1031 and 1617 mt per year. Pacific ocean perch was declared
overfished in 1999, resulting in large reductions in harvest in years
since the declaration. Since 2000, annual landings of Pacific ocean
perch have ranged between 54-270 mt, with landings in 2016 totaling 68
mt.

Pacific ocean perch are a desirable market species and discarding has
historically been low. However, management restrictions (e.g.~trip
limits) resulted in increased discarding starting in the early 1990s.
During the 2000s discarding increased for Pacific ocean perch due to
harvest restrictions imposed to allow rebuilding, with estimated discard
rates from the fishery peaking in 2009 and 2010 to approximately 50\%,
prior to implementation of catch shares in 2011. Since 2011, discarding
of Pacific ocean perch has been estimated to be less than 3.5\%.

\begin{table}[ht]
\centering
\caption{Landings (mt) for the past 10 years for Pacific ocean perch by source.} 
\label{tab:Exec_catch}
\begin{tabular}{l>{\centering}p{0.7in}>{\centering}p{0.7in}>{\centering}p{0.7in}>{\centering}p{0.7in}>{\centering}p{0.7in}>{\centering}p{0.7in}}
  \hline
Year & California & Oregon & Washington & At-sea hake & Survey & Total Landings \\ 
  \hline
2007 & 0.15 & 83.65 & 45.12 & 4.05 & 0.58 & 133.55 \\ 
  2008 & 0.39 & 58.64 & 16.61 & 15.93 & 0.80 & 92.36 \\ 
  2009 & 0.92 & 58.74 & 33.22 & 1.56 & 2.72 & 97.17 \\ 
  2010 & 0.14 & 58.00 & 22.29 & 16.87 & 1.68 & 98.98 \\ 
  2011 & 0.12 & 30.26 & 19.66 & 9.17 & 1.94 & 61.14 \\ 
  2012 & 0.18 & 30.41 & 21.79 & 4.52 & 1.62 & 58.51 \\ 
  2013 & 0.08 & 34.86 & 14.83 & 5.41 & 1.71 & 56.89 \\ 
  2014 & 0.18 & 33.91 & 15.82 & 3.92 & 0.57 & 54.40 \\ 
  2015 & 0.12 & 38.05 & 11.41 & 8.71 & 1.59 & 59.88 \\ 
  2016 & 0.23 & 40.81 & 13.12 & 10.30 & 3.10 & 67.56 \\ 
   \hline
\end{tabular}
\end{table}

\FloatBarrier

\begin{figure}
\centering
\includegraphics{PacificOceanPerch2017_Assessment_files/figure-latex/unnamed-chunk-14-1.pdf}
\caption{Landings of Pacific ocean perch for California, Oregon,
Washington, the foriegn fishery (1966-1976), at-sea hake fishery, and
fishery-independent surveys. \label{fig:Exec_catch1}}
\end{figure}

\FloatBarrier

\subsection*{Data and Assessment}\label{data-and-assessment}
\addcontentsline{toc}{subsection}{Data and Assessment}

This a new full assessment for Pacific ocean perch, which was last
assessed in 2011. In this assessment, aspects of the model including
landings, data, and modelling assumptions were re-evaluated. The
assessment was conducted using the length- and age-structured modeling
software Stock Synthesis (version 3.30.03.05). The coastwide population
was modeled allowing separate growth and mortality parameters for each
sex (a two-sex model) from 1918 to 2017 and forecasted beyond 2017.

All of the data sources included in the base model for Pacific ocean
perch have been re-evaluated for 2017. Changes of varying degrees have
occurred in the data from those used in previous assessments. The
landings history has been updated and extended back to 1918. Harvest was
negligible prior to that year. Survey data from the Alaska and Northwest
Fisheries Science Centers have been used to construct indices of
abundance analyzed using a spatio-temporal delta-model. Length, marginal
age or conditional age-at-length compositions were also created for each
fishery-dependent and -independent data source.

The definition of fishing fleets have changed from those in the 2011
assessment. Three fishing fleets were specified within the model: 1) a
combined bottom trawl, mid-water trawl, and fixed gear fleet, where only
a small fraction of Pacific ocean perch were captured by fixed gear
(termed the fishery fleet), 2) the historical foreign fleet, and 3) the
at-sea hake fishery. The fleet grouping was based on discarding
practices. The fishery fleet estimated a retention curve based on
discarding data and known management restrictions. However, very little
if any discarding is assumed to have occurred by the foreign fleet and
the catch reported by the at-sea hake fishery accounts for both
discarded and landed fish and hence, no additional mortality was
estimated for each of these fleets.

The assessment uses landings data and discard-fraction estimates; survey
indices of abundance; length- or age-composition data for each year and
fishery or survey (with conditional age-at-length compositional data for
the NWFSC shelf-slope survey); information on weight-at-length,
maturity-at-length, and fecundity-at-length; information on natural
mortality and the steepness of the Beverton-Holt stock-recruitment
relationship; and estimates of ageing error. Recruitment at
``equilibrium spawning output'', length-based selectivity of the
fisheries and surveys, retention of the fishery, catchability of the
surveys, growth, the time-series of spawning output, age and size
structure, and current and projected future stock status are outputs of
the model. Natural mortality (0.054 yr\textsuperscript{-1}) and
steepness (0.50) were fixed in the final model. This was done due to
relatively flat likelihood surfaces, such that fixing parameters and
then varying them in sensitivity analyses was deemed the best way to
characterize uncertainty.

Although this assessment using many types of data since the 1980s, there
is little information about steepness and natural mortality. Estimates
of steepness are uncertain partly because of highly variable
recruitment. Uncertainty in natural mortality is common in many fish
stock assessments even when length and age data are available.

A number of sources of uncertainty are explicitly included in this
assessment. This assessment includes gender differences in growth, a
non-linear relationship between individual spawner biomass and effective
spawning output, and an updated relationship between length and
maturity, based upon non-published information (Melissa Head, personal
communication, NOAA, NWFSC). As is always the case, overall uncertainty
is greater than that predicted by a single model specification. Among
other sources of uncertainty that are not included in the current model
are the degree of connectivity between the stocks of Pacific ocean perch
off of Vancouver Island, British Columbia and those in US waters, and
the effect of climatic variables on recruitment, growth and survival.

A base model was selected that best captures the central tendency for
those sources of uncertainty considered in the model.

\subsection*{Stock Biomass}\label{stock-biomass}
\addcontentsline{toc}{subsection}{Stock Biomass}

The predicted spawning output from the base model generally showed a
slight decline prior to 1966 when fishing by the foreign fleet
commenced. A short, but sharp decline occurred between 1966 and 1970,
followed by a period of the spawning output stabilizing or with a
minimal decline until the late 1990s. The stock showed increases in
stock size following the year 2000 due to a combination of strong
recruitment and low catches. The 2017 estimated spawning output relative
to unfished equilibrium spawning output is above the target of 40\% of
unfished spawning output at 76.6\% (\(\sim\) 95\% asymptotic interval:
\(\pm\) 55.6\%-97.7\%). Approximate confidence intervals based on the
asymptotic variance estimates show that the uncertainty in the estimated
spawning output is high.

\begin{figure}
\centering
\includegraphics{r4ss/plots_mod1/ts7_Spawning_output_with_95_asymptotic_intervals_intervals.png}
\caption{Estimated time-series of spawning output trajectory (circles
and line: median; light broken lines: 95\% credibility intervals) for
the base assessment model. \label{fig:Spawnbio_all}}
\end{figure}

\begin{figure}
\centering
\includegraphics{r4ss/plots_mod1/ts9_Spawning_depletion_with_95_asymptotic_intervals_intervals.png}
\caption{Estimated time-series of relative spawning output (depletion)
(circles and line: median; light broken lines: 95\% credibility
intervals) for the base assessment model. \label{fig:RelDeplete_all}}
\end{figure}

\begin{table}[ht]
\centering
\caption{Recent trend in estimated spawning output (million eggs) and estimated relative spawning output (depletion).} 
\label{tab:SpawningDeplete_mod1}
\begin{tabular}{l>{\centering}p{1.3in}>{\centering}p{1.2in}>{\centering}p{1in}>{\centering}p{1.2in}}
  \hline
Year & Spawning Output (million eggs) & \~{} 95\% Confidence Interval & Estimated Depletion & \~{} 95\% Confidence Interval \\ 
  \hline
2008 & 3745 & 1620 - 5870 & 0.544 & 0.380 - 0.708 \\ 
  2009 & 3885 & 1688 - 6083 & 0.564 & 0.395 - 0.733 \\ 
  2010 & 3976 & 1731 - 6221 & 0.577 & 0.405 - 0.749 \\ 
  2011 & 4032 & 1759 - 6305 & 0.585 & 0.412 - 0.759 \\ 
  2012 & 4067 & 1780 - 6354 & 0.590 & 0.416 - 0.764 \\ 
  2013 & 4091 & 1797 - 6384 & 0.594 & 0.420 - 0.768 \\ 
  2014 & 4197 & 1857 - 6538 & 0.609 & 0.433 - 0.785 \\ 
  2015 & 4516 & 2021 - 7011 & 0.656 & 0.470 - 0.841 \\ 
  2016 & 4931 & 2231 - 7630 & 0.716 & 0.517 - 0.914 \\ 
  2017 & 5280 & 2407 - 8153 & 0.766 & 0.556 - 0.977 \\ 
   \hline
\end{tabular}
\end{table}

\FloatBarrier

\subsection*{Recruitment}\label{recruitment}
\addcontentsline{toc}{subsection}{Recruitment}

Recruitment deviations were estimated for the entire assessment period.
There is little information regarding recruitment prior to 1965, and the
uncertainty in these estimates is expressed in the model. Past
assessments estimated large recruitments in 1999 and 2000. In recent
years, a recruitment of unprecedented size is estimated to have occurred
in 2008. Additionally, there is early evidence of a strong recruitment
in 2013. The four lowest recruitments estimated within the model (in
ascending order) occurred in 2012, 2003, 2005, and 2007.

\begin{figure}
\centering
\includegraphics{r4ss/plots_mod1/ts11_Age-0_recruits_(1000s)_with_95_asymptotic_intervals.png}
\caption{Time-series of estimated Pacific ocean perch recruitments for
the base model with 95\% confidence or credibility intervals.
\label{fig:Recruits_all}}
\end{figure}

\begin{table}[ht]
\centering
\caption{Recent estimated trend in recruitment and estimated recruitment deviations determined from the base model} 
\label{tab:Recruit_mod1}
\begin{tabular}{>{\centering}p{.8in}>{\centering}p{1.0in}>{\centering}p{1.4in}>{\centering}p{1.0in}>{\centering}p{1.4in}}
  \hline
Year & Estimated Recruitment & \~{} 95\% Confidence Interval & Estimated Recruitment Devs. & \~{} 95\% Confidence Interval \\ 
  \hline
2008 & 116128 & 66566 - 202591 & 2.623 & 2.323 - 2.923 \\ 
  2009 & 4731 & 2047 - 10932 & -0.592 & -1.347 - 0.163 \\ 
  2010 & 7499 & 3650 - 15404 & -0.140 & -0.732 - 0.453 \\ 
  2011 & 15198 & 7730 - 29880 & 0.562 & 0.031 - 1.093 \\ 
  2012 & 2101 & 879 - 5026 & -1.420 & -2.237 - -0.603 \\ 
  2013 & 29027 & 13826 - 60941 & 1.118 & 0.482 - 1.754 \\ 
  2014 & 4630 & 1629 - 13160 & -0.813 & -1.863 - 0.238 \\ 
  2015 & 10661 & 2987 - 38052 & -0.004 & -1.372 - 1.364 \\ 
  2016 & 11016 & 3082 - 39382 & 0.000 & -1.372 - 1.372 \\ 
  2017 & 11253 & 3151 - 40194 & 0.000 & -1.372 - 1.372 \\ 
   \hline
\end{tabular}
\end{table}

\FloatBarrier

\subsection*{Exploitation Status}\label{exploitation-status}
\addcontentsline{toc}{subsection}{Exploitation Status}

The spawning output of Pacific ocean perch reached a low in 1989.
Landings for Pacific ocean perch decreased significantly in 2000
compared to previous years. The estimated relative depletion was
possibly below the target biomass level between the 1970s and 1990s, but
has likely remained above the target otherwise, and currently is
significantly greater than the 40\% unfished spawning output target.
Throughout the late 1960s and the early 1970s the exploitation rate and
values of relative spawning potential
((1-SPR)/(1-SPR\textsubscript{50\%})) were mostly above target levels.
Recent exploitation rates on Pacific ocean perch were predicted to be
significantly below target levels.

\begin{table}[ht]
\centering
\caption{Recent trend in spawning potential ratio (1-SPR)/(1-SPR50) and summary exploitation rate for Pacific ocean perch.} 
\label{tab:SPR_Exploit_mod1}
\begin{tabular}{l>{\centering}p{0.9in}>{\centering}p{1.2in}>{\centering}p{1.2in}>{\centering}p{1.2in}}
  \hline
Year & (1-SPR)/ (1-SPR50\%) & \~{} 95\% Confidence Interval & Exploitation Rate & \~{} 95\% Confidence Interval \\ 
  \hline
2007 & 0.087 & 0.039 - 0.134 & 0.002 & 0.001 - 0.003 \\ 
  2008 & 0.072 & 0.031 - 0.113 & 0.002 & 0.001 - 0.002 \\ 
  2009 & 0.097 & 0.040 - 0.153 & 0.002 & 0.001 - 0.004 \\ 
  2010 & 0.092 & 0.039 - 0.145 & 0.002 & 0.001 - 0.003 \\ 
  2011 & 0.032 & 0.014 - 0.050 & 0.001 & 0.000 - 0.001 \\ 
  2012 & 0.031 & 0.014 - 0.048 & 0.001 & 0.000 - 0.001 \\ 
  2013 & 0.030 & 0.013 - 0.046 & 0.001 & 0.000 - 0.001 \\ 
  2014 & 0.026 & 0.012 - 0.040 & 0.000 & 0.000 - 0.001 \\ 
  2015 & 0.026 & 0.012 - 0.040 & 0.001 & 0.000 - 0.001 \\ 
  2016 & 0.027 & 0.012 - 0.041 & 0.001 & 0.000 - 0.001 \\ 
   \hline
\end{tabular}
\end{table}

\FloatBarrier

\begin{figure}
\centering
\includegraphics{r4ss/plots_mod1/SPR3_ratiointerval.png}
\caption{Estimated relative spawning potential ratio (1-SPR)/(1-SPR50\%)
for the base model. One minus SPR is plotted so that higher exploitation
rates occur on the upper portion of the y-axis. The management target is
plotted as a red horizontal line and values above this reflect harvests
in excess of the overfishing proxy based on the SPR50\% harvest rate.
The last year in the time-series is 2016. \label{fig:SPR_all}}
\end{figure}

\begin{figure}
\centering
\includegraphics{r4ss/plots_mod1/SPR4_phase.png}
\caption{Phase plot of estimated (1-SPR)/(1-SPR50\%) vs.~depletion
(B/Btarget) for the base case model.\label{fig:Phase_all}}
\end{figure}

\FloatBarrier

\subsection*{Ecosystem Considerations}\label{ecosystem-considerations}
\addcontentsline{toc}{subsection}{Ecosystem Considerations}

Rockfish are an important component of the California Current ecosystem
along the US west coast, with more than sixty five species filling
various niches in both soft and hard bottom habitats from the nearshore
to the continental slope, as well as near bottom and pelagic zones.
Pacific ocean perch are generally considered to be semi-demersal, but
there can, at times, be a significant pelagic component to their
distribution.

Recruitment is one mechanism by which the ecosystem may directly impact
the population dynamics of Pacific ocean perch. The 1999 cohort for many
species of rockfish was large -- sometimes significantly so. Long-term
averages suggest that environmental conditions may influence the
spawning success and survival of larvae and juvenile rockfish. Pacific
ocean perch showed above average recruitment deviations in 1999 and
2000. The specific pathways through which environmental conditions exert
influence on Pacific ocean perch dynamics are unclear; however, changes
in water temperature and currents, distribution of prey and predators,
and the amount and timing of upwelling are all possible linkages.
Changes in the environment may also result in changes in
length-at-maturity, fecundity, growth, and survival which can affect the
status of the stock and its susceptibility to fishing. Unfortunately,
there are few data available for Pacific ocean perch that provide
insights into these effects.

Fishing has effects on both the age-structure of a population, as well
as the habitat with which the target species is associated. Fishing
often targets larger, older fish and years of fishing mortality results
in a truncated age-structure when compared to unfished conditions.
Rockfish are often associated with habitats containing living structure
such as sponges and corals, and fishing may alter that habitat to a less
desirable state. This assessment provides a look at the effects of
fishing on age structure, and recent studies on essential fish habitat
are beginning to characterize important locations for rockfish
throughout their life history; however, there is little current
information available to evaluate the specific effects of fishing on the
ecosystem issues specific to Pacific ocean perch.

\subsection*{Reference Points}\label{reference-points}
\addcontentsline{toc}{subsection}{Reference Points}

This stock assessment estimates that the spawning output of Pacific
ocean perch is above the management target. Due to reduced landing and
the large 2008 year-class, an increasing trend in spawning output was
estimated in the base model. The estimated depletion in 2017 is 76.6\%
(\(\sim\) 95\% asymptotic interval: \(\pm\) 55.6\%-97.7\%),
corresponding to an unfished spawning output of 5,280 million eggs
(\(\sim\) 95\% asymptotic interval: 2,407-8,153 million eggs). Unfished
age 3+ biomass was estimated to be 147,286 mt in the base model. The
target spawning output based on the biomass target (\(SB_{40\%}\)) is
2,755.7 million eggs, with an equilibrium catch of 1,808.3 mt.
Equilibrium yield at the proxy \(F_{MSY}\) harvest rate corresponding to
\(SPR_{50\%}\) is 1,822.5 mt. Estimated MSY catch is at a 1,825.3
spawning output of 2,425 million eggs (35.2\% depletion)

\begin{table}[ht]
\centering
\caption{Summary of reference 
                                      points and management quantities for the 
                                      base case.} 
\label{tab:Ref_pts_mod1}
\begin{tabular}{>{\raggedright}p{4.1in}>{\centering}p{.65in}>{\centering}p{1.4in}}
  \hline
\textbf{Quantity} & \textbf{Estimate} & \textbf{$\sim$95\%  Confidence Interval} \\ 
  \hline
Unfished spawning output (million eggs) & 6889.2 &   4860.7 -   8917.6 \\ 
  Unfished age 3+ biomass (mt) & 147286 & 104000.8 - 190571.2 \\ 
  Unfished recruitment (R0, thousands) & 12110.2 &   9046.1 -  16212.1 \\ 
  Spawning output(2017 million eggs) & 5280.4 &   2407.4 -   8153.3 \\ 
  Relative spawning output (depletion) (2017) & 0.766 &    0.556 -    0.977 \\ 
  \textbf{$\text{Reference points based on } \mathbf{SB_{40\%}}$} &  &  \\ 
  Proxy spawning output ($B_{40\%}$) & 2755.7 &   1944.3 -     3567 \\ 
  SPR resulting in $B_{40\%}$ ($SPR_{B40\%}$) & 0.55 &     0.55 -     0.55 \\ 
  Exploitation rate resulting in $B_{40\%}$ & 0.028 &    0.028 -    0.029 \\ 
  Yield with $SPR_{B40\%}$ at $B_{40\%}$ (mt) & 1808.3 &   1278.2 -   2338.4 \\ 
  \textbf{\textit{Reference points based on SPR proxy for MSY}} &  &  \\ 
  Spawning output & 2296.4 &   1620.2 -   2972.5 \\ 
  $SPR_{proxy}$ & 0.5 &  \\ 
  Exploitation rate corresponding to $SPR_{proxy}$ & 0.033 &    0.033 -    0.034 \\ 
  Yield with $SPR_{proxy}$ at $SB_{SPR}$ (mt) & 1822.5 &   1288.5 -   2356.5 \\ 
  \textbf{\textit{Reference points based on estimated MSY values}} &  &  \\ 
  Spawning output at $MSY$ ($SB_{MSY}$) & 2425 &   1708.1 -   3141.8 \\ 
  $SPR_{MSY}$ & 0.514 &    0.512 -    0.516 \\ 
  Exploitation rate at $MSY$ & 0.032 &    0.031 -    0.032 \\ 
  $MSY$ (mt)  & 1825.3 &   1290.4 -   2360.2 \\ 
   \hline
\end{tabular}
\end{table}

\FloatBarrier

\subsection*{Management Performance}\label{management-performance}
\addcontentsline{toc}{subsection}{Management Performance}

Exploitation rates on Pacific ocean perch exceeded MSY proxy target
harvest rates during the 1960s and 1970s, resulting in sharp declines in
the spawning output. Exploitation rates subsequently declined to rates
at or below the management target in the late 1970s. Management
restrictions imposed in the 1990s further reduced exploitation rates. An
overfished declaration for Pacific ocean perch resulted in very low
exploitation rates since 2001 with Annual Catch Limits (ACLs) being set
far below the Overfishing Limit (OFL) and Acceptable Biological Catch
(ABC) values.

\begin{table}[ht]
\centering
\caption{Recent trend in total catch and  
                              landings (mt) relative to the management guidelines. 
                              Estimated total catch reflect the landings 
                              plus the model estimated discarded biomass.} 
\label{tab:mnmgt_perform}
\scalebox{0.9}{
\begin{tabular}{>{\raggedleft}p{0.5in}>{\centering}p{1.1in}>{\centering}p{1.1in}>{\centering}p{1.1in}>{\centering}p{1.1in}>{\centering}p{1.1in}}
  \hline
Year & OFL (mt; ABC prior to 2011) & ABC (mt) & ACL (mt; OY prior to 2011) & Total Landings (mt) & Estimated Total Catch (mt) \\ 
  \hline
\text{2007} & 900 &  & 150 & 134 & 159 \\ 
  \text{2008} & 911 &  & 150 & 92 & 135 \\ 
  \text{2009} & 1,160 &  & 189 & 97 & 194 \\ 
  \text{2010} & 1,173 &  & 200 & 99 & 183 \\ 
  \text{2011} & 1,026 & 981 & 180 & 61 & 62 \\ 
  \text{2012} & 1,007 & 962 & 183 & 59 & 60 \\ 
  \text{2013} & 844 & 807 & 150 & 57 & 58 \\ 
  \text{2014} & 838 & 801 & 153 & 54 & 56 \\ 
  \text{2015} & 842 & 805 & 158 & 60 & 61 \\ 
  \text{2016} & 850 & 813 & 164 & 68 & 68 \\ 
   \hline
\end{tabular}
}
\end{table}

\FloatBarrier

\subsection*{Unresolved Problems and Major
Uncertainties}\label{unresolved-problems-and-major-uncertainties}
\addcontentsline{toc}{subsection}{Unresolved Problems and Major
Uncertainties}

\begin{enumerate}

\item The current data for Pacific ocean perch weighted according to the Francis weighting approach do not contain information regarding steepness.  The estimated final status is highly dependent upon the assumed steepness value, as is typical for most US west coast groundfish assessments.  The data available and the modeling approach applied in 2011 supported a steepness value of 0.40.  However, the current data no longer support this value.  Models the used the mean to the 2017 steepness prior (0.72) resulted in an estimated a stock size near unfished conditions leading to low survey catchabiltiy for the NWFSC shelf-slope survey which was deemed implausible by the Scientific and Statistical Committee (SSC).  A steepness value in the final model was determined by calculating spawning output across a range of steepness values (0.25-0.95) which were considered equally likely.  The expected (i.e. arithmetic mean) ending spawning output was calculated and the steepness value most closely associated with the expected value was identified, a value of 0.50.  Additional research for alternative approaches for determining steepness values when traditional approaches do not seem appropriate should be identified.  

\item Pacific ocean perch off the US west coast may be a fraction of a much large population extending into Canada or even Alaska. Modelling only a part of the total population might contribute to the lack of correspondence between the survey indices and other data sources, as seen in the ln($R0$) profiles and age-structured production model diagnostics as well as some of the observation variability. While this comment is not intended to reflect badly on the STAT's capabilities, it is important to recognize that stock structure could potentially be a major source of uncertainty regarding the assessment results.
  
\item The indices of abundance used in the final base model provide almost no information on population scale, as demonstrated in the ln($R0$) profiles examined during the review. The Triennial survey was the only index that provided signal with respect to population scale. However, this survey was removed in the final base model due to concerns about the quality of the survey and conflicts with other data. There are large amounts of composition data in the model, with both age- and length-compositions being included for some fleets. The compositional data and catch are providing the majority of the information on the estimated and derived quantities.

\item Use of conditional-age-at-length composition data provides information on parameters beyond those of the length-at-age relationship. The conditional-age-at-length data are robust to length-based processes (Piner et al. 2016), however they are also influenced by age-based processes (Lee et al. 2017). No age-based processes were used in the assessment model as a link to the data, meaning that the conditional-age-at-length data were assumed to be unbiased with respect to the population. The conditional-age-at-length data were shown to be very influential on the estimated dynamics beyond growth estimates. More theoretical work in this area is needed to understand how to best the use this type of information and what potential systems or observation model processes could invalidate the assumption of randomness at length.

\end{enumerate}

\subsection*{Decision Table}\label{decision-table}
\addcontentsline{toc}{subsection}{Decision Table}

Model uncertainty has been described by the estimated uncertainty within
the base model and by the sensitivities to different model structure.
The results from the final base model were sensitivey to both the
assumed steepness or natural mortality values. The STAT team and the
STAR panel agreed to select natural mortality (\(M\)) as the main axis
for uncertainty when projecting the population under alternative harvest
strategies. The 12.5\% and 87.5\% quantiles based on spawning output
uncertainty were used to determine the low and high values for \(M\) of
0.04725 and 0.0595 yr\textsuperscript{-1}.

Due to the sensitivity associated with the assessment given the assumed
steepness value the assessment is classified as a Category 2 stock
assessment. Therefore, the sigma for P* to determine the catch reduction
to account for scientific uncertainty is 0.72, since the estimated sigma
in the assessment is less than this for current spawning biomass (0.27).

\begin{table}[ht]
\centering
\caption{Projections of potential OFL (mt) and ABC (mt) and the estimated spawning output and relative depletion based on ABC removals.  The 2017 and 2018 
                                               removals are set at the harvest limits currently set by management of 281 mt per year.} 
\label{tab:OFL_projection}
\begin{tabular}{>{\raggedleft}p{0.5in}>{\centering}p{1.1in}>{\centering}p{1.1in}>{\centering}p{1.6in}>{\centering}p{1.1in}}
  \hline
Year & OFL & ABC & Spawning Output (million eggs) & Relative Depletion (\%) \\ 
  \hline
2019 & 4753 & 4340 & 5741 & 83 \\ 
  2020 & 4632 & 4229 & 5745 & 83 \\ 
  2021 & 4499 & 4108 & 5723 & 83 \\ 
  2022 & 4364 & 3984 & 5666 & 82 \\ 
  2023 & 4230 & 3862 & 5586 & 81 \\ 
  2024 & 4105 & 3748 & 5494 & 80 \\ 
  2025 & 3991 & 3644 & 5395 & 78 \\ 
  2026 & 3889 & 3551 & 5292 & 77 \\ 
  2027 & 3797 & 3467 & 5188 & 75 \\ 
  2028 & 3712 & 3389 & 5084 & 74 \\ 
   \hline
\end{tabular}
\end{table}

\FloatBarrier

\begin{table}[ht]
\centering
\caption{Summary of 10-year 
                                             projections beginning in 2019 
                                             for alternate states of nature based on 
                                             an axis of uncertainty for the base model. The range of natural mortality values corresponded to the 12.5 and 87.5th quantile
                                             from the uncertainty around final spawning biomass.
                                             Columns range over low, mid, and high
                                             states of nature, and rows range over different 
                                             assumptions of catch levels. The SPR50 catch stream is based on the equilibrium yield applying the SPR50 harvest rate.} 
\label{tab:Decision_table_mod1}
\scalebox{0.85}{
\begin{tabular}{l|cc|>{\centering}p{.7in}c|>{\centering}p{.7in}c|>{\centering}p{.7in}c}
   \multicolumn{3}{c}{}  &  \multicolumn{2}{c}{} 
                               & \multicolumn{2}{c}{\textbf{States of nature}} 
                               & \multicolumn{2}{c}{} \\
  \multicolumn{3}{c}{}  &  \multicolumn{2}{c}{M = 0.04725} 
                               & \multicolumn{2}{c}{M = 0.054} 
                               &  \multicolumn{2}{c}{M = 0.0595} \\
 \hline
 & Year & Catch & Spawning Output & Depletion (\%) & Spawning Output & Depletion (\%) & Spawning Output & Depletion (\%) \\ 
  \hline
 & 2019 & 4340 & 3944 & 62.9 & 5741 & 83.3 & 7505 & 96.8 \\ 
   & 2020 & 4229 & 3909 & 62.4 & 5745 & 83.4 & 7542 & 97.3 \\ 
   & 2021 & 4108 & 3858 & 61.6 & 5723 & 83.1 & 7546 & 97.3 \\ 
  ABC & 2022 & 3984 & 3784 & 60.4 & 5666 & 82.2 & 7503 & 96.8 \\ 
   & 2023 & 3862 & 3695 & 59.0 & 5586 & 81.1 & 7427 & 95.8 \\ 
   & 2024 & 3748 & 3600 & 57.4 & 5494 & 79.7 & 7332 & 94.6 \\ 
   & 2025 & 3644 & 3502 & 55.9 & 5395 & 78.3 & 7226 & 93.2 \\ 
   & 2026 & 3551 & 3404 & 54.3 & 5292 & 76.8 & 7113 & 91.8 \\ 
   & 2027 & 3467 & 3308 & 52.8 & 5188 & 75.3 & 6996 & 90.3 \\ 
   & 2028 & 3389 & 3213 & 51.3 & 5084 & 73.8 & 6879 & 88.7 \\ 
   \hline
 & 2019 & 1822 & 3944 & 62.9 & 5741 & 83.3 & 7505 & 96.8 \\ 
   & 2020 & 1822 & 4022 & 64.2 & 5857 & 85.0 & 7654 & 98.7 \\ 
   & 2021 & 1822 & 4083 & 65.1 & 5946 & 86.3 & 7768 & 100.2 \\ 
  SPR50 & 2022 & 1822 & 4117 & 65.7 & 5996 & 87.0 & 7830 & 101.0 \\ 
   & 2023 & 1822 & 4131 & 65.9 & 6016 & 87.3 & 7852 & 101.3 \\ 
   & 2024 & 1822 & 4133 & 65.9 & 6017 & 87.3 & 7848 & 101.2 \\ 
   & 2025 & 1822 & 4125 & 65.8 & 6004 & 87.1 & 7824 & 100.9 \\ 
   & 2026 & 1822 & 4110 & 65.6 & 5979 & 86.8 & 7786 & 100.4 \\ 
   & 2027 & 1822 & 4090 & 65.3 & 5947 & 86.3 & 7736 & 99.8 \\ 
   & 2028 & 1822 & 4067 & 64.9 & 5908 & 85.8 & 7679 & 99.1 \\ 
   \hline
\end{tabular}
}
\end{table}

\FloatBarrier

\subsection*{Research and Data Needs}\label{research-and-data-needs}
\addcontentsline{toc}{subsection}{Research and Data Needs}

There are many areas of research that could be improved to benefit the
understanding and assessment of Pacific ocean perch. Below, are issues
that are considered of importance.

\begin{enumerate}

\item \textbf{Natural mortality}: Uncertainty in natural mortality translates into uncertain estimates of status and sustainable fishing levels for Pacific ocean perch. The collection of additional age data, re-reading of older age samples, reading old age samples that are unread, and improved understanding of the life history of Pacific ocean perch may reduce that uncertainty.

\item \textbf{Steepness}: The amount of stock resilience, steepness, dictates the rate at which a stock can rebuild from low stock sizes.  Improved understating regarding the steepness parameter for US west coast Pacific ocean perch will reduce our uncertainty regarding current stock status.

\item \textbf{Basin-wide understanding of stock structure, biology, connectivity, and distribution:} This is a stock assessment for Pacific ocean perch off of the west coast of the US and does not consider data from British Columbia or Alaska. Further investigating and comparing the data and predictions from British Columbia and Alaska to determine if there are similarities with the US west coast observations would help to define the connectivity between Pacific ocean perch north and south of the US-Canada border.

\end{enumerate}

\begin{sidewaystable}[ht]
\centering
\caption{Base model results summary.} 
\label{tab:base_summary}
\scalebox{0.6}{
\begin{tabular}{r>{\centering}p{1.1in}>{\centering}p{1.1in}>{\centering}p{1.1in}>{\centering}p{1.1in}>{\centering}p{1.1in}>{\centering}p{1.1in}>{\centering}p{1.1in}>{\centering}p{1.1in}>{\centering}p{1.1in}>{\centering}p{1.1in}}
  \hline
Quantity & 2008 & 2009 & 2010 & 2011 & 2012 & 2013 & 2014 & 2015 & 2016 & 2017 \\ 
  \hline
OFL (mt) & 911 & 1,160 & 1,173 & 1,026 & 1,007 & 844 & 838 & 842 & 850 & 964 \\ 
  ACL (mt) & 150 & 189 & 200 & 180 & 183 & 150 & 153 & 158 & 164 & 281 \\ 
  Landings (mt) & 92 & 97 & 99 & 61 & 59 & 57 & 54 & 60 & 68 &  \\ 
  Total Est. Catch (mt) & 135 & 194 & 183 &  62 &  60 &  58 &  56 &  61 &  68 &  \\ 
   \hline
(1-$SPR$)(1-$SPR_{50\%}$) & 0.072 & 0.097 & 0.092 & 0.032 & 0.031 & 0.030 & 0.026 & 0.026 & 0.027 &  \\ 
   \hline
Exploitation rate & 0.002 & 0.002 & 0.002 & 0.001 & 0.001 & 0.001 & 0.000 & 0.001 & 0.001 &  \\ 
  Age 3+ biomass (mt) &  86308.1 &  86803.2 &  86769.2 &  98173.2 & 103709.0 & 109254.0 & 115075.0 & 119187.0 & 124995.0 & 128529.0 \\ 
   \hline
Spawning Output & 3745 & 3885 & 3976 & 4032 & 4067 & 4091 & 4197 & 4516 & 4931 & 5280 \\ 
  ~95\% CI & 1620 - 5870 & 1688 - 6083 & 1731 - 6221 & 1759 - 6305 & 1780 - 6354 & 1797 - 6384 & 1857 - 6538 & 2021 - 7011 & 2231 - 7630 & 2407 - 8153 \\ 
   \hline
Relative Depletion & 0.544 & 0.564 & 0.577 & 0.585 & 0.590 & 0.594 & 0.609 & 0.656 & 0.716 & 0.766 \\ 
  ~95\% CI & 0.380 - 0.708 & 0.395 - 0.733 & 0.405 - 0.749 & 0.412 - 0.759 & 0.416 - 0.764 & 0.420 - 0.768 & 0.433 - 0.785 & 0.470 - 0.841 & 0.517 - 0.914 & 0.556 - 0.977 \\ 
   \hline
Recruits & 116128 &   4731 &   7499 &  15198 &   2101 &  29027 &   4630 &  10661 &  11016 &  11253 \\ 
  ~95\% CI & 66566 - 202591 & 2047 - 10932 & 3650 - 15404 & 7730 - 29880 & 879 - 5026 & 13826 - 60941 & 1629 - 13160 & 2987 - 38052 & 3082 - 39382 & 3151 - 40194 \\ 
   \hline
\end{tabular}
}
\end{sidewaystable}

\FloatBarrier

\begin{figure}
\centering
\includegraphics{r4ss/plots_mod1/yield1_yield_curve.png}
\caption{Equilibrium yield curve for the base case model. Values are
based on the 2016 fishery selectivity and with steepness fixed at 0.50.
\label{fig:Yield_all}}
\end{figure}

\FloatBarrier

\newpage

\renewcommand{\thefigure}{\arabic{figure}}
\renewcommand{\thetable}{\arabic{table}}

\setcounter{figure}{0} \setcounter{table}{0}

\pagenumbering{arabic}

\section{Introduction}\label{introduction}

\subsection{Distribution and Stock
Structure}\label{distribution-and-stock-structure}

Pacific ocean perch (\emph{Sebastes alutus}) are most abundant in the
Gulf of Alaska and have been observed off of Japan, in the Bering Sea,
and south to Baja California, although they are sparse south of Oregon
and rare in southern California. While genetic studies have found three
populations of Pacific ocean perch off of British Columbia related to
unique geography and oceanic conditions (Seeb and Gunderson
\protect\hyperlink{ref-seeb_genetic_1988}{1988}, Withler et al.
\protect\hyperlink{ref-withler_co-existing_2001}{2001}) with, notably, a
separate stock off of Vancouver Island, no significant genetic
differences have been found in the range covered by this assessment.
Pacific ocean perch show dimorphic growth, with females reaching a
slightly larger size than males. Males and females are equally abundant
on rearing grounds at age 1.5.

The Pacific ocean perch population has been modeled as a single stock
off of the US west coast (essentially northern California to the
Canadian border, since Pacific ocean perch are seen extremely rarely in
central and southern California). Good recruitments show up in
size-composition data throughout all portions of this area, which
supports the single stock hypothesis. This assessment includes landings
and catch data for Pacific ocean perch from the states of Washington,
Oregon and California, along with records from foreign fisheries, the
at-sea hake fleet, and fishery-independent surveys.

\subsection{Historical and Current
Fishery}\label{historical-and-current-fishery}

Prior to 1966, the Pacific ocean perch resource off of the northern
portion of the US west coast was harvested almost entirely by Canadian
and US vessels. Harvest was negligible prior to 1940, reached 1,367 mt
in 1950, 3,243 mt in 1961 and 7,636 mt in 1965. Catches increased
dramatically after 1965, with the introduction of large distant-water
fishing fleets from the Soviet Union and Japan. Both nations employed
large factory stern trawlers as their primary method for harvesting
Pacific ocean perch. Peak removals are estimated at 18,883 mt in 1966
and 14,591 mt in 1967. These numbers are based upon a re-analysis of the
foreign catch data (Rogers
\protect\hyperlink{ref-rogers_species_2003}{2003}), which focused on
deriving a more realistic species composition for catches previously
identified only as Pacific ocean perch. Catches declined rapidly
following these peak years, and Pacific ocean perch stocks were
considered to be severely depleted throughout the Oregon-Vancouver
Island region by 1969 (Gunderson
\protect\hyperlink{ref-gunderson_population_1977}{1977}, Gunderson et
al. \protect\hyperlink{ref-gunderson_status_1977}{1977}). Landed harvest
averaged 1,381 mt over the period 1977-94. Landings have continued to
decline since 1994, primarily due to more restrictive management (Table
\ref{tab:Comm_Catch} and Figure \ref{fig:Catch}).

\subsection{Summary of Management History and
Performance}\label{summary-of-management-history-and-performance}

Prior to 1977, Pacific ocean perch in the northeast Pacific were managed
by the Canadian Government in its waters and by the individual states in
waters off of the US. With the implementation of the Magnuson Fishery
Conservation and Management Act (MFCMA) in 1977, US territorial waters
were extended to 200 nautical miles from shore and primary
responsibility for management of the groundfish stocks off Washington,
Oregon, and California shifted from the states to the Pacific Fishery
Management Council (PFMC) and the National Marine Fisheries Service
(NMFS). At that time, however, a Fishery Management Plan for the West
Coast groundfish stocks had not yet been approved. In the interim, the
state agencies worked with the PFMC to address conservation issues. In
1981, the PFMC adopted a management strategy to rebuild the depleted
Pacific ocean perch stocks to levels that would produce Maximum
Sustainable Yield (MSY) within 20 years. On the basis of cohort analysis
(Gunderson \protect\hyperlink{ref-gunderson_results_1978}{1978}), the
PFMC set Acceptable Biological Catch (ABC) levels at 600 mt for the US
portion of the Vancouver INPFC area and 950 mt for the Columbia
International North Pacific Fishery Commission (INPFC) area. To
implement this strategy, the states of Oregon and Washington each
established landing limits for Pacific ocean perch. Trawl trip limits of
various forms remained in effect through 2016 (Table \ref{tab:Regs}).

The landings of Pacific ocean perch have been historically governed by
harvest guidelines and trip limits, while recently management has
imposed total catch harvest limits in the form of overfishing limits
(OFLs), acceptable biological catches (ABCs), and annual catch limits
(ACLs). A trawl rationalization program, consisting of an individual
fishing quota (IFQ) catch shares system was implemented in 2011 for the
limited entry trawl fleet targeting non-whiting groundfish, including
Pacific ocean perch and the trawl fleet targeting and delivering whiting
to shore-based processors. The limited entry at-sea trawl sectors
(motherships and catch-processors) that target whiting and process
at-sea are managed in a system of harvest cooperatives.

Limits on Pacific ocean perch were first established in 1983 (Table
\ref{tab:Regs}). These were implemented as area closures, trip limits,
and cumulative landing limits. In 1999, Pacific ocean perch was declared
overfished with the assessment estimating the spawning output below the
management limit (25\% of virgin biomass or output). In reaction to the
overfished declaration, harvest limits were reduced relative to previous
years and a rebuilding plan was implemented in 2001 with recent ACLs
being set well below the estimated OFLs (Table
\ref{tab:mnmgt_perform_tables}).

\subsection{Fisheries off Canada and
Alaska}\label{fisheries-off-canada-and-alaska}

Pacific ocean perch can be found in waters off the US west coast and
northward through Alaskan waters. In contrast to the Pacific ocean perch
stock off the US west coast, each assessed portion of the stock in
Canadian and Alaskan waters have historically been estimated to be above
management targets. The subset of the stock off the US west coast
represents the tail of the species distribution with little to no
Pacific ocean perch being encountered south of northern California. The
most recent updated assessments for the Bering Sea and the Gulf of
Alaska stocks determined that neither stock is in an overfished state
and recommended acceptable biological catches of 43,723 mt and 23,918
mt, respectively, for 2017.

In Canadian waters Pacific ocean perch has the largest single-species
quota, accounting for approximately 25\% of all rockfish landings by
weight in the bottom trawl fleet. The Canadian Pacific ocean perch stock
is broken into three separate areas that are individually assessed. The
status of the stock within each area is above Canadian management
targets.

\section{Data}\label{data}

Data used in the Pacific ocean perch assessment are summarized in Figure
\ref{fig:data_plot}. A description of each data source is provided
below.

\subsection{Fishery-Independent Data}\label{fishery-independent-data}

Research surveys have been used to provide fishery-independent
information about the abundance, distribution, and biological
characteristics of Pacific ocean perch. A coast-wide survey was
conducted in 1977 (Gunderson and Sample
\protect\hyperlink{ref-gunderson_distribution_1980}{1980}) and repeated
every three years through 2004 (referred to as the `Triennial shelf
survey'). The NMFS coordinated a cooperative research survey of the
Pacific ocean perch stocks off Washington and Oregon with the Washington
Department of Fish and Wildlife (WDFW) and the Oregon Department of Fish
and Wildlife (ODFW) in March-May 1979 (Wilkins and Golden
\protect\hyperlink{ref-wilkins_condition_1983}{1983}). This survey was
repeated in 1985 (referred to as the Pacific ocean perch survey). Two
slope surveys have been conducted off the West Coast in recent years,
one using the research vessel Miller Freeman, which ended in 2001
(referred to as the `AFSC slope survey'), and another ongoing
cooperative survey using commercial fishing vessels which began in 1998
as a DTS (Dover sole, thornyhead, and sablefish) survey and was expanded
to other groundfish in 1999 (referred to as the `NWFSC slope survey').
In 2003, this survey was expanded spatially to include the shelf. This
last survey, conducted by the NWFSC, continues to cover depths from
30-700 fathoms (55-1280 meters) on an annual basis (referred to as the
`NWFSC shelf-slope survey').

Age estimates for Pacific ocean perch prior to the 1980s were made via
surface ageing of otoliths, which misses the very tight annuli at the
edge of the otolith once the fish reaches near maximum size. Ages are
highly biased by age 14, and maximum age was estimated to be in the 20s,
which lead to an overestimate of the natural mortality rate and the
productivity of the stock. Using break and burn methods, Pacific ocean
perch have been aged to over 100 years. Otoliths from
fishery-independent and -dependent sources that were only surface age
reads were excluded from this assessment due to the bias associated with
these age reads.

\subsubsection{Northwest Fisheries Science Center (NWFSC) Shelf-Slope
Survey}\label{northwest-fisheries-science-center-nwfsc-shelf-slope-survey}

The NWFSC shelf-slope survey is based on a random-grid design; covering
the coastal waters from a depth of 55 m to 1,280 m (Bradburn et al.
\protect\hyperlink{ref-bradburn_2003_2011}{2011}). This design uses four
chartered industry vessels in most years, assigned to a roughly equal
number of randomly selected grid cells. The survey, which has been
conducted from late-May to early-October each year, is divided into two
2-vessel passes off the coast, which are executed from north to south.
This design therefore incorporates both vessel-to-vessel differences in
catchability as well as variance associated with selecting a relatively
small number (approximately 700) of cells from a very large population
of possible cells (greater than 11,000) distributed from the Mexican to
the Canadian border.

The data from the NWFSC shelf-slope survey was analyzed using a
spatio-temporal delta-model (Thorson et al.
\protect\hyperlink{ref-thorson_geostatistical_2015}{2015}), implemented
as an R package, VAST (Thorson and Barnett
\protect\hyperlink{ref-thorson_comparing_2017}{2017}), which is publicly
available online (\url{https://github.com/James-Thorson/VAST}). Spatial
and spatio-temporal variation is specifically included in both encounter
probability and positive catch rates, a logit-link for encounter
probability and a log-link for positive catch rates. Vessel-year effects
were included for each unique combination of vessel and year in the data
to account for the random selection of commercial vessels used during
sampling (Helser et al.
\protect\hyperlink{ref-helser_generalized_2004}{2004}, Thorson and Ward
\protect\hyperlink{ref-thorson_accounting_2014}{2014}). Spatial
variation was approximated using 1,000 knots, and the model used the
bias-correction algorithm (Thorson and Kristensen
\protect\hyperlink{ref-thorson_implementing_2016}{2016}) in Template
Model Builder (Kristensen et al.
\protect\hyperlink{ref-kristensen_tmb:_2016}{2016}). Further details
regarding model structure are available in the user manual
(\url{https://github.com/James-Thorson/VAST/blob/master/examples/VAST_user_manual.pdf}).
The stratification and modeling configuration are provided in Table
\ref{tab:strata}.

The smallest Pacific ocean perch tend to occur in the shallower depths
(\textless{} 200 m) with only larger individuals occurring at depths
deeper than 300 m. Data collected by the NWFSC shelf-slope survey
between depths of 55 - 549 m and north of \(42^\circ\) and south of
\(49^\circ\) were used to generate an index of abundance from 2003-2016.
The estimated index of abundance is shown in Table
\ref{tab:Index_Summary}. For contrast, the design based values are shown
in Table \ref{tab:Design_Based}. The lognormal distribution with random
strata-year and vessel effects had the lowest AIC and was chosen as the
final model. The Q-Q plot does not show any departures from the assumed
distribution (Figure \ref{fig:nw_qq}). The indices for the NWFSC
shelf-slope survey show a tentative decline in the population between
2003 and 2009, with an increasing trend in biomass between the 2009 and
2016 median point estimates.

Length compositions were expanded based upon the stratification and the
age data was used as conditional age-at-length data. The number of tows
with length data ranged from 33 in 2006 to 69 in 2015 (Table
\ref{tab:NWcombo_Lengths}), where ages were collected for Pacific ocean
perch in nearly every tow length data were collected (Table
\ref{tab:NWcombo_Ages}). The expanded length frequencies from this
survey show an increase in small fish starting in 2010 (Figure
\ref{fig:nw_Length}). The age frequencies provide clear evidence of
large year-classes moving through the population from the 1999, 2000,
and 2008 recruitments; with early indications of a large 2013
recruitment (Figure \ref{fig:nw_Age}).

The input sample sizes for length and marginal age-composition data for
all fishery-independent surveys were calculated according to Stewart and
Hamel (\protect\hyperlink{ref-stewart_bootstrapping_2014}{2014}), which
determined that the approximate realized sample size for shelf/slope
rockfish species was \(2.43*N_{\text{tow}}\). The effective sample size
of conditional-age-at-length data was set at the number of fish at each
length by sex and by year.

\subsubsection{Northwest Fisheries Science Center (NWFSC) Slope
Survey}\label{northwest-fisheries-science-center-nwfsc-slope-survey}

The NWFSC slope survey covered waters throughout the summer from 183 m
to 1,280 m north of \(34^\circ 30^\prime\) S, which is near Point
Conception, from 1999 and 2002. Tows conducted between the depths of 183
and 549 m were used to create an index of abundance using a bayesian
delta-GLMM and the VAST delta-GLMM models. The estimated index of
abundance is shown in Table \ref{tab:Index_Summary}. The stratification
and modeling configuration are provided in Table \ref{tab:strata}. Based
on the diagnostics of the bayesian delta-GLMM, which does not account
for spatial effects, a gamma distribution allowing for additional
probability of extreme catch events with year-vessel random effects was
selected as the final model. The Q-Q plot does show a minimal departure
from the assumed distribution (Figure \ref{fig:nw_slope_qq}), but was
determined to be acceptable based on the alternative model
distributions. The trend of abundance across the four surveys years was
generally flat with high estimated annual variance. Sensitivities (not
shown) were done evaluating the excluding of this index within the base
model or using the VAST estimated index and neither approach was found
to be influential on the model estimates.

Length and age compositions were available for 2001 and 2002 and were
expanded based upon the survey stratification (Tables
\ref{tab:NWslope_Lengths} and \ref{tab:NWslope_Ages}). The expanded
length frequencies from this survey shows that primarily only large fish
were captured both years (Figure \ref{fig:nw_slope_Length}). The
majority of fish observed by this survey were aged at greater than 10
years (Figure \ref{fig:nw_slope_Age}).

The input sample sizes for length and marginal age-composition data were
calculated according to Stewart and Hamel
(\protect\hyperlink{ref-stewart_bootstrapping_2014}{2014}) described in
Section
\ref{northwest-fisheries-science-center-nwfsc-shelf-slope-survey}.

\subsubsection{Alaska Fisheries Science Center (AFSC) Slope
Survey}\label{alaska-fisheries-science-center-afsc-slope-survey}

The AFSC slope survey operated during autumn (October-November) aboard
the R/V Miller Freeman. Partial survey coverage of the US west coast
occurred during 1988-96 and complete coverage (north of
\(34^\circ 30^\prime\) S) during 1997, 1999, 2000, and 2001. Only the
four years of consistent and complete surveys plus 1996, which surveyed
north of \(43^\circ\) N latitude to the US-Canada border, were used in
this assessment. These same data years were used in the last assessment.
The number of tows with length data ranged from 19 in 2000 to 48 in 1996
(Table \ref{tab:AFSC_Lengths}). Because a large number of positive tows
occurred in 1996, it was decided to include that year, which surveyed
from \(43^\circ\) N latitude to the US-Canada border. Therefore, only
tows from \(43^\circ\) N latitude to the US-Canada border were used.

An index of abundance was estimated based on the data using the VAST
delta-GLMM model. The estimated index of abundance is shown in Table
\ref{tab:Index_Summary}. The stratification and modeling configuration
are provided in Table \ref{tab:strata}. The lognormal distribution with
random strata-year had the lowest AIC and was chosen as the final model.
The Q-Q plot does not show any departures from the assumed distribution
(Figure \ref{fig:afsc_qq}). The trend in the indices was generally flat
over time.

Length compositions were available for each year the survey was
conducted. No age data were available from this survey. The expanded
length frequencies from this survey were generally of larger fish (
\textgreater{} 30 cm), for 1997 where the highest frequency of fish were
between 20 and 30 cm for both females and males (Figure
\ref{fig:afsc_Length}).

The input sample sizes for length and marginal age composition data were
calculated according to Stewart and Hamel
(\protect\hyperlink{ref-stewart_bootstrapping_2014}{2014}) described in
Section
\ref{northwest-fisheries-science-center-nwfsc-shelf-slope-survey}.

\subsubsection{Pacific Ocean Perch
Survey}\label{pacific-ocean-perch-survey}

A survey designed to sample Pacific ocean perch was conducted in 1979
and again in 1985 (for a detailed description see Ianelli et al.
(\protect\hyperlink{ref-ianelli_status_1992}{1992})). An index of
abundance was estimated based on the data using the VAST delta-GLMM
model. The estimated index of abundance is shown in Table
\ref{tab:Index_Summary}. The stratification and modeling configuration
are provided in Table \ref{tab:strata}. The lognormal distribution with
random strata-year had the lowest AIC and was chosen as the final model.
The Q-Q plot does not show any departures from the assumed distribution
(Figure \ref{fig:pop_qq}). The index shows a clear decline in abundance
between the two survey years.

Length and age compositions were expanded based on the survey
stratification. The survey had 125 and 126 Pacific ocean perch tows
(Table \ref{tab:POP_Lengths}) and ages were only available in 1985 due
to surface reads for the 1979 data (Table \ref{tab:POP_Ages}). The
length frequencies for both years are highest between the 30-45 cm range
(Figure \ref{fig:POP_Length}) with ages in 1985 having a large number of
fish age 40 and greater (Figure \ref{fig:POP_Age}).

The input sample sizes for length and marginal age-composition data were
calculated according to Stewart and Hamel
(\protect\hyperlink{ref-stewart_bootstrapping_2014}{2014}) described in
Section
\ref{northwest-fisheries-science-center-nwfsc-shelf-slope-survey}.

\subsubsection{Fishery Independent Data Not Included in the Base
Model}\label{fishery-independent-data-not-included-in-the-base-model}

The follow datasets were evaluated but not included in the base model.

\paragraph{Triennial Shelf Survey}\label{triennial-shelf-survey}

The Triennial shelf survey was first conducted by the AFSC in 1977 and
spanned the time-frame from 1977-2004. The survey's design and sampling
methods are most recently described in Weinberg et al.
(\protect\hyperlink{ref-weinberg_estimation_2002}{2002}). Its basic
design was a series of equally-spaced transects from which searches for
tows in a specific depth range were initiated. The survey design has
changed slightly over the period of time. In general, all of the surveys
were conducted in the mid-summer through early fall: the 1977 survey was
conducted from early July through late September; the surveys from 1980
through 1989 ran from mid-July to late September; the 1992 survey
spanned from mid-July through early October; the 1995 survey was
conducted from early June to late August; the 1998 survey ran from early
June through early August; and the 2001 and 2004 surveys were conducted
in May-July.

Haul depths ranged from 91-457 m during the 1977 survey with no hauls
shallower than 91 m. The surveys in 1980, 1983, and 1986 covered the
West Coast south to \(36.8^\circ\) N latitude and a depth range of
55-366 m. The surveys in 1989 and 1992 covered the same depth range but
extended the southern range to \(34.5^\circ\) N (near Point Conception).
From 1995 through 2004, the surveys covered the depth range 55-500 m and
surveyed south to \(34.5^\circ\) N. In the final year of the Triennial
series, 2004, the NWFSC's Fishery Resource and Monitoring division
(FRAM) conducted the survey and followed very similar protocols as the
AFSC.

The Triennial shelf survey was not used in the final base model for a
number of reasons. First, there were concerns regarding the varying
sampling and targeting of specific species by year across the
time-series. Secondly, the Triennial shelf survey targeted the shelf of
the West Coast and would not be expected to sample well slope species
such as Pacific ocean perch. There were limited observations of Pacific
ocean perch relative to other surveys (e.g.~NWFSC shelf-slope survey)
and the length and age distributions varied in such a manner that would
indicate either poor sampling of Pacific ocean perch or inconsistent
sampling of the population.

Information regarding the Triennial shelf survey index of abundance and
the number of samples available and plots of the composition data are
available in Appendix C, section
\ref{appendix-c.-description-of-cpue-and-triennial-data}.

\paragraph{Washington Research
Lengths}\label{washington-research-lengths}

Research length and ages were provided by WDFW. However, the information
regarding the nature of the research cruise and collection methods have
been lost to time. The data set includes lengths and ages that were
collected between 1967-1972 and in 1979. The distribution of lengths
across years collected were consistent with primarily only larger
Pacific ocean perch, 35-40 cm, being selected. All age data were based
upon surface reads which unfortunately are highly biased at relatively
young ages for Pacific ocean perch. Due to the lack of information
regarding the collection of these data, they were not selected to be
apart of the base model but a sensitivity was conducted which evaluated
the impact of these data.

\subsection{Fishery-Dependent Data}\label{fishery-dependent-data}

\subsubsection{Commercial Fishery
Landings}\label{commercial-fishery-landings}

\textbf{Washington}

Historical commercial fishery landings of Pacific ocean perch in
Washington for the years 1908-2016 were obtained from Theresa Tsou
(WDFW) and Phillip Weyland (WDFW). This assessment is the first Pacific
ocean perch assessment to include a historical catch reconstruction
provided by Washington state and, hence, the historical catches for
Washington differ from those used in the 2011 assessment. WDFW also
provided catches for 1981-2016 period to include re-distribution of the
``URCK'' landings in the PacFIN database. These data are currently not
available from PacFIN.

\textbf{Oregon}

Historical commercial fishery landings of Pacific ocean perch in Oregon
for the years 1892-1986 were obtained from Alison Whitman (ODFW). A
description of the methods can be found in Karnowski et al.
(\protect\hyperlink{ref-karnowski_historical_2014}{2014}). Recent
landings (1987-2016) were obtained from PacFIN (retrieval dated May 2,
2017, Pacific States Marine Fisheries Commission, Portland, Oregon;
www.psmfc.org). The catch data from the POP and POP2 categories
contained within PacFIN for Pacific ocean perch were used for this
assessment. Additional catches from 1987-1999 for Pacific ocean perch
under the URCK category not yet available in PacFIN were received
directly from the state and combined with the landings data available
for that period within PacFIN (Patrick Mirrick, personal communication,
ODFW).

\textbf{California}

Historical commercial fishery landings of Pacific ocean perch were
obtained directly from John Field at the SWFSC due to database issues
for the historical period for the California Cooperative Groundfish
Survey data system, also known as CALCOM Database (128.114.3.187) for
the years 1916-1980. The catches received included revisions in the
catch history from 1948-1960 based on fish that were landed in northern
California that were not included in the original reconstruction. A
description of the historical reconstruction methods can be found in
Ralston et al.
(\protect\hyperlink{ref-ralston_documentation_2010}{2010}). Recent
landings (1981-2016) were obtained from PacFIN (retrieval dated May 2,
2017, Pacific States Marine Fisheries Commission, Portland, Oregon;
www.psmfc.org).

\textbf{At-Sea Hake Fishery}

Catches of Pacific ocean perch are monitored aboard the vessel by
observers in the at-sea hake Observer program (ASHOP) and were available
for the years of 1975-2016. Observers use a spatial sample design, based
on weight, to randomly choose a portion of the haul to sample for
species composition. For the last decade, this is typically 30-50\% of
the total weight. The total weight of the sample is determined by all
catch passing over a flow scale. All species other than hake are removed
and weighed by species on a motion compensated flatbed scale. Observers
record the weights of all non-hake species. Non-hake species total
weights are expanded in the database by using the proportion of the haul
sampled to the total weight of the haul. The catches of non-hake species
in unsampled hauls is determined using bycatch rates determined from
sampled hauls. Since 2001, more than 97\% of the hauls have been
observed and sampled.

\textbf{Foreign Catches}

From the 1960s through the early 1970s, foreign trawling enterprises
harvested considerable amounts of rockfish off Washington and Oregon,
and along with the domestic trawling fleet, landed large quantities of
Pacific ocean perch. Foreign catches of individual species were
estimated by Rogers (\protect\hyperlink{ref-rogers_species_2003}{2003})
and attributed to INPFC areas for the years of 1966-1976 for Pacific
ocean perch. The foreign catches were combined across areas for a
coastwide removal total.

\subsubsection{Discards}\label{discards}

Data on discards of Pacific ocean perch are available from two different
data sources. The earliest source is referred to as the Pikitch data and
comes from a study organized by Ellen Pikitch that collected trawl
discards from 1985-1987 (Pikitch et al.
\protect\hyperlink{ref-pikitch_evaluation_1988}{1988}). The northern and
southern boundaries of the study were \(48^\circ 42^\prime\) N latitude
and \(42^\circ 60^\prime\) N latitude respectively, which is primarily
within the Columbia INPFC area (Pikitch et al.
\protect\hyperlink{ref-pikitch_evaluation_1988}{1988}, Rogers and
Pikitch \protect\hyperlink{ref-rogers_numerical_1992}{1992}).
Participation in the study was voluntary and included vessels using
bottom, midwater, and shrimp trawl gears. Observers of normal fishing
operations on commercial vessels collected the data, estimated the total
weight of the catch by tow, and recorded the weight of species retained
and discarded in the sample. Results of the Pikitch data were obtained
from John Wallace (personal communication, NWFSC, NOAA) in the form of
ratios of discard weight to retained weight of Pacific ocean perch and
sex-specific length frequencies. Discard estimates are shown in Table
\ref{tab:Discard}.

The second source is from the West Coast Groundfish Observer Program
(WCGOP). This program is part of the NWFSC and has been recording
discard observations since 2003. Table \ref{tab:Discard} shows the
discard ratios (discarded/(discarded + retained)) of Pacific ocean perch
from WCGOP. Since 2011, when the trawl rationalization program was
implemented, observer coverage rates increased to nearly 100\% for all
the limited entry trawl vessels in the program and discard rates
declined compared to pre-2011 rates. Discard rates were obtained for
both the catch-share and the non-catch share sector for Pacific ocean
perch. A single discard rate was calculated by weighting discard rates
based on the commercial landings by each sector. Coefficient of
variations were calculated for the non-catch shares sector and pre-catch
share years by bootstrapping vessels within ports because the observer
program randomly chooses vessels within ports to be observed. Post-ITQ,
all catch-share vessels have 100\% observer coverage and discarding is
assumed to be known. Discard length composition for the trawl fleet
varied by year, with larger fish being discarded prior to 2011 (Figure
\ref{fig:WCGOP_discard}).

\subsubsection{Fishery Length and Age
Data}\label{fishery-length-and-age-data}

Biological data from commercial fisheries that caught Pacific ocean
perch were extracted from PacFIN on May 4, 2017. Lengths taken during
port sampling in Oregon and Washington were used to calculate length and
age compositions. There were no biological data from California for
Pacific ocean perch available within PacFIN or CALCOM databases. The
overwhelming majority of these data were collected from the mid-water
and bottom trawl gear, but additional biological data were collected
from non-trawl gear which was grouped together with trawl gear data.
Tables \ref{tab:Comm_Lengths} and \ref{tab:Comm_Ages} show the number of
trips and fish sampled, along with the calculated sample sizes. Length
and age data were acquired at the trip level and then aggregated to the
state level. The input sample sizes were calculated via the Stewart
method (Ian Stewart, personal communication, IPHC):

\begin{centering}

Input effN = $N_{\text{trips}} + 0.138 * N_{\text{fish}}$ if $N_{\text{fish}}/N_{\text{trips}}$ is $<$ 44

Input effN = $7.06 * N_{\text{trips}}$ if $N_{\text{fish}}/N_{\text{trips}}$ is $\geq$ 44

\end{centering}

The fishery fleet observed Pacific ocean perch that were generally
greater than 30 cm across all years of available data (Figure
\ref{fig:Comm_Length}). The fishery fleet age data has clear patterns
showing a large cohort moving through the population (Figure
\ref{fig:Comm_Age}). Lengths and ages were also available for the at-sea
hake fishery and are shown in Figures \ref{fig:ASHOP_Length} and
\ref{fig:ASHOP_Age}.

\subsubsection{Fishery Data Not Included in the Base
Model}\label{fishery-data-not-included-in-the-base-model}

Several dataset available from the fishery were explored but not used in
the final assessment.

\paragraph{Historical Commercial Catch-Per-Unit
Effort}\label{historical-commercial-catch-per-unit-effort}

Data on catch-per-unit-effort (CPUE) in mt/hr from the domestic fishery
were combined for the INPFC Vancouver and Columbia areas from Gunderson
(\protect\hyperlink{ref-gunderson_population_1977}{1977}). Although
these data reflect catch rates for the US fleet, the highest catch rates
coincided with the beginning of removals by the foreign fleet. This
suggests that, barring unaccounted changes in fishing efficiency during
this period, the level of abundance was high at that time.
Unfortunately, the original data and the analysis methods used to create
this CPUE series have been lost to time precluding a re-analysis of
these data. Due to the inability to examine the assumptions made during
the original analysis or the data used this time-series has been
excluded from the base model. These data were included in the previous
assessment but were deemed not influential in the model estimates.
Information regarding the fishery CPUE are available in Appendix C,
section \ref{appendix-c.-description-of-cpue-and-triennial-data}.

\paragraph{Oregon Special Projects Length and Age
Data}\label{oregon-special-projects-length-and-age-data}

Oregon special project data were provided by ODFW. These data represent
samples made at either the dock or at processing plants from fishery
landings. Length data were collected primarily from 1970-1986, with
limited samples from more recent years. Age data were primarily
available from 1981-1984. These data were collected for special projects
and may not have been sampled randomly from the fishery landings. Due to
these concerns, these data were not included in the base model but were
included in a model sensitivity. This was the first time these data were
explored for consideration in the assessment.

\subsection{Biological Data}\label{biological-data}

\subsubsection{Natural Mortality}\label{natural-mortality}

Historical Pacific ocean perch ages determined using scales and surface
reading methods of otoliths resulted in estimates of natural mortality
(\(M\)) between 0.10 and 0.20 yr\textsuperscript{-1} with a longevity
less than 30 years (Gunderson
\protect\hyperlink{ref-gunderson_population_1977}{1977}). Based on
break-and-burn method of age determination using otoliths, the maximum
age of Pacific ocean perch was revised to be 90 years (Chilton and
Beamish \protect\hyperlink{ref-chilton_age_1982}{1982}). The updated
understanding concerning Pacific ocean perch longevity reduced the
estimate of natural mortality based on Hoenig's
(\protect\hyperlink{ref-hoenig_empirical_1983}{1983}) relationship to
0.059 yr\textsuperscript{-1}. The previous assessment applied a prior
distribution on natural mortality based upon multiple life-history
correlates (including Hoenig's method, Gunderson
(\protect\hyperlink{ref-gunderson_trade-off_1997}{1997}) gonadosomatic
index, and McCoy and Gillooly's
(\protect\hyperlink{ref-mccoy_predicting_2008}{2008}) theoretical
relationship) developed separately for female and male Pacific ocean
perch.

Hamel (\protect\hyperlink{ref-hamel_method_2015}{2015}) developed a
method for combining meta-analytic approaches relating the \(M\) rate to
other life-history parameters such as longevity, size, growth rate, and
reproductive effort to provide a prior on \(M\). In that same issue of
\emph{ICES Journal of Marine Science}, Then et al.
(\protect\hyperlink{ref-then_evaluating_2015}{2015}) provided an updated
data set of estimates of \(M\) and related life history parameters
across a large number of fish species from which to develop an \(M\)
estimator for fish species in general. They concluded by recommending
\(M\) estimates be based on maximum age alone, based on an updated
Hoenig non-linear least squares estimator \(M=4.899A^{-0.916}_{max}\).
The approach of basing \(M\) priors on maximum age alone was one that
was already being used for West Coast rockfish assessments. However, in
fitting the alternative model forms relating \(M\) to
\(A_{\text{max}}\), Then et al.
(\protect\hyperlink{ref-then_evaluating_2015}{2015}) did not
consistently apply their transformation. In particular, in real space,
one would expect substantial heteroscedasticity in both the observation
and process error associated with the observed relationship of \(M\) to
\(A_{\text{max}}\). Therefore, it would be reasonable to fit all models
under a log transformation. This was not done. Re-evaluating the data
used in Then et al. (\protect\hyperlink{ref-then_evaluating_2015}{2015})
by fitting the one-parameter \(A_{\text{max}}\) model under a log-log
transformation (such that the slope is forced to be -1 in the
transformed space (Hamel
\protect\hyperlink{ref-hamel_method_2015}{2015})), the point estimate
for \(M\) is:

\begin{centering}

$M=\frac{5.4}{A_{\text{max}}}$

\end{centering}

The above is also the median of the prior. The prior is defined as a
lognormal distribution with mean \(ln(5.4/A_{\text{max}})\) and SE =
0.438. Using a maximum age of 100, the point estimate and median of the
prior is 0.054 yr\textsuperscript{-1}. The maximum age was selected
based on available age data from all West Coast data sources. The oldest
aged rockfish was 120 years, captured by the commercial fishery in 2007.
However, age data are subject to ageing error which could impact this
estimate of longevity. The selection of 100 years was based on the range
of other ages available with multiple observations of fish between 90
and 102 years of age.

\subsubsection{Sex Ratio, Maturation, and
Fecundity}\label{sex-ratio-maturation-and-fecundity}

Examining all biological data sources, the sex ratio of young fish are
within 5\% of 1:1 by length until larger sizes which are dominated by
females who reach a larger maximum size relative to males (Figure
\ref{fig:sexratio}), with the sex ratio being approximately equal across
ages (Figure \ref{fig:sexratio_Age}), and hence this assessment assumed
the sex ratio at birth was 1:1. This assessment assumed a logistic
maturity-at-length curve based on analysis of 537 fish maturity samples
collected from the NWFSC shelf-slope survey. This is revised from the
previous assessment that assumed maturity-at-age based on the work of
Hannah and Parker
(\protect\hyperlink{ref-hannah_age-modulated_2007}{2007}). Additionally,
the new maturity-at-length curve is based on the estimate of functional
maturity, an approach that classifies rockfish maturity with developing
oocytes as mature or immature based on the proportion of vitellogenin in
the cytoplasm and the measured frequency of atretic cells (Melissa Head,
personal communication, NWFSC, NOAA). The 50\% size-at-maturity was
estimated at 32.1 cm with maturity asymptoting to 1.0 for larger fish
(Figure \ref{fig:mat}). Comparison between the maturity-at-age used in
the previous assessment and the updated functional maturity-at-length is
shown in Figure \ref{fig:mat_compare}.

The fecundity-at-length has also been updated from the previous
assessment based on new research. Dick et al.
(\protect\hyperlink{ref-dick_meta-analysis_2017}{2017}) estimated new
fecundity relationships for select West Coast stocks where fecundity for
Pacific ocean perch was estimated equal to
8.66e-10\(L\)\textsuperscript{4.98} in millions of eggs where \(L\) is
length in cm. Fecundity-at-length is shown in Figure
\ref{fig:fecundity}.

\subsubsection{Length-Weight
Relationship}\label{length-weight-relationship}

The length-weight relationship for Pacific ocean perch was estimated
outside the model using all biological data available from
fishery-dependent and -independent data sources, where the female
weight-at-length in grams was estimated at
1.003e-05\(L\)\textsuperscript{3.1} and males at
9.881e-06\(L\)\textsuperscript{3.1} where \(L\) is length in cm (Figures
\ref{fig:Wt_len} and \ref{fig:Wt_len_pred}).

\subsubsection{Growth (Length-at-Age)}\label{growth-length-at-age}

The length-at-age was estimated for male and female Pacific ocean perch
using data collected from both fishery-dependent and -independent data
sources that were collected from 1981-2016. Figure \ref{fig:Len_Age}
shows the lengths and ages for all years and all data as well as
predicted von Bertalanffy fits to the data. Females grow larger than
males and sex-specific growth parameters were estimated at the following
values:

\begin{centering}

Females $L_{\infty}$ = 42.32; $k$ = 0.169; $t_0$ = -1.466

Males $L_{\infty}$ = 39.03; $k$ = 0.212; $t_0$ = -1.02

\end{centering}

These values were used as starting parameter values within the base
model prior to estimating each parameter for male and female Pacific
ocean perch.

\subsubsection{Ageing Precision and
Bias}\label{ageing-precision-and-bias}

Uncertainty surrounding the age-reading error process for Pacific ocean
perch was incorporated by estimating ageing error by age.
Age-composition data used in the model were from break-and-burn otolith
reads aged by the Cooperative Ageing Project (CAP) in Newport, Oregon.
Break-and-burn double reads of more than 1500 otoliths were provided by
the CAP lab. An ageing-error estimate was made based on these double
reads using a computational tool specifically developed for estimating
ageing error (Punt et al.
\protect\hyperlink{ref-punt_quantifying_2008}{2008}) and using release
1.0.0 of the R package nwfscAgeingError (Thorson et al.
\protect\hyperlink{ref-thorson_nwfscageingerror:_2012}{2012}) for input
and output diagnostics, publicly available at:
\url{https://github.com/nwfsc-assess/nwfscAgeingError}. A non-linear
standard error was estimated by age, where there is more variability in
the age of older fish (Table \ref{tab:Age_Error} and Figure
\ref{fig:Age_Error}).

\subsection{History of Modeling Approaches Used for This
Stock}\label{history-of-modeling-approaches-used-for-this-stock}

\subsubsection{Previous Assessments}\label{previous-assessments}

The status of Pacific ocean perch off British Columbia, Washington, and
Oregon have been periodically assessed since the intensive exploitation
that occurred in the 1960s. Concerns regarding Pacific ocean perch
status off the coast the US west coast were raised in the late 1970s
(Gunderson \protect\hyperlink{ref-gunderson_results_1978}{1978},
\protect\hyperlink{ref-gunderson_updated_1981}{1981}) and in 1981 the
PFMC adopted a 20-year plan to rebuild the stock.

The 1992 assessment determined that Pacific ocean perch remained at low
levels relative to the population size in 1960 (Ianelli et al.
\protect\hyperlink{ref-ianelli_status_1992}{1992}) and recommended
additional harvest restrictions to allow for stock rebuilding. The 1998
assessment (Ianelli and Zimmermann
\protect\hyperlink{ref-ianelli_status_1998}{1998}) estimated that the
stock was 13\% of the unfished level, leading the National Marine
Fishery Service (NMFS) to declare the stock overfished in 1999. A formal
rebuilding plan was implemented in 2001. The rebuilding plan reduced the
SPR harvest rate used to determine catches to 0.864 (in contrast to the
default harvest rate of 0.50). The last full assessment of Pacific ocean
perch was conducted in 2011 (Hamel and Ono
\protect\hyperlink{ref-hamel_stock_2011}{2011}), which concluded that
the stock was still well below the target biomass of \(40\%SB_{0}\)
estimating the relative stock status at 19.1\%.

\section{Assessment}\label{assessment}

\subsection{General Model Specifications and
Assumptions}\label{general-model-specifications-and-assumptions}

Stock Synthesis version 3.30.03.05 was used to estimate the parameters
in the model. R4SS, version 1.27.0, along with R version 3.3.2 were used
to investigate and plot model fits. A summary of the data sources used
in the model (details discussed above) is shown in Figure
\ref{fig:data_plot}.

\subsubsection{Changes Between the 2011 Assessment Model and Current
Model}\label{changes-between-the-2011-assessment-model-and-current-model}

The current model for Pacific ocean perch has many similar assumptions
as the 2011 assessment but differs in some key ways. In this assessment,
fleets were disaggregated into a trawl/other gear, at-sea hake,
historical foreign fleet, and research fleets. The previous assessment
implemented a single fleet where removals from all sources were
aggregated together. The separating of fleets applied in this assessment
allowed for differing assumptions regarding current and historical
discarding practices. Although there are no compositional data available
from the foreign fleet, it is assumed that very little to no discarding
of fish occurred. Additionally, the at-sea hake fishery removals
represent both discarded and retained fish and hence an additional
discard rate would not be appropriate. Similar logic was applied in
regard to survey removals.

The historical landings used in the model differ from those used in
2011. This assessment includes the first state provided historical
reconstruction landings for Washington. The historical reconstruction
has removals starting in 1908 and has larger removals in the 1940s
relative to those used in the 2011 assessment (Figure
\ref{fig:Catch_Compare}). The starting year for modeling the stock was
revised to 1918, the first year Pacific ocean perch landings exceeded 1
mt, rather than 1940 as modeled in the previous assessment, given the
new information regarding historical removals prior to 1940.
Explorations were conducted relative to the model starting year and no
differences were found between the 1918 start year compared to starting
the model in 1892, which is the first year there is record landings of
Pacific ocean perch between California, Oregon, and Washington.

Selectivity in this model is assumed to be length-based and is modeled
using double-normal selectivity for all fleets, except the Pacific ocean
perch survey which retained the assumption used in previous assessment
of logistic selectivity. The previous assessment mirrored selectivity
among the Pacific ocean perch and both slope surveys (AFSC and NWFSC).
This assessment allows for survey-specific selectivity.

All fishery-independent indices have been re-evaluated for this
assessment using a spatial-temporal delta generalized linear mixed model
(VAST delta-GLMM) which is an updated approach from that used in 2011,
which did not incorporate spatial effects. This assessment opted to not
include the fishery CPUE and the Triennial shelf index and composition
data based upon discussions during the STAR panel. The data used to
create the CPUE index were not available for reanalysis and hence were
excluded from this assessment due to questions regarding this index that
could not be addressed. In regards to the Triennial survey, Pacific
ocean perch is considered a slope species off the US west coast and this
survey did not sample the prime habitat for Pacific ocean perch and had
limited observations relative to the other surveys. It was concluded
during the STAR panel that this data set was not a good source of
information regarding this species and would not be included in the base
model.

Maturity and fecundity were updated for this assessment based upon new
research. Fecundity for Pacific ocean perch used in this assessment was
based on a re-evaluation of the fecundity of West Coast rockfish by Dick
et al. (\protect\hyperlink{ref-dick_meta-analysis_2017}{2017}), updating
the previous fecundity estimates used in the 2011 assessment (Dick
\protect\hyperlink{ref-dick_modeling_2009}{2009}) (Figure
\ref{fig:fecundity}). Maturity in this assessment was based on
examination of 537 fish samples which were used to estimate functional
maturity, an approach that classifies rockfish maturity with developing
oocytes as mature or immature based on the proportion of vitellogenin in
the cytoplasm and the measured frequency of atretic cells (Melissa Head,
personal communication, NWFSC, NOAA). The updated maturity curve was
based on maturity-at-length where the previous estimates used in 2011
were based on maturity-at-age (Figure \ref{fig:mat_compare}).

In this assessment, the beta prior developed from a meta-analysis of
West Coast groundfish was updated to the 2017 value (James Thorson,
personal communication, NWFSC, NOAA) in preliminary models, with
steepness fixed at an alternative value in the final base model. The
estimated spawning output, relative stock status, and model diagnostics
in preliminary models using the steepness prior were deemed unrealistic
(e.g.~estimated near unfished conditions with low catchability by the
NWFSC shelf-slope survey). Steepness was fixed in the base model at the
value corresponding to the median spawning output resulting from
steepness values ranging from 0.25 - 0.95. Additionally, the prior for
natural mortality was updated based on an analysis conducted by Owen
Hamel (personal communication, NWFSC, NOAA), where female and male
natural mortality were fixed at the median of the prior (0.054
yr\textsuperscript{-1}).

\subsubsection{Summary of Fleets and
Areas}\label{summary-of-fleets-and-areas}

Pacific ocean perch are most frequently observed in Oregon and
Washington waters in survey and fishery observations. Multiple fisheries
encounter Pacific ocean perch. Bottom trawl, mid-water trawl, fixed
gear, and the at-sea (mid-water) hake fisheries account for the majority
of the current Pacific ocean perch landings.

The majority of removals of Pacific ocean perch are attributable to
trawl gears with fixed gear accounting for a small fraction of the
catches available within PacFIN. Trawl and fixed gears were combined
into a coast-wide fleet. For the period from 1918 to the early 1990s,
prior to the introduction of trip limits for rockfish, limited
discarding of Pacific ocean perch was assumed. Observations of Pacific
ocean perch in the Pikitch et al.
(\protect\hyperlink{ref-pikitch_evaluation_1988}{1988}) data (1986-1987)
allowed for a formal analysis of discard rates that were applied to the
historical period of the fishery. Foreign trawl catches (1966-1976) were
modeled as a single fleet. The at-sea hake fishery operates as a
mid-water fishery targeting Pacific whiting but encounters Pacific ocean
perch as a bycatch species. This fleet was also modeled as a single
fleet.

\subsubsection{Other Specifications}\label{other-specifications}

The specifications of the assessment are listed in Table
\ref{tab:Model_setup}. The model is a two-sex, age-structured model
starting in 1918 with an accumulated age group at 60 years. Growth and
natural mortality were assumed time invariant with a constant growth
estimated and natural mortality fixed at the median of the prior. The
lengths in the population were tracked by 1 cm intervals and the length
data were binned into 1 cm intervals. A curvilinear ageing imprecision
relationship was estimated and used to model ageing error.
Fecundity-at-length was fixed at the values from Dick et al.
(\protect\hyperlink{ref-dick_meta-analysis_2017}{2017}) for Pacific
ocean perch and spawning output was defined in millions of eggs.

Age data for the commercial and at-sea hake fisheries, as well as the
Pacific ocean perch, the NWFSC slope, and the NWFSC shelf-slope surveys
were used in this assessment. The ages from the NWFSC shelf-slope survey
were entered into the model as conditional age-at-length. The assessment
used length-frequencies collected by the fishery fleet, the at-sea hake
fishery, and Pacific ocean perch, AFSC slope, NWFSC slope, and the NWFSC
shelf-slope surveys.

The specification of when to estimate recruitment deviations is an
assumption that likely affects model uncertainty. Recruitment deviations
were estimated from 1900-2014 to appropriately quantify uncertainty. The
earliest length-composition data occur in 1966 and the earliest age data
were in 1981. The most informed years for estimating recruitment
deviations were from about the mid-1970s to 2013. The period from
1900-1974 was fit using an early series with little or no bias
adjustment, the main period of recruitment deviates occurred from
1975-2014 with an upward and downward ramping of bias adjustment (Figure
\ref{fig:bias_ramp}), and 2015 onward were fit using forecast
recruitment deviates with no bias adjustment. Methot and Taylor
(\protect\hyperlink{ref-methot_adjusting_2011}{2011}) summarize the
reasoning behind varying levels of bias adjustment based on the
information available to estimate the deviates. The standard deviation
of recruitment variability was assumed to be 0.70 based on the estimated
variation in recruitment from the base model.

The recommended selectivity in Stock Synthesis is the double-normal
parameterization and it was used in this assessment for the all fleets,
except the Pacific ocean perch survey, which was assumed logistic based
on the length-composition data. Changes in retention curves were
estimated for the fishery fleet.

Time blocks for the fishery fleet are provided in Table
\ref{tab:Model_setup}. Fishery selectivity and retention has changed
over the modeled period due to management changes. The time block for
fishery selectivity was set from 1918-1999 and 2000-2017 based on
changes in selectivity arising from the overfished declaration. The time
blocks on the retention curves for the fishery were set from 1918-1991,
1992-2001, 2002-2007, 2008, 2009-2010, 2011-2016 based on available
discarding data and changes in trip limits that likely resulted in
changes to discarding patterns of Pacific ocean perch. No discarding was
assumed in the at-sea hake and the foreign fisheries. The length data
are not available from the foreign fleet. The selectivity from this
fleet was mirrored to the main fishery fleet.

The following distributions were assumed for data fitting: survey
indices were lognormal, total discards were lognormal, and multinomial
error structure for compositional data.

\subsubsection{Modeling Software and Model
Bridging}\label{modeling-software-and-model-bridging}

The STAT team used Stock Synthesis version 3.30.03.05 developed by
Dr.~Richard Methot at the NWFSC (Methot and Wetzel
\protect\hyperlink{ref-methot_stock_2013}{2013}). This most recent
version was used because it included improvements and corrections to
older versions. The previous assessment of Pacific ocean perch also used
Stock Synthesis but an earlier version, 3.24; model bridging was
performed between both versions of Stock Synthesis and are shown in
Figure \ref{fig:bridge}.

\subsubsection{Priors}\label{priors}

A prior distribution was developed for natural mortality (\(M\)) from an
analysis based on an assumed maximum age of 100 years. The analysis was
performed by Owen Hamel (personal communications, NWFSC, NOAA) and used
data from Then et al.
(\protect\hyperlink{ref-then_evaluating_2015}{2015}) to provide a
lognormal distribution for natural mortality. The lognormal prior has a
median of 0.054 and a standard error of 0.438.

The prior for steepness (\(h\)) assumed a beta distribution with
parameters based on an update of the Thorson-Dorn rockfish prior
(commonly used in past West Coast rockfish assessments) conducted by
James Thorson (personal communication, NWFSC, NOAA) which was reviewed
and endorsed by the Scientific and Statistical Committee (SSC) in 2017.
The prior is a beta distribution with \(\mu\)=0.72 and \(\sigma\)=0.15.
However, fixing steepness within the model resulted in what was deemed
to be on the verge of implausible catchability for the NWFSC shelf-slope
survey. The Groundfish Subcommittee of the SSC (GFSC) recommended fixing
steepness, determined by the expected (i.e.~arithmetic mean) spawning
output value across a range of steepness values (see Appendix D, section
\ref{appendix-d.-ssc-groundfish-subcommittee-discussion-regarding-steepness}
for GFSC comments). The steepness value of 0.50 most closely
corresponded with the expected spawning output and was used in the final
base model. The previous assessment estimated and fixed steepness equal
to 0.40. The current data do not contain information regarding
steepness. This change in perception is likely due to the observation of
large recruitment events in this assessment, updated data weighting
approaches, and varying model specifications between the 2011 and the
current model.

\subsubsection{Data Weighting}\label{data-weighting}

Length and age-at-length compositions from the NWFSC shelf-slope survey
were fit along with length and marginal age compositions from the
fishery and other survey fleets. Length data started with a sample size
determined from the equation listed in Sections
\ref{northwest-fisheries-science-center-nwfsc-shelf-slope-survey}
(survey data) and \ref{fishery-length-and-age-data} (fishery data). It
was assumed for age-at-length data that each age was a random sample
within the length bin and the model started with a sample size equal to
the number of fish in that length bin.

One extra variability parameter was estimated and added to the input
variance for the NWFSC shelf-slope survey index. Estimating additional
variance for the other surveys was explored and determined to not be
required. WCGOP data were bootstrapped to provide uncertainty of the
total discards (Table \ref{tab:Discard}).

The base assessment model was weighted using the ``Francis method'',
which was based on equation TA1.8 in Francis
(\protect\hyperlink{ref-francis_data_2011}{2011}). This formulation
looks at the mean length or age and the variance of the mean to
determine if across years, the variability is explained by the model. If
the variability around the mean does not encompass the model
predictions, then that data source should be down-weighted. This method
accounts for correlation in the data (i.e., the multinomial
distribution) as opposed to the McAllister and Ianelli
(\protect\hyperlink{ref-mcallister_bayesian_1997}{1997}) method
(Harmonic Mean weighting) of looking at the difference between
individual observations and predictions. A sensitivity was performed
examining the difference between the weighting approaches. The weights
applied to each length and age data set for the base model are shown in
Table \ref{tab:francis}.

\subsubsection{Estimated and Fixed
Parameters}\label{estimated-and-fixed-parameters}

There were 163 estimated parameters in the base model. These included
one parameter for \(R_0\), 8 parameters for growth, 1 parameters for
extra variability for the NWFSC shelf-slope survey index, 24 parameters
for selectivity, retention, and time blocking of the fleets and the
surveys, 117 recruitment deviations, and 12 forecast recruitment
deviations (Table \ref{tab:model_params}).

Fixed parameters in the model were as follows. Steepness was fixed at
0.50. A sensitivity analysis and a likelihood profile were performed for
steepness. Natural mortality was fixed at 0.054 yr\textsuperscript{-1}
for females and males, which is the median of the prior. The standard
deviation of recruitment deviates was fixed at 0.70. Maturity-at-length
was fixed as described in Section
\ref{sex-ratio-maturation-and-fecundity}. Length-weight parameters were
fixed at estimates using all length-weight observations (Figure
\ref{fig:Wt_len_pred}).

Dome-shaped selectivity was explored for all fleets within the model.
Older Pacific ocean perch are often found in deeper waters and may move
into areas that limit their availability to fishing gear, especially
trawl gear. The final base model estimated dome-shaped selectivity for
only the fishery. The selectivties for the the at-sea hake fishery and
all surveys were estimated asymptotic.

\subsection{Model Selection and
Evaluation}\label{model-selection-and-evaluation}

The base assessment model for Pacific ocean perch was developed to
balance parsimony and realism, and the goal was to estimate a spawning
output trajectory for the population of Pacific ocean perch off the west
coast of the US. The model contains many assumptions to achieve
parsimony and uses many different sources of data to estimate reality. A
series of investigative model runs were done to achieve the final base
model.

\subsubsection{Key Assumptions and Structural
Choices}\label{key-assumptions-and-structural-choices}

The key assumptions in the model were that the assessed population is a
single stock with biological parameters characterizing the entire coast;
natural mortality, maturity-at-length, length-at-age, and
weight-at-length have remained constant over the period modeled; the
standard deviation in recruitment deviation is 0.70; and steepness is
0.50. These are simplifying assumptions that unfortunately cannot be
verified or disproved. Sensitivity analyses were conducted for most of
these assumptions to determine their effect on the results.

Structurally, the model assumed that the landings from each fleet were
representative of the coastwide population, instead of specific areas,
and fishing mortality prior to 1918 was negligible. It also assumed that
discards were low prior to 1992.

\subsubsection{Alternate Models
Considered}\label{alternate-models-considered}

The exploration of models began by bridging from the 2011 assessment to
Stock Synthesis version 3.30.03.05, which produced no discernible
difference (Figure \ref{fig:bridge}). The updated landings data and
discard rates added to the 2011 assessment produced insignificant
differences in the relative scale of the population although the updated
historical removals resulted in an increase in the estimate of unfished
spawning output (Figure \ref{fig:data_update}). Updating the survey
indices produced small differences in the relative scale of the
population. Adding age and length data each resulted in less of a
population decline from the 1970s to pre-2000, resulting in an increase
in the estimated 2017 final stock status. However, the addition of new
data resulted in an early pattern within recruitment, indicating that
the assumptions within the previous model may not represent the best fit
to the current data.

This assessment estimated discards in the model, so time was spent
investigating time blocks for changes in selectivity and retention to
match the discard data as best as possible. Using major changes in
management and observed changes in landings, a set of blocks for
retention were determined for the fishery fleet. In the spirit of
parsimony, as few blocks as possible were used by only allowing blocks
during time periods with data or when we felt they were justified by
changes in management.

Natural mortality was also investigated and a new prior was developed
assuming a maximum age of 100 years for females and males. The previous
assessment estimated male natural mortality as an offset from a fixed
female natural mortality. This assessment attempted to estimate natural
mortality for both sexes using the 2017 updated prior, but based on the
results of profiles over the parameter it was determined to be little to
no information on natural mortality within the data and hence opted to
fix the value for females within the base model. Upon additional
exploration, the model estimated very little difference in male natural
mortality relative to females (\textless{} 0.002) and in the interest of
selecting the model that fit the data with the fewest parameters
required, natural mortality for males was fixed equal to the female
natural mortality in the base model.

Finally, multiple models were investigated where steepness was either
estimated, fixed at the prior, or at an alternate value. The assessment
in 2011 determined that there was sufficient information concerning
steepness where the parameter was estimated and then fixed at the
estimated value of 0.40. Based upon likelihood profiles performed on the
current model, there was no longer support for a steepness value of
0.40. The likelihood profile was flat across various levels of steepness
with a very small improvement in likelihood (\textless{}0.50 log
likelihood units) at the lowest steepness values. Estimating steepness
starting at the mean of the ``type C'' prior, the meta-analysis prior
evaluated omitting information from Pacific ocean perch, of 0.76
resulted in very little if any movement from the mean value due to the
flat likelihood surface across values for this parameter with the final
relative stock status for 2017 being estimated to \textgreater{} 100\%
of unfished spawning output. The base model fixed steepness at 0.50,
determined by calculating current ending spawning outputfor steepness
values ranging from 0.25 to 0.95 in increments of 0.05 and assuming each
value to be equally plausible. The expected (i.e arithmetic mean)
spawning output was identified and the most closely corresponding
steepness of 0.50 was selected for use in the final base model. Model
sensitivities are provided when steepness was fixed at the 2011 value of
0.40 or when fixed at the mean of the current prior of 0.72.

\subsubsection{Convergence}\label{convergence}

Proper convergence was determined by starting the minimization process
from dispersed values of the maximum likelihood estimates to determine
if the model found a better minimum. Starting parameters were jittered
by 10\%. This was repeated 50 times and a better minimum was not found
(Table \ref{tab:jitter}). The model did not experience convergence
issues when provided reasonable starting values. Through the jittering
done as explained above and likelihood profiles, we are confident that
the base model as presented represents the best fit to the data given
the assumptions made. There were no difficulties in inverting the
Hessian to obtain estimates of variability, although much of the early
model investigation was done without attempting to estimate a Hessian.

\subsection{STAR Panel Review and
Recommendations}\label{star-panel-review-and-recommendations}

\subsection{Response to the 2011 STAR Panel
Recommendations}\label{response-to-the-2011-star-panel-recommendations}

Recommendation: Considering trans-boundary stock effects should be
pursued. In particular, the consequences of having spawning
contributions from external stock components should be evaluated
relative to the steepness estimates obtained in the present assessment.

\emph{STAT response: The STAT team agrees that this should be an ongoing
area of research and collaboration between the US and Canada. This
assessment presents a sensitivity where the inclusion of Canadian data
are included within the model.}

Recommendation: The benefits of adopting the complex model used this
year should be evaluated relative to simpler assumptions and models.
While the transition from the simpler old model to Stock Synthesis was
shown to be similar for the historical period, the depletion estimates
in the most recent years were different enough to warrant further
investigation.

\emph{STAT response: This assessment was performed in Stock Synthesis,
an integrated model, which can be modified to either simple or complex
structural forms based upon the available data and the processes being
modeled. There were not addtional explorations of alternative modeling
platforms.}

Recommendation: Discard estimates from observer programs should be
presented, reviewed (similar to the catch reconstructions), and be made
available to the assessment process.

\emph{STAT response: This assessment uses discard rates and discard
lengths collected by the WCGOP from 2003-2015.}

Recommendation: The ability to allow different ``plus groups'' for
specific data types should be evaluated (and implemented in Stock
Synthesis). For example, this would provide the ability to use the
biased surface-aged data in an appropriate way.

\emph{STAT response: The STAT team agrees that this should be explored,
but additional research needs to completed which evaluates the amount of
bias and imprecision in surface-read ages. Evaluating available
surface-read ages within the PacFIN database fish of lengths between
23-44 cm can be aged at 10 years old. This large range of lengths at the
same age indicates considerable bias in ages for fish surface-read
younger aged fish.}

Recommendation: Historical catch reconstruction estimates should be
formally reviewed prior to being used in assessments and should be
coordinated so that interactions between stocks are appropriately
treated. The relative reliability of the catch estimates over time could
provide an axis of uncertainty in future assessments.

\emph{STAT response: California and Oregon have undergone extensive work
to create historical catch reconstructions. This is the first assessment
for Pacific ocean perch which includes a Washington historical catch
reconstruction. The data used in this assessment represent Washington
state's current best estimate for historical catches. An historical
catch reconstruction meeting was held in November of 2016 where states
discussed methods and approaches to improve historical catch estimates.
Additionally, both California and Washington are conducting research to
estimate uncertainty surrounding historical catches which could be used
to propegate uncertainty within the assessment.}

\subsection{Response to the 2017 STAR Panel
Requests}\label{response-to-the-2017-star-panel-requests}

The stock assessment review (STAR) panel for this assessment was held at
the NWFSC in Seattle, WA from June 26-30, 2017. David Sampson was the
chair, while Norman Hall, Kevin Piner, and Yiota Apostolaki were invited
reviewers. It was a productive and busy review that thoroughly reviewed
many facets of the assessments. As mentioned above, changes to the data
used in this assessment were made during the panel.

Recommendation: Further investigation of Pacific ocean perch stock
structure is recommended. One approach would be to look for correlations
of US west coast recruitment deviations and survey biomass estimates
with corresponding results from Pacific ocean perch assessments in
Canada and Gulf of Alaska.

\emph{STAT response: We agree. A preliminary analysis using a subset of
Canadian data was provided as a sensitivity, but further investigations
should be conducted.}

STAR Recommendation: The next iteration of this assessment could be an
update assessment.

GFSC Recommendation: Given the considerable uncertainty associated with
the assessment, the GFSC recommends that the next assessment be a full
assessment.

\emph{STAT response: We agree with the GFSC recommmendation.}

Additionally, a number of general recommendations were made for all West
Coast assessments:

Recommendation: Comprehensively evaluate the appropriateness of using
the Triennial survey in assessments for other rockfish species and
whether the survey should be split into early and late segments. The
lingcod assessment reviewed during this STAR split the Triennial survey
into separate early and late surveys, whereas the draft Pacific ocean
perch assessment brought to the STAR had a single Triennial survey.

\emph{STAT response: We agree. As a whole this dataset should be
evaluated to determine which West Coast species were well sampled.
However, the treatment of keeping the data set as a single time-series
or splitting into early or late periods may need to be considered on a
species specific level. Changes in sampling range and timing may not
impact or having differeing levels of impact for West Coast species.}

Recommendation: Explore the assumption that conditional age-at-length
data are random samples of the age-composition.

\emph{STAT response: We agree. The conditional age-at-length data are
highly influential in the model. Some explorations were conducted during
the STAR panel examining the impact of these data and to determine if
the underlying assumptions of the data were violated based on age-based
processes. Further research should be conducted examining the
assumptions of these data.}

Recommendation: A standard approach for combining conditional
age-at-length sample data into annual conditional-age-at-length
compositions should be developed and reviewed. If age data are not
selected in proportion to the available lengths, simple aggregation of
the ages by length-bin may provide biased views of the overall
age-composition and year-class strength.

\emph{STAT response: We agree.}

Recommendation: Further explore the VAST approach for constructing
relative abundance indices. The upcoming workshop at the Center for the
Advancement of Population Assessment Methodology (CAPAM) will address
this issue.

\emph{STAT response: The trend of the indices created using VAST and the
Bayesian delta-glmm which did not explicitly account for spatial
dynamics were consistent with each other. However, assessments in
general will benefit, from continued research regarding the best way to
generate indices of abundance for fishery and non-fishery data.}

\subsection{Base Model Results}\label{base-model-results}

The base model parameter estimates along with approximate asymptotic
standard errors are shown in Table \ref{tab:model_params} and the
likelihood components are shown in Table \ref{tab:like}. Estimates of
derived reference points and approximate 95\% asymptotic confidence
intervals are shown in Table \ref{tab:Ref_pts}. Estimates of stock size
over time are shown in Table \ref{tab:Timeseries_mod1}.

\subsubsection{Parameter Estimates}\label{parameter-estimates}

The estimates of maximum length and the von Bertanlaffy growth
coefficient, \(k\), were less than the the external estimates for males
and female but were well within the 95\% confidence interval given the
estimated uncertainty (Table \ref{tab:model_params}, Figures
\ref{fig:sizeatage} and \ref{fig:length_compare}). Female and male
Pacific ocean perch grow quickly at younger ages, reaching near maximum
length by age 20, with female Pacific ocean perch reaching larger
maximum lengths.

Selectivity curves were estimated for the fishery and survey fleets. The
estimated selectivities for all fleets within the model are shown in
Figure \ref{fig:selex}. The fishery selectivity was estimated to be dome
shaped, reaching maximum selectivity for fish between 35 and 40 cm. An
shift in selectivity for the final asymptotic selectivity was estimated
for the fishery for prior to the overfished declaration and post
(1918-1999 and 2000-2016). The at-sea hake fishery was estimated to have
little selectivity for smaller Pacific ocean perch reaching full
selectivity at the largest sizes. The foreign fleet for which only catch
data are available was assumed to be identical to the main fishery,
although a sensitivity was performed (not shown) that mirrored the
foreign selectivity to that of the Pacific ocean perch survey
selectivity resulting in a negligible difference in stock status. Survey
selectivities were estimated to be asymptotic during model explorations
with the final selectivity forced to be asymptotic in the final base
model.

Retention curves were estimated for the fishery fleet only and were
allowed to vary based upon discard data within the model over time
(Figure \ref{fig:retention}). Historical retention was estimated to be
high and declined over time due to management restriction on landings of
Pacific ocean perch with the lowest retention occurring in 2009 and 2010
prior to the implementation of ITQs. Post-2011 retention was estimated
to be nearly 100\% for the fishery fleet.

Additional survey variability (process error added directly to each
year's input variability) for the NWFSC shelf-slope survey was estimated
within the model. The model estimated a small added variance for the
NWFSC shelf-slope survey of 0.018. Preliminary models explored
estimating added variance for each of the other indices, but resulted in
no added variance being estimated and hence the added variance
parameters were not estimated in the base model.

Estimates of recruitment suggest that the Pacific ocean perch population
is characterized by variable recruitment with occasional strong
recruitments and periods of low recruitment (Figures \ref{fig:recruits}
and \ref{fig:recdevs}). There is little information regarding
recruitment prior to 1970 and the uncertainty in those estimates is
expressed in the model. The four lowest recruitments (in ascending
order) occurred in 2012, 2003, 2005, and 2007. There are very large, but
uncertain, estimates of recruitment in 2008, 2013, 2000, and 1999. The
2008 recruitment event is estimated to be larger by an order of
magnitude compared to other recruitments estimated in the model. The
uncertainty interval around the number of recruits is large based on the
uncertainty surrounding the spawning output in that year. However, the
uncertainty around the recruitment deviation estimated is low.

\subsubsection{Fits to the Data}\label{fits-to-the-data}

There are numerous types of data for which the fits are discussed:
survey abundance indices, discard data (biomass and length
compositions), length-composition data for the fisheries and surveys,
marginal age compositions for the fisheries and surveys, and conditional
age-at-length observations for the NWFSC shelf-slope survey.

The fits to the survey indices are shown in Figure \ref{fig:index_fits}.
Extra standard error was estimated for the NWFSC shelf-slope survey. The
Pacific ocean perch survey index were fit well by the model. Both the
AFSC and NWFSC slope survey indices were generally flat and fit well by
the model. The recent NWFSC shelf-slope survey showed a variable trend
over the time period with the 2016 data point being the highest estimate
of the series and given the uncertainty around each data point (input
and model estimated added variance) the model fit fell within the
uncertainty interval for all years.

Fits to the total observed discards required time blocks (Figure
\ref{fig:discard_fits}). Fits to the trawl discards from the Pikitch
data in 1985-1987 were quite good. The change in the discard rate
modeled over 1992-2001 was based on management restrictions, which were
assumed to have increased discarding practices in the fishery fleet. The
next required time block was based on the WCGOP data from 2002-2007 and
were fit well by the model. Discarding increased prior to the
implementation of ITQs requiring blocks for 2008 and the 2009-2010
periods. The model fit the very low post-ITQ discard rates based on the
WCGOP data well. The total estimated discard amount over time is shown
in Figure \ref{fig:total_discard}.

Fits to the length data are shown based on the proportions of lengths
observed by year and the Pearson residuals-at-length for all fleets.
Detailed fits to the length data by year and fleet are provided in
Appendix A, section
\ref{appendix-a.-detailed-fit-to-length-composition-data}. Aggregate
fits by fleet are shown in Figure \ref{fig:length_agg}. There are a few
things that stand out when examining the aggregated length composition
data. First, the sexed discard lengths appear to be poorly fit by the
model but this is related to small sample sizes. The NWFSC slope survey
lengths were under estimated by the model, but these data only represent
two years.

Discard lengths from WCGOP were fit well by the model and show no
obvious pattern in the residuals (Figure \ref{fig:discard_len_pearson}).
The residuals to the fishery lengths clearly showed the growth
differential between males and females where the majority of positive
residuals at larger sizes were from female fish (Figure
\ref{fig:fishery_len_pearson}). The fishery showed large positive
residuals for smaller fish for 2013-2016 which were attributed to the
strong 2008 year class moving through the fishery. The at-sea hake
fishery did not show an obvious pattern in residuals but clearly showed
the selectivity of larger fish (Figure \ref{fig:ashop_len_pearson}). The
residuals for each of the surveys are shown in Figures
\ref{fig:pop_len_pearson}, \ref{fig:afsc_len_pearson},
\ref{fig:nwfsc_len_pearson}, and \ref{fig:nwfsc_combo_len_pearson}. The
Pearson residuals from the NWFSC shelf-slope survey clearly showed the
strong year classes moving through the population.

Length data were weighted according to the Francis weights that adjust
the weight given to a data set based on the fit to the mean lengths by
year. The mean lengths from the fishery were consistent across the
sampled period, showing only a decline in the mean length in 2013-2015
likely due to the large 2008 cohort (Figure
\ref{fig:weighting_len_fishery}). The at-sea hake fishery showed an
increase in the mean length of fish observed to 2009 and then fluctuated
at larger mean lengths thereafter (Figure
\ref{fig:weighting_len_ashop}). The mean lengths were consistent across
the two sample years of the Pacific ocean perch survey (Figure
\ref{fig:weighting_len_pop}). However, the model expected a decline in
mean length over the period. The trend in the mean lengths observed by
the AFSC slope survey was generally flat excluding the samples from 1997
which were smaller fish (Figure \ref{fig:weighting_len_afsc}). The NWFSC
slope length data from 2001 and 2002 were highly variable with differing
mean lengths between the years which were not fit well by the model
(Figure \ref{fig:weighting_len_nwfsc}). The mean length for the NWFSC
shelf-slope survey declined in 2012 and 2016 due to observations of
young, small fish by the survey (Figure
\ref{fig:weighting_len_nwfsccombo}).

Age data were fitted to as marginal age compositions for the main
fishery fleet, the at-sea hake fishery, the Pacific ocean perch survey,
and the NWFSC slope survey. The NWFSC shelf-slope ages were treated as
conditional age-at-length data to facilitate the estimation of growth
within the model. The aggregated fits to the marginal age data are shown
in Figure \ref{fig:age_agg}. The aggregated age data were fit well for
the fishery fleet which had the largest sample of ages. The at-sea hake
fishery and the surveys had significantly lower sample sizes that
resulted in spiky patterns in the aggregated data. However, the model
generally captured the pattern of the data. Detailed fits to the age
data by year and fleet are provided in Appendix B, section
\ref{appendix-b.-detailed-fit-to-age-composition-data}.

The Pearson residuals for the main fishery fleet are shown in Figure
\ref{fig:fishery_age_pearson}. There are diagonal patterns in the
residuals across years, which likely are cohorts moving through the
fishery. The at-sea hake fishery only had age data for four
non-consecutive years, combined with the tendency of this fleet to
select older fish, preventing general conclusions regarding fits to the
data and cohort strength over time (Figure \ref{fig:ashop_age_pearson}).
The Pacific ocean perch survey only had one year of age data (the 1979
were all surface reads), but both sexes had a larger observed number of
older fish relative to the model estimates (Figure
\ref{fig:pop_age_pearson}). The Pearson residuals for the two years of
age data from NWFSC slope survey are shown in Figure
\ref{fig:nwfsc_age_pearson}. The residual pattern differs between the
years and by sex with positive residuals of male fish across ages in the
2001 data.

The observed and expected conditional age-at-length fits are shown in
Figures \ref{fig:nwfsc_combo_andre_1}, \ref{fig:nwfsc_combo_andre_2},
\ref{fig:nwfsc_combo_andre_3}, \ref{fig:nwfsc_combo_andre_4}, and
\ref{fig:nwfsc_combo_andre_5} for the NWFSC shelf-slope survey
observations. The fits generally match the observations. Some outliers
are apparent with large residuals. The 2016 data varies from previous
years, where larger fish across all ages have higher observations
compared to the model expectations.

The age data were also weighted according to Francis weighting which
adjusted the weight given to a data set based on the fit to the mean age
by year. The mean ages from the fishery appear to have declined in
recent years which could be due to incoming cohorts (Figure
\ref{fig:weighting_fishery}). The at-sea hake fishery mean ages are
similar for 2006 and 2007, but both 2003 and 2014 have lower average
ages in the samples (Figure \ref{fig:weighting_ashop}). The NWFSC slope
had a decline in the mean age between the two data years (Figure
\ref{fig:weighting_nwfscslope}). The mean age for the NWFSC shelf-slope
survey generally showed a declining trend over the time-series excluding
2013 and 2016 which sampled older fish relative to the other years
(Figure \ref{fig:weighting_nwfsccombo}).

\subsubsection{Population Trajectory}\label{population-trajectory}

The predicted spawning output (in millions of eggs) is given in Table
\ref{tab:Timeseries_mod1} and plotted in Figure \ref{fig:ssb}. The
predicted spawning output from the base model generally showed a slight
decline over the time-series until when the foreign fleet began. A
short, but sharp decline occurred during the period of the foreign
fishery in the late 1960s. The stock continued to decline minimally
until 1989 (37\%), at which point the stock size remained relatively
flat, until 2000, when a combination of strong recruitment and low
catches resulted in an increase in spawning output at the end of the
time-series. The recent increase is even faster for total biomass
(Figure \ref{fig:total_bio}) because not all fish from the 2008
recruitment are mature (Figure \ref{fig:mat_compare}) with the model
estimating the final year total biomass higher than unfished conditions.
The 2017 spawning output relative to unfished equilibrium spawning
output is above the target of 40\% of unfished spawning output (76.6\%)
(Figure \ref{fig:depl}). Approximate confidence intervals based on the
asymptotic variance estimates show that the uncertainty in the estimated
spawning output is high, especially in the early years. The standard
deviation of the log of the spawning output in 2017 is 0.27.

Recruitment deviations were estimated for the entire time-series that
was modeled (Figure \ref{fig:recruits} and discussed in Section
\ref{parameter-estimates}) and provide a realistic portrayal of
uncertainty. Recruitment predictions from the mid-1970s and early 1980s
were mostly below average, with the 1999, 2000, 2008, and 2013 cohorts
being the strongest over the modeled period. Many other stock
assessments of rockfish along the west coast of the US have estimated a
large recruitment event in 1999 (e.g., greenstriped rockfish (Hicks et
al. \protect\hyperlink{ref-hicks_status_2009}{2009}), chilipepper
rockfish (Field \protect\hyperlink{ref-field_status_2007}{2007}),
darkblotched rockfish (Gertseva et al.
\protect\hyperlink{ref-gertseva_status_2015}{2015})). The 2008 year
class was estimated as the strongest year class measured to date for
Pacific ocean perch. This year has been estimated to have very strong
year classes for other West Coast stocks (e.g., darkblotched rockfish
(Gertseva et al. \protect\hyperlink{ref-gertseva_status_2015}{2015}),
widow rockfish (Hicks and Wetzel
\protect\hyperlink{ref-hicks_status_2015}{2015})). It may be worthwhile
to investigate the periods of strong and weak year classes further to
see if it is an artifact of the data, a consistent autocorrelation, or a
result of the environment.

The stock-recruit curve resulting from a fixed value of steepness is
shown in Figure \ref{fig:stock_recruit_curve} with estimated
recruitments also shown. The stock is predicted to have never fallen to
low enough levels that the effects of steepness are obvious. However,
the lowest levels of predicted spawning output showed some of the
smallest recruitments and very few above average recruitments. Steepness
was not estimated in this model, but a sensitivity to an alternative
value of steepness is discussed below.

\subsubsection{Uncertainty and Sensitivity
Analyses}\label{uncertainty-and-sensitivity-analyses}

A number of sensitivity analyses were conducted. Each of the
sensitivities conducted were single explorations from the base model
assumptions and data and were not performed in a cumulative fashion.

\begin{enumerate}

  \item Data weighting according to the harmonic mean (McAllister, and Ianelli, 1997).  The data weights used in this sensitivity are showing in Table \ref{tab:harm} and can be compared to the weights used in the base model shown in Table \ref{tab:francis}.
  
  \item Fixed steepness at the value assumed in the 2011 assessment of 0.40.
  
  \item Fixed steepness at the mean of the 2017 steepness prior of 0.72.
  
  \item Maturity relationship used in the previous assessment.
  
  \item Fecundity relationship used in the previous assessment.
  
  \item Remove the influence of the large 2008 year-class by setting the 2008 recruitment deviation to zero (estimated straight from the stock recruitment curve)
  
  \item Include Triennial shelf survey (1980-2004) and composition data.
  
  \item Remove all other surveys and associated length and age data, except for the Triennial shelf survey.  Fishery length and ages were retained.  
  
  \item Include the historical commercial CPUE index.
  
  \item Inclusion of available Canadian fishery and survey data (does not constitute all data used in Canadian assessments).  This sensitivity includes Canadian fishery landings (1997-2016 with landings ranging from 260-400 mt by year) and survey removals (2004, 2006, 2008, 2010, 2012, 2014, 2016), no fishery or survey index of abundance, but length- and age-composition data from both the fishery and survey.  
  
  \item Inclusion of historical Washington research lengths.  
  
  \item Inclusion of Oregon special projects length and age data, which are sampled at the dockside or processing facilities.  
  
\end{enumerate}

Likelihood values and estimates of key parameters from each sensitivity
are available in Tables \ref{tab:Sensitivity1} and
\ref{tab:Sensitivity2}. Plots of the estimated time-series of spawning
output and relative depletion are shown in Figures \ref{fig:sens1_ssb},
\ref{fig:sens1_depl}, \ref{fig:sens2_ssb}, and \ref{fig:sens2_depl}.

The sensitivities that explored steepness and using only Triennial
survey data exhibited the largest changes in estimated stock status
relative to the base model. The sensitivities that explored alternative
steepness values differed the greatest from the base model. Assuming a
the lower steepness value of 0.40 resulted in the final stock status
being above the management target but having historically declining
below the target. Using only the Triennial shelf survey data resulted in
a reduction in stock size and status relative to the base model. The
model estimated extended positive recruitments in the early period of
the model in order to create an population age-structure that was
consistent with the composition data resulting in an increase in biomass
prior to the start of the foreign fishing fleet, indicating that
potential model misspecification in the absence of the other survey
data.

Weighting the data according to the harmonic means resulted in a
decrease in the estimated stock status relative to the base model with
the stock being estimated at 60\% of unfished spawning output.

The sensitivity that removed the large 2008 year-class resulted in a
large change in estimated stock status relative to the base model.
Assuming a recruitment even straight from the stock recruitment curve
resulted in an estimated stock status of 59\%.

Including additional data from either Canada, Washington research
lengths, and or Oregon special projects data resulted in minimal
reduction in the stock status relative to the base model.

The sensitivities that explored the inclusion of the CPUE index, the
2011 maturity, or fecundity relationship had little impact relative to
the base model estimated stock status.

\subsubsection{Retrospective Analysis}\label{retrospective-analysis}

A 5-year retrospective analysis was conducted by running the model using
data only through 2011, 2012, 2013, 2014, and 2015, progressively
(Figure \ref{fig:retro_sb} and \ref{fig:retro_recdev}). The initial
scale of the spawning population was basically unchanged for all of
these retrospectives. The estimation of the 2008 recruitment deviation
decreased as more data were removed. Overall, no alarming patterns were
present in the retrospective analysis.

\subsubsection{Likelihood Profiles}\label{likelihood-profiles}

Likelihood profiles were conducted for \(R_0\), steepness, and natural
mortality values separately. These likelihood profiles were conducted by
fixing the parameter at specific values and estimated the remaining
parameters based on the fixed parameter value.

For steepness, the negative log-likelihood was essentially flat between
values of 0.40 - 0.80 (Figure \ref{fig:piner_h}). Likelihood components
by data source show that the fishery length and age data support a low
steepness value, but the NWFSC shelf-slope age data support a higher
value for steepness. The surveys generally do not provide information
concerning steepness. The relative depletion for Pacific ocean perch has
a wide range across different assumed values of steepness (Figure
\ref{fig:h_trajectory}).

The negative log-likelihood was minimized at a natural mortality value
of 0.06, but the 95\% confidence interval extends over values ranging
from 0.035 - 0.08. The age and length data likelihood contribution was
minimized at natural morality values ranging from 0.055-0.06 (Figure
\ref{fig:m_like}). The relative depletion for Pacific ocean perch widely
varied across alternative values of natural mortality (Figure
\ref{fig:m_trajectory}).

In regards to values of \(R_0\), the negative log-likelihood was
minimized at approximately log(\(R_0\)) of 9.4 (Figure
\ref{fig:piner_R0}). The fishery and survey composition data were in
opposition regarding values of \(R_0\) where the fishery length and age
data indicated lower values of \(R_0\) while the survey ages from the
Pacific ocean perch and the NWFSC shelf-slope surveys indicated a higher
value. The survey indices were uninformative concerning \(R_0\), an
issue that was explored and discussed in depth during the week of the
STAR panel.

\subsubsection{Reference Points}\label{reference-points-1}

Reference points were calculated using the estimated selectivities and
catch distributions among fleets in the most recent year of the model
(2016). Sustainable total yields (landings plus discards) were 1,822.5
mt when using an \(SPR_{50\%}\) reference harvest rate and with a 95\%
confidence interval of 1,288.5 - 2,356.5 mt based on estimates of
uncertainty. The spawning output equivalent to 40\% of the unfished
spawning output (\(SB_{40\%}\)) was 2,755.7 millions of eggs. The recent
catches (landings plus discards) have been below the point estimate of
potential long-term yields calculated using an \(SPR_{50\%}\) reference
point and the population has been increasing sharply over the last 15
years.

The predicted spawning output from the base model generally showed a
sharp decline during the 1960s followed by less of a decline until 1989
(Figure \ref{fig:ssb}). Since 2001, the spawning output has been rapidly
increasing due to small catches, and recently, above average
recruitment. The 2017 spawning output relative to unfished equilibrium
spawning output is above the target of 40\% of unfished spawning output
(Figure \ref{fig:depl}). The fishing intensity,
\((1-SPR)/(1-SPR_{50\%})\), exceeded the current estimates of the
harvest rate limit (\(SPR_{50\%}\)) throughout the 1960s as seen in
Figure \ref{fig:SPR}. Recent exploitation rates on Pacific ocean perch
were predicted to be much less than target levels. In recent years, the
stock has experienced exploitation rates that have been below the target
level while the spawning output level has remained above the target
level.

Table \ref{tab:Ref_pts} shows the full suite of estimated reference
points for the base model and Figure \ref{fig:yield} shows the
equilibrium curve based on a steepness value fixed at 0.50.

\section{Harvest Projections and Decision
Tables}\label{harvest-projections-and-decision-tables}

A ten year projection of the base model with catches equal to the
estimated ACL for years 2019-2028 and a catch allocation equal to the
percentages for each fleet over the period of 2014-2016 predicts an
increase in the spawning output due to large 2008 cohort, with a slight
downturn beginning in 2023 (Table \ref{tab:Forecast_mod1}). Additional
projections with the current ACL or the SPR50 MSY using the low and high
states of nature are provided in Table
\ref{tab:Decision_table_mod1_back} and show the spawning output
remaining well above the management target for either catch level.

\section{Regional Management
Considerations}\label{regional-management-considerations}

The distribution of Pacific ocean perch occur primarily in the US west
coast waters of Washington, Oregon, and northern California and is
currently managed to a species level with harvest limits set for the
stock north of the \(40^\circ 10^\prime\) latitude. The population
within this area is treated as a single stock due to the lack of
biological and genetic data indicating the presence of multiple stocks.
Analysis conducted within this assessment did not find support for
regional management within the area that Pacific ocean perch occur.

\section{Research Needs}\label{research-needs}

There are many areas of research that could be improved to benefit the
understanding and assessment of Pacific ocean perch. Below, are issues
that are considered of importance.

\begin{enumerate}

\item \textbf{Natural mortality}: Uncertainty in natural mortality translates into uncertain estimates of status and sustainable fishing levels for Pacific ocean perch. The collection of additional age data, re-reading of older age samples, reading old age samples that are unread, and improved understanding of the life history of Pacific ocean perch may reduce that uncertainty.

\item \textbf{Steepness}: The amount of stock resilience, steepness, dictates the rate at which a stock can rebuild from low stock sizes.  Improved understanting regarding the steepness of US west coast Pacific ocean perch will reduce our uncertainty regarding current stock status.

\item \textbf{Basin-wide understanding of stock structure, biology, connectivity, and distribution:} This is a stock assessment for Pacific ocean perch off of the west coast of the US and does not consider data from British Columbia or Alaska. Further investigating and comparing the data and predictions from British Columbia and Alaska to determine if there are similarities with the US west coast observations would help to define the connectivity between Pacific ocean perch north and south of the U.S.-Canada border.


\end{enumerate}

\section{Acknowledgments}\label{acknowledgments}

Many people were instrumental in the successful completion of this
assessment and their contribution is greatly appreciated. Jason Cope
(NWFSC), Ian Taylor (NWFSC), and Owen Hamel (NWFSC), contributed greatly
with discussions about data, modeling, and SS. Allan Hicks (IPHC)
provided guidance on model approaches and data exploration. We are
grateful to Theresa Tsou (WDFW) and Phillip Wyland (WDFW) who provided
research data and the first historical reconstruction of catch for
Washington state. Alison Whitman (ODFW), Patrick Mirrick (ODFW), and Ted
Calavan (ODFW) provided Oregon composition data, historical catches,
corrected PacFIN catches, and quickly uploaded age data that were
critical to this assessment. We appreciate Vanessa Tuttle's patience and
responsiveness to providing data. John Field (SWFSC) provided historical
catch information and Don Pearson (SWFSC) compiled the extensive
management changes for Pacific ocean perch which were critical in
understanding and modeling fishery behavior. John Wallace provided
multiple last minute PacFIN extractions and analyzed historical discard
rates for use in the assessment.

We are very grateful to Patrick McDonald and the team of agers at CAP
for their hard work reading numerous otoliths and availability to answer
questions when needed. Beth Horness was always eager to help, quick to
supply survey extractions, and answered numerous questions we had. Jason
Jannot and Kayleigh Sommers assisted with data from the WCGOP and
entertained our many questions. We appreciate Melissa Head's effort to
produce updated maturity estimates for the assessment. We would like to
acknowledge our survey team and their dedication to improving the
assessments we do. The assessment was greatly improved through the many
discussions within the Population Ecology team in the FRAM division at
the NWFSC.

The reviews of the STAR panel (Norman Hall, Kevin Piner, and Yiota
Apostolaki) and STAR panel chair (David Sampson), the GMT representative
(Lynn Mattes), and the GAP representative (Louie Zimm) are greatly
appreciated for their patience, advice, focus, understanding, and
comments during and after the STAR panel meeting. John Devore was an
indispensable resource of the past, current, and future management as
well as a great organizer and representative of the Pacific Fishery
Management Council.

\newpage

\FloatBarrier

\section{Tables}\label{tables}

\begin{table}[ht]
\centering
\caption{Landings for each state (all gears combined), the at-sea hake fishery, the foreign fleet, and surveys for the modeled years.} 
\label{tab:Comm_Catch}
\begin{tabular}{>{\centering}p{.5in}>{\centering}p{.75in}>{\centering}p{.75in}>{\centering}p{.75in}>{\centering}p{1in}>{\centering}p{.75in}>{\centering}p{.75in}}
  \hline
Year & California & Oregon & Washington & At-Sea Hake & Foreign & Survey \\ 
  \hline
1918 & 0.1 & 0.0 & 1.1 & 0.0 &  0 & 0.0 \\ 
  1919 & 0.0 & 0.0 & 0.4 & 0.0 &  0 & 0.0 \\ 
  1920 & 0.0 & 0.0 & 0.3 & 0.0 &  0 & 0.0 \\ 
  1921 & 0.0 & 0.0 & 0.3 & 0.0 &  0 & 0.0 \\ 
  1922 & 0.0 & 0.0 & 0.1 & 0.0 &  0 & 0.0 \\ 
  1923 & 0.0 & 0.0 & 0.2 & 0.0 &  0 & 0.0 \\ 
  1924 & 0.1 & 0.0 & 0.5 & 0.0 &  0 & 0.0 \\ 
  1925 & 0.1 & 0.0 & 0.6 & 0.0 &  0 & 0.0 \\ 
  1926 & 0.1 & 0.0 & 1.0 & 0.0 &  0 & 0.0 \\ 
  1927 & 0.1 & 0.0 & 1.4 & 0.0 &  0 & 0.0 \\ 
  1928 & 0.1 & 0.1 & 1.2 & 0.0 &  0 & 0.0 \\ 
  1929 & 0.3 & 0.1 & 0.7 & 0.0 &  0 & 0.0 \\ 
  1930 & 0.2 & 0.1 & 0.9 & 0.0 &  0 & 0.0 \\ 
  1931 & 0.4 & 0.1 & 0.4 & 0.0 &  0 & 0.0 \\ 
  1932 & 0.3 & 0.1 & 0.4 & 0.0 &  0 & 0.0 \\ 
  1933 & 0.6 & 0.1 & 0.5 & 0.0 &  0 & 0.0 \\ 
  1934 & 0.4 & 0.0 & 2.3 & 0.0 &  0 & 0.0 \\ 
  1935 & 0.4 & 0.1 & 7.7 & 0.0 &  0 & 0.0 \\ 
  1936 & 0.2 & 0.2 & 1.6 & 0.0 &  0 & 0.0 \\ 
  1937 & 0.5 & 0.4 & 2.0 & 0.0 &  0 & 0.0 \\ 
  1938 & 0.6 & 0.1 & 5.1 & 0.0 &  0 & 0.0 \\ 
  1939 & 0.9 & 0.4 & 8.7 & 0.0 &  0 & 0.0 \\ 
  1940 & 0.9 & 9.1 & 12.2 & 0.0 &  0 & 0.0 \\ 
  1941 & 1.3 & 14.0 & 13.6 & 0.0 &  0 & 0.0 \\ 
  1942 & 0.4 & 26.6 & 18.6 & 0.0 &  0 & 0.0 \\ 
  1943 & 1.0 & 94.3 & 453.6 & 0.0 &  0 & 0.0 \\ 
  1944 & 2.8 & 164.5 & 739.3 & 0.0 &  0 & 0.0 \\ 
  1945 & 6.7 & 247.1 & 1887.1 & 0.0 &  0 & 0.0 \\ 
  1946 & 7.3 & 193.2 & 845.9 & 0.0 &  0 & 0.0 \\ 
  1947 & 2.6 & 167.2 & 385.3 & 0.0 &  0 & 0.0 \\ 
  1948 & 4.2 & 177.8 & 491.1 & 0.0 &  0 & 0.0 \\ 
  1949 & 2.2 & 472.9 & 409.5 & 0.0 &  0 & 0.0 \\ 
  1950 & 1.5 & 690.1 & 675.7 & 0.0 &  0 & 0.0 \\ 
  1951 & 4.3 & 840.1 & 735.1 & 0.0 &  0 & 0.0 \\ 
  1952 & 3.1 & 2030.5 & 305.6 & 0.0 &  0 & 0.0 \\ 
  1953 & 146.4 & 1223.5 & 361.6 & 0.0 &  0 & 0.0 \\ 
  1954 & 123.6 & 1837.5 & 538.8 & 0.0 &  0 & 0.0 \\ 
  1955 & 50.6 & 1346.4 & 555.6 & 0.0 &  0 & 0.0 \\ 
  1956 & 4.1 & 2563.8 & 548.2 & 0.0 &  0 & 0.0 \\ 
   \hline
\end{tabular}
\end{table}

\begin{table}[ht]
\centering
\begin{tabular}{>{\centering}p{.5in}>{\centering}p{.75in}>{\centering}p{.75in}>{\centering}p{.75in}>{\centering}p{1in}>{\centering}p{.75in}>{\centering}p{.75in}}
  \hline
Year & California & Oregon & Washington & At-Sea Hake & Foreign & Survey \\ 
  \hline
1957 & 1.7 & 2128.1 & 538.5 & 0.0 &  0 & 0.0 \\ 
  1958 & 3.1 & 1564.9 & 530.4 & 0.0 &  0 & 0.0 \\ 
  1959 & 1.6 & 892.6 & 337.0 & 0.0 &  0 & 0.0 \\ 
  1960 & 20.9 & 1358.8 & 928.1 & 0.0 &  0 & 0.0 \\ 
  1961 & 1.2 & 2061.9 & 1179.8 & 0.0 &  0 & 0.0 \\ 
  1962 & 0.6 & 2584.9 & 1725.2 & 0.0 &  0 & 0.0 \\ 
  1963 & 33.1 & 3693.9 & 2006.0 & 0.0 &  0 & 0.0 \\ 
  1964 & 47.1 & 4261.6 & 1770.7 & 0.0 &  0 & 0.0 \\ 
  1965 & 36.3 & 5627.8 & 1972.1 & 0.0 &  0 & 0.0 \\ 
  1966 & 5.3 & 1591.2 & 1725.5 & 0.0 & 15561 & 0.0 \\ 
  1967 & 18.1 & 354.7 & 1861.0 & 0.0 & 12357 & 0.0 \\ 
  1968 & 22.3 & 466.4 & 2501.2 & 0.0 & 6639 & 0.0 \\ 
  1969 & 8.4 & 422.3 & 1236.0 & 0.0 & 469 & 0.0 \\ 
  1970 & 8.7 & 507.4 & 1293.3 & 0.0 & 441 & 0.0 \\ 
  1971 & 12.2 & 290.4 & 673.6 & 0.0 & 902 & 0.0 \\ 
  1972 & 11.4 & 105.3 & 796.5 & 0.0 & 950 & 0.0 \\ 
  1973 & 11.9 & 121.2 & 713.1 & 0.0 & 1773 & 0.0 \\ 
  1974 & 15.7 & 136.7 & 641.8 & 0.0 & 1457 & 0.0 \\ 
  1975 & 11.4 & 181.3 & 413.9 & 62.3 & 496 & 0.0 \\ 
  1976 & 17.1 & 663.7 & 521.1 & 31.9 & 239 & 0.0 \\ 
  1977 & 16.7 & 457.1 & 752.0 & 3.8 &  0 & 11.9 \\ 
  1978 & 42.5 & 498.7 & 1391.5 & 15.4 &  0 & 0.0 \\ 
  1979 & 136.7 & 735.9 & 581.4 & 15.1 &  0 & 34.5 \\ 
  1980 & 19.2 & 948.6 & 666.2 & 47.0 &  0 & 4.6 \\ 
  1981 & 10.8 & 929.7 & 390.3 & 15.4 &  0 & 0.0 \\ 
  1982 & 145.9 & 584.0 & 273.0 & 28.3 &  0 & 0.0 \\ 
  1983 & 102.0 & 1032.7 & 437.7 & 10.9 &  0 & 4.4 \\ 
  1984 & 47.6 & 750.4 & 815.7 & 2.3 &  0 & 0.9 \\ 
  1985 & 70.9 & 789.5 & 503.2 & 11.4 &  0 & 13.6 \\ 
  1986 & 52.8 & 676.5 & 588.9 & 19.8 &  0 & 1.4 \\ 
  1987 & 120.9 & 550.0 & 399.4 & 5.4 &  0 & 0.0 \\ 
  1988 & 75.4 & 749.8 & 509.8 & 4.5 &  0 & 0.5 \\ 
  1989 & 29.5 & 927.8 & 466.2 & 4.3 &  0 & 4.2 \\ 
  1990 & 18.3 & 567.8 & 427.2 & 80.9 &  0 & 0.0 \\ 
  1991 & 8.4 & 853.2 & 530.1 & 46.1 &  0 & 0.0 \\ 
  1992 & 15.3 & 623.4 & 435.2 & 373.3 &  0 & 4.9 \\ 
  1993 & 11.0 & 797.8 & 464.7 & 0.9 &  0 & 0.2 \\ 
  1994 & 6.7 & 626.4 & 352.0 & 83.8 &  0 & 0.0 \\ 
  1995 & 9.2 & 515.0 & 289.8 & 46.6 &  0 & 2.8 \\ 
  1996 & 18.4 & 531.1 & 236.7 & 6.3 &  0 & 1.2 \\ 
  1997 & 15.8 & 439.1 & 184.9 & 6.4 &  0 & 0.1 \\ 
  1998 & 21.6 & 436.7 & 172.4 & 22.3 &  0 & 3.8 \\ 
  1999 & 19.8 & 326.8 & 145.8 & 16.5 &  0 & 1.4 \\ 
   \hline
\end{tabular}
\end{table}

\begin{table}[ht]
\centering
\begin{tabular}{>{\centering}p{.5in}>{\centering}p{.75in}>{\centering}p{.75in}>{\centering}p{.75in}>{\centering}p{1in}>{\centering}p{.75in}>{\centering}p{.75in}}
  \hline
Year & California & Oregon & Washington & At-Sea Hake & Foreign & Survey \\ 
  \hline
2000 & 6.8 & 95.1 & 33.0 & 10.1 &  0 & 0.6 \\ 
  2001 & 0.5 & 193.4 & 51.8 & 21.0 &  0 & 2.8 \\ 
  2002 & 0.8 & 107.0 & 39.5 & 3.9 &  0 & 0.3 \\ 
  2003 & 0.2 & 94.6 & 30.2 & 6.3 &  0 & 3.6 \\ 
  2004 & 2.1 & 97.7 & 22.3 & 1.1 &  0 & 2.5 \\ 
  2005 & 0.1 & 51.2 & 10.4 & 1.7 &  0 & 1.8 \\ 
  2006 & 0.2 & 52.2 & 15.8 & 3.1 &  0 & 1.2 \\ 
  2007 & 0.2 & 83.7 & 45.1 & 4.0 &  0 & 0.6 \\ 
  2008 & 0.4 & 58.6 & 16.6 & 15.9 &  0 & 0.8 \\ 
  2009 & 0.9 & 58.7 & 33.2 & 1.6 &  0 & 2.7 \\ 
  2010 & 0.1 & 58.0 & 22.3 & 16.9 &  0 & 1.7 \\ 
  2011 & 0.1 & 30.3 & 19.7 & 9.2 &  0 & 1.9 \\ 
  2012 & 0.2 & 30.4 & 21.8 & 4.5 &  0 & 1.6 \\ 
  2013 & 0.1 & 34.9 & 14.8 & 5.4 &  0 & 1.7 \\ 
  2014 & 0.2 & 33.9 & 15.8 & 3.9 &  0 & 0.6 \\ 
  2015 & 0.1 & 38.1 & 11.4 & 8.7 &  0 & 1.6 \\ 
  2016 & 0.2 & 40.8 & 13.1 & 10.3 &  0 & 3.1 \\ 
   \hline
\end{tabular}
\end{table}

\begin{table}[ht]
\centering
\caption{West Coast history of regulations.} 
\label{tab:Regs}
\begingroup\fontsize{9pt}{10pt}\selectfont
\begin{tabular}{>{\centering}p{.60in}>{\centering}p{1.0in}>{\raggedright}p{4.20in}}
  \hline
Date & Area & Regulation \\ 
  \hline
11/10/1983 &  Columbia  &  Closed Columbia area to Pacific ocean perch fishing until the end of the year, as 950 mt OY for this species has been reached;  \\ 
  11/10/1983 &  Vancouver  &  retained 5,000-pound trip limit or 10\% of total trip weight on landings of Pacific ocean perch in the Vancouver area.  \\ 
  1/1/1984 &  ALL  &  Continued 5,000-pound trip limit or 10\% of total trip weight on Pacific ocean perch as specified in FMP. Fishery to close when area OYs are reached (see action effective November 10, 1983 above).  \\ 
  8/1/1984 &  Vancouver Columbia  &  Reduced trip limit for Pacific ocean perch in the Vancouver and Columbia areas to 20\% by weight of all fish on board, not to exceed 5,000 pounds per vessel per trip. \\ 
  8/16/1984 &  Columbia  &  Commercial fishing for Pacific ocean perch in the Columbia area closed for remainder of the year. \\ 
  1/10/1985 &  Vancouver Columbia  &  Established Vancouver and Columbia areas Pacific ocean perch trip limit of 20\% by weight of all fish on board (no 5,000-pound limit as specified in last half of 1984). \\ 
  4/28/1985 &  Vancouver Columbia  &  Reduced the Vancouver and Columbia areas Pacific ocean perch trip limit to 5,000 pounds or 20\% by weight of all fish on board, whichever is less.  \\ 
  4/28/1985 &  ALL  &  Landings of Pacific ocean perch less than 1,000 pounds will be unrestricted. The fishery for this species will close when the OY in each area is reached. \\ 
  6/10/1985 &  ALL  &   Landings of Pacific ocean perch up to 1,000 pounds per trip will be unrestricted regardless of the percentage of these fish on board.  \\ 
  1/1/1986 &  Cape Blanco North  &  Established the Pacific ocean perch trip limit north of Cape Blanco (4250) at 20\% (by weight) of all fish on board or 10,000 pounds whichever is less;  \\ 
  1/1/1986 &  ALL  &  landings of Pacific ocean perch unrestricted if less than 1,000 pounds regardless of percentage on board; Vancouver area OY = 600 mt; Columbia area OY =950 mt.  \\ 
  12/1/1986 &  Vancouver  &  OY quota for Pacific ocean perch reached in the Vancouver area; fishery closed until January 1, 1987.  \\ 
  1/1/1987 &  ALL  &  Established coastwide Pacific ocean perch limit at 20\% of all legal fish on board or 5,000 pounds whichever is less (in round weight); landings of Pacific ocean perch unrestricted if less than 1,000 pounds regardless of percentage on board; Vancouver area OY =500 mt; Columbia area OY = 800 mt.  \\ 
  1/1/1988 &  ALL  &  Established the coastwide Pacific ocean perch trip limit at 20\% (by weight) of all fish on board or 5,000 pounds, whichever is less; landings of Pacific ocean perch unrestricted if less than 1,000 pounds regardless of percentage on board;  \\ 
  1/1/1989 &  ALL  &  Established the coastwide Pacific ocean perch trip limit at 20\% (by weight) of all fish on board or 5,000 pounds whichever is less;  \\ 
  1/1/1989 &  ALL  &  landings of Pacific ocean perch unrestricted if less than 1,000 pounds regardless of percentage on board (Vancouver area OY =500 mt; Columbia area OY =800 mt).  \\ 
  7/26/1989 &  ALL  &  Reduced the coastwide trip limit for Pacific ocean perch to 2,000 pounds or 20\% of all fish on board, whichever is less, with no trip frequency restriction.  \\ 
  12/13/1989 &  Columbia  &   Closed the Pacific ocean perch fishery in the Columbia area because 1,040 mt OY reached.  \\ 
  1/1/1990 &  ALL  &  Established the coastwide Pacific ocean perch trip limit at 20\% (by weight) of all fish on board or 3,000 pounds whichever is less; landings of Pacific ocean perch be unrestricted if less than 1,000 pounds regardless of percentage on board.  (Vancouver area OY = 500 mt; Columbia area OY = 1,040 mt). \\ 
  1/1/1991 &  ALL  &  Established the coastwide Pacific ocean perch trip limit at 20\% (by weight) of all groundfish on board or 3,000 pounds whichever is less; landings of Pacific ocean perch be unrestricted if less than 1,000 pounds regardless of percentage on board (harvest guideline for combined Vancouver and Columbia areas = 1,000 mt). \\ 
  1/1/1992 &  ALL  &  For Pacific ocean perch, established the coastwide trip limit at 20\% (by weight) of all groundfish on board or 3,000 pounds whichever is less; landings of Pacific ocean perch be unrestricted if less than 1,000 pounds regardless of percentage on board (harvest guideline for combined Vancouver and Columbia areas = 1,550 mt). \\ 
   \hline
\end{tabular}
\endgroup
\end{table}\begin{table}[ht]
\centering
\begingroup\fontsize{9pt}{10pt}\selectfont
\begin{tabular}{>{\centering}p{.75in}>{\centering}p{.75in}>{\raggedright}p{4.25in}}
  \hline
Date & Area & Regulation \\ 
  \hline
1/1/1993 &  Cape Mendocino Coos Bay  &  For Pacific ocean perch, continued the coastwide trip limit at 20\% (by weight) of all groundfish on board or 3,000 pounds whichever is less; landings of Pacific ocean perch unrestricted if less than 1,000 pounds regardless of percentage on board (harvest guideline for combined Vancouver and Columbia areas = 1,550 mt). \\ 
  1/1/1994 &  ALL  &  Pacific Ocean Perch trip limit of 3,000 pounds or 20\% of all fish on board, whichever is less, in landings of Pacific ocean perch above 1,000 pounds. \\ 
  1/1/1995 &  ALL  &  For Pacific Ocean Perch, established a cumulative trip limit of 6,000 pounds per month \\ 
  1/1/1996 &  ALL  &  Pacific Ocean Perch cumulative trip limit of 10,000 pounds per two-month period. \\ 
  7/1/1996 &  4030 North  &  Reduced the cumulative 2-month limit for Pacific ocean perch to 8,000 pounds, and established the cumulative 2-month limit for Dover sole north of Cape Mendocino at 38,000 pounds \\ 
  1/1/1997 &  ALL  &  Pacific Ocean Perch  limited entry fishery cumulative trip limit of 8,000 pounds per two-month period \\ 
  1/1/1998 &  ALL  &  Pacific Ocean Perch:  limited entry fishery Cumulative trip limit of 8,000 pounds per two-month period. \\ 
  7/1/1998 &  ALL  &  Open Access Rockfish: removed overall rockfish monthly limit and replaced it with limits for component rockfish species: for Sebastes complex, monthly cumulative limit is 33,000 pounds, for widow rockfish, monthly cumulative trip limit is 3,000 pounds, for Pacific Ocean Perch, monthly cumulative trip limit is 4,000 pounds. \\ 
  1/1/1999 &  ALL  &  for the limited entry fishery  A new three phase cumulative limit period system is introduced for 1999.  Phase 1 is a single cumulative limit period that is 3 months long, from January 1 - March 31.  Phase 2 has 3 separate 2 month cumulative limit periods of April 1 -  May 31, June 1 -  July 31, and August 1 - September 30.  Phase 3 has 3 separate 1 month cumulative limit periods of October 1-31, November 1-30, and December 1-31.  For all species except Pacific ocean perch and Bocaccio, there will be no monthly limit within the cumulative landings limit periods.  An option to apply cumulative trip limits lagged by 2 weeks (from the 16th to the 15th) was made available to limited entry trawl vessels when their permits were renewed for 1999.  Vessels that are authorized to operate in this "B" platoon may take and retain, but may not land, groundfish during January 1-15, 1999. \\ 
  1/1/1999 &  ALL  &  for the limited entry fishery Pacific Ocean Perch: cumulative limit, Phase 1: 4,000 pounds per month; Phase 2: 4,000 pounds per month; Phase 3: 4,000 pounds per month. \\ 
  1/1/1999 &  ALL  &  for open access gear: Pacific Ocean Perch: coastwide, 100 pounds per month. \\ 
  1/1/2000 &  ALL  &  Limited entry trawl, Pacific Ocean Perch, 500 lbs per month \\ 
  1/1/2000 &  ALL  &  Pacific Ocean Perch, Open Access gear except exempted trawl, 100 lbs per month \\ 
  1/1/2000 &  ALL  &  Pacific Ocean Perch, limited entry fixed gear, 500 lbs per month \\ 
  5/1/2000 &  ALL  &  Limited entry trawl, Pacific Ocean Perch, 2500 lbs per 2 months \\ 
  5/1/2000 &  ALL  &  Pacific Ocean Perch, limited entry fixed gear, 2500 lbs per month \\ 
  11/1/2000 &  ALL  &  Limited entry trawl, Pacific Ocean Perch, 500 lbs per month \\ 
  11/1/2000 &  ALL  &  Pacific Ocean Perch, limited entry fixed gear, 500 lbs per month \\ 
  1/1/2001 &  3600 North  &  Pacific Ocean Perch, open access, 100 lbs per month \\ 
  1/1/2001 &  4010 North  &  Pacific Ocean Perch, limited entry trawl, 1500 lbs per mont \\ 
  1/1/2001 &  ALL  &  Pacific Ocean Perch, limited entry fixed gear,  1500 lbs per month \\ 
  5/1/2001 &  4010 North  &  Pacific Ocean Perch, limited entry trawl,  2500 lbs per month \\ 
  5/1/2001 &  ALL  &  Pacific Ocean Perch, limited entry fixed gear,  2500 lbs per month \\ 
  10/1/2001 &  4010 North  &  Pacific Ocean Perch, limited entry trawl, 1500 lbs per month \\ 
  11/1/2001 &  ALL  &  Pacific Ocean Perch, limited entry fixed gear,  1500 lbs per month \\ 
  1/1/2002 &  4010 North  &  Pacific Ocean Perch, open access, 100 lbs per month \\ 
  1/1/2002 &  4010 North  &  Pacific Ocean Perch, limited entry fixed gear, 2000 lbs per month \\ 
  1/1/2002 &  4010 North  &  Pacific Ocean Perch, limited entry trawl, 2000 lbs per month \\ 
  4/1/2002 &  4010 North  &  Pacific Ocean Perch, limited entry fixed gear, 4000 lbs per month \\ 
  5/1/2002 &  4010 North  &  Pacific Ocean Perch, limited entry trawl, 4000 lbs per month \\ 
  11/1/2002 &  4010 North  &  Pacific Ocean Perch, limited entry fixed gear, 2000 lbs per month \\ 
  11/1/2002 &  4010 North  &  Pacific Ocean Perch, limited entry trawl, 2000 lbs per month \\ 
  1/1/2003 &  3800 South  &  minor slope rockfish south including pacific ocean perch, open access gear, 10000 lbs per 2 months \\ 
   \hline
\end{tabular}
\endgroup
\end{table}\begin{table}[ht]
\centering
\begingroup\fontsize{9pt}{10pt}\selectfont
\begin{tabular}{>{\centering}p{.75in}>{\centering}p{.75in}>{\raggedright}p{4.25in}}
  \hline
Date & Area & Regulation \\ 
  \hline
1/1/2003 &  3800 South  &  Minor slope rockfish south including Pacific ocean perch, limited entry fixed gear, 30000 lbs per 2 months \\ 
  1/1/2003 &  3800 South  &  Minor slope rockfish south including Pacific ocean perch , limited entry trawl, 30000 lbs per 2 months \\ 
  1/1/2003 &  3800 4010  &  minor slope rockfish south including pacific ocean perch, open access gear, per trip no more than 25\% (by weight) of sablefish landed \\ 
  1/1/2003 &  3800 4010  &  Minor slope rockfish south including Pacific ocean perch, limited entry fixed gear, 1800 lbs per 2 months \\ 
  1/1/2003 &  3800 4010  &  Minor slope rockfish south including Pacific ocean perch , limited entry trawl, 1800 lbs per 2 months \\ 
  1/1/2003 &  4010 North  &  pacific ocean perch, open access gears, 100 lbs per month \\ 
  1/1/2003 &  4010 North  &  pacific ocean perch, limited entry fixed gear, 1800 lbs per 2 months \\ 
  1/1/2003 &  4010 North  &  Pacific Ocean Perch, Limited entry trawl gear, 3000 lbs per 2 months \\ 
  3/1/2003 &  3800 4010  &  Minor slope rockfish south including Pacific ocean perch, limited entry fixed gear, no more than 25\% of the weight of sablefish landed per trip \\ 
  11/1/2003 &  3800 4010  &  Minor slope rockfish south including Pacific ocean perch, limited entry fixed gear, 1800 lbs per 2 months \\ 
  1/1/2004 &  3800 South  &  Minor slope rockfish south including Pacific ocean perch, open access gear, 10000 lbs per 2 months \\ 
  1/1/2004 &  3800 South  &  minor slope rockfish south inclding pacific ocean perch, limited entry fixed gear,  40000 lbs per 2 months \\ 
  1/1/2004 &  3800 South  &  minor slope rockfish south including pacific ocean perch, limited entry trawl, 40000 lbs per 2 months \\ 
  1/1/2004 &  3800 4010  &  Minor slope rockfish south including Pacific ocean perch, open access gear, per trip no more than 25\% of the weight of sablefish landed \\ 
  1/1/2004 &  3800 4010  &  minor slope rockfish south including pacific ocean perch, limited entry fixed gear, 7000 lbs per 2 months \\ 
  1/1/2004 &  3800 4010  &  minor slope rockfish south including pacific ocean perch, limited entry trawl, 7000 lbs per 2 months \\ 
  1/1/2004 &  4010 North  &  pacific ocean perch, open access gear, 100 lbs per month \\ 
  1/1/2004 &  4010 North  &  pacific ocean perch, limited entry fixed gear, 1800 lbs per 2 months \\ 
  1/1/2004 &  4010 North  &  pacific ocean perch, limited entry trawl, 3000 lbs per 2 months \\ 
  5/1/2004 &  3800 South  &  minor slope rockfish south inclding pacific ocean perch, limited entry fixed gear, 50000 lbs per 2 months \\ 
  5/1/2004 &  3800 South  &  minor slope rockfish south including pacific ocean perch, limited entry trawl, 50000 lbs per 2 months \\ 
  5/1/2004 &  3800 4010  &  minor slope rockfish south including pacific ocean perch, limited entry fixed gear,  50000 lbs per 2 months \\ 
  5/1/2004 &  3800 4010  &  minor slope rockfish south including pacific ocean perch, limited entry trawl, 50000 lbs per 2 months \\ 
  11/1/2004 &  3800 South  &  minor slope rockfish south inclding pacific ocean perch, limited entry fixed gear, 50000 lbs per 2 months \\ 
  11/1/2004 &  3800 South  &  minor slope rockfish south including pacific ocean perch, limited entry trawl, 50000 lbs per 2 months \\ 
  11/1/2004 &  3800 4010  &  minor slope rockfish south including pacific ocean perch, limited entry fixed gear, 10000 lbs per 2 months \\ 
  11/1/2004 &  3800 4010  &  minor slope rockfish south including pacific ocean perch, limited entry trawl, 10000 lbs per 2 months \\ 
  1/1/2005 &  3800 South  &  minor slope rockfish south including darkblotched and pacific ocean perch, open access gear, 10000 lbs per 2 months \\ 
  1/1/2005 &  3800 South  &  minor slope rockfish south including darkblotched rockfish and pacific ocean perch, limited entry trawl, closed \\ 
  1/1/2005 &  3800 4010  &  minor slope rockfish south including darkblotched and pacific ocean perch, open access gear, per trip no more than 25\% of weight of sablefish onboard \\ 
  1/1/2005 &  3800 4010  &  minor slope rockfish south including darkblotched rockfish and pacific ocean perch, limited entry trawl, 4000 lbs per 2 months \\ 
  1/1/2005 &  4010 North  &  pacific ocean perch, open access gears, 100 lbs per month \\ 
  1/1/2005 &  4010 North  &  pacific ocean perch, limited entry trawl gear, 3000 lbs per 2 months \\ 
  1/1/2005 &  4010 North  &  pacific ocean perch, limited entry fixed gear, 1800 lbs per 2 months \\ 
  1/1/2005 &  4010 South  &  minor slope rockfish south including darkblotched and pacific ocean perch, limited entry fixed gear, 40000 lbs per 2 months \\ 
  5/1/2005 &  3800 4010  &  minor slope rockfish south including darkblotched rockfish and pacific ocean perch, limited entry trawl, 8000 lbs per 2 months \\ 
   \hline
\end{tabular}
\endgroup
\end{table}\begin{table}[ht]
\centering
\begingroup\fontsize{9pt}{10pt}\selectfont
\begin{tabular}{>{\centering}p{.75in}>{\centering}p{.75in}>{\raggedright}p{4.25in}}
  \hline
Date & Area & Regulation \\ 
  \hline
1/1/2008 &  3800 4010  &  minor slope rockfish south including pacific ocean perch and darkblotched rockfish, limited entry trawl, 15000 lbs per 2 months \\ 
  1/1/2008 &  4010 North  &  pacific ocean perch, limited entry trawl, 1500 lbs per 2 months \\ 
  1/1/2009 &  4010 North  &  pacific ocean perch, limited entry fixed gear, 1800 lbs per 2 months \\ 
  1/1/2009 &  4010 South  &  minor slope rockfish south including pacific ocean perch and darkblotched, limited entry fixed gear, 40000 lbs per 2 months \\ 
  1/1/2009 &  3800 South  &  minor slope rockfish south including pacific ocean perch and darkblotched rockfish, open access gear, 10000 lbs per 2 months \\ 
  1/1/2009 &  3800 4010  &  minor slope rockfish south including pacific ocean perch and darkblotched rockfish, open access gear,  per trip no more than 25\% (by weight) of sablefish landed \\ 
  1/1/2009 &  4010 North  &  pacific ocean perch, open access gears, 100 lbs per month \\ 
  1/1/2009 &  3800 South  &  minor slope rockfish southincluding pacific ocean perch and darkblotched rockfish, limited entry trawl, 55000 lbs per 2 months \\ 
  1/1/2009 &  3800 4010  &  minor slope rockfish south including pacific ocean perch and darkblotched rockfish, limited entry trawl, 15000 lbs per 2 months \\ 
  1/1/2009 &  4010 North  &  pacific ocean perch, limited entry trawl, 1500 lbs per 2 months \\ 
  7/1/2009 &  3800 4010  &  minor slope rockfish south including pacific ocean perch and darkblotched rockfish, limited entry trawl, 10000 lbs per 2 months \\ 
  11/1/2009 &  3800 4010  &  minor slope rockfish south including pacific ocean perch and darkblotched rockfish, limited entry trawl, 15000 lbs per 2 months \\ 
  1/1/2010 &  4010 North  &  pacific ocean perch, limited entry fixed gear, 1800 lbs per 2 months \\ 
  1/1/2010 &  4010 South  &  minor slope rockfish south including pacific ocean perch and darkblotched,limited  entry fixed gear, 40000 lbs per 2 months \\ 
  1/1/2010 &  3800 South  &  minor slope rockfish south including pacific ocean perch and darkblotched rockfish, open access gear, 10000 lbs per 2 months \\ 
  1/1/2010 &  3800 4010  &  minor slope rockfish south including pacific ocean perch and darkblotched rockfish, open access gear,  per trip no more than 25\% (by weight) of sablefish landed \\ 
  1/1/2010 &  4010 North  &  pacific ocean perch, open access gears, 100 lbs per month \\ 
  1/1/2010 &  3800 South  &  minor slope rockfish south including pacific ocean perch and darkblotched rockfish, limited entry trawl, 55000 lbs per 2 months \\ 
  1/1/2010 &  3800 4010  &  minor slope rockfish south including pacific ocean perch and darkblotched rockfish, limited entry trawl, 15000 lbs per 2 months \\ 
  1/1/2010 &  4010 North  &  pacific ocean perch, limited entry trawl, 1500 lbs per 2 months \\ 
  1/1/2011 &  4010 North  &  pacific ocean perch, limited entry fixed gear, 1800 lbs per 2 months \\ 
  1/1/2011 &  4010 South  &  minor slope rockfish south including pacific ocean perch and darkblotched, limited entry fixed gear, 40000 lbs per 2 months \\ 
  1/1/2011 &  3800 South  &  minor slope rockfish south including pacific ocean perch and darkblotched rockfish, open access gear, 10000 lbs per 2 months \\ 
  1/1/2011 &  3800 4010  &  minor slope rockfish south including pacific ocean perch and darkblotched rockfish, open access gear,  per trip no more than 25\% (by weight) of sablefish landed \\ 
  1/1/2011 &  4010 North  &  pacific ocean perch, open access gears, 100 lbs per month \\ 
  1/1/2011 &  ALL  &  Pacific Ocean Perch managed in part by IFQ \\ 
  1/1/2012 &  4010 North  &  pacific ocean perch, limited entry fixed gear, 1800 lbs per 2 months \\ 
  1/1/2012 &  4010 South  &  minor slope rockfish southincluding pacific ocean perch and darkblotched, limited entry fixed gear, 40000 lbs per 2 months \\ 
  1/1/2012 &  3800 South  &  minor slope rockfish south including pacific ocean perch and darkblotched rockfish, open access gear, 10000 lbs per 2 months \\ 
  1/1/2012 &  3800 4010  &  minor slope rockfish south including pacific ocean perch and darkblotched rockfish, open access gear,  per trip no more than 25\% (by weight) of sablefish landed \\ 
  1/1/2012 &  4010 North  &  pacific ocean perch, open access gears, 100 lbs per month \\ 
  1/1/2013 &  4010 North  &  pacific ocean perch, open access gears, 100 lbs per month \\ 
  1/1/2013 &  4010 North  &  pacific ocean perch, limited entry fixed gear, 1800 lbs per 2 months \\ 
  1/1/2013 &  4010 South  &  minor slope rockfish south including pacific ocean perch and darkblotched, limited entry fixed gear, 40000 lbs per 2 months no more than 1375 lbs may be blackgill \\ 
  1/1/2013 &  4010 South  &  minor slope rockfish south including pacific ocean perch and darkblotched rockfish, open access gear,  10000 lbs per 2 months no more than 475 lbs of which may be blackgill rockfish \\ 
  1/1/2014 &  4010 North  &  non-trawl, limited entry, pacific ocean perch, 1800 lbs per 2 months \\ 
  1/1/2014 &  4010 South  &  non-trawl, limited entry, minor slope rockfish and darkblotched rockfish and pacific ocean perch, 40000 lbs per 2 months of which no more than 1375 lbs may be blackgill rockfish \\ 
   \hline
\end{tabular}
\endgroup
\end{table}\begin{table}[ht]
\centering
\begingroup\fontsize{9pt}{10pt}\selectfont
\begin{tabular}{>{\centering}p{.75in}>{\centering}p{.75in}>{\raggedright}p{4.25in}}
  \hline
Date & Area & Regulation \\ 
  \hline
1/1/2014 &  4010 North  &  non-trawl, open access, pacific ocean perch, 100 lbs per month \\ 
  1/1/2014 &  4010 South  &  non-trawl, open access, minor slope rockfish including darkblotched rockfishand pacific ocean perch, 10000 lbs per 2 months of which no more than 475 lbs may be blackgill rockfish \\ 
  1/1/2015 &  4010 North  &  non-trawl, limited entry, pacific ocean perch, 1800 lbs per 2 months \\ 
  1/1/2015 &  4010 South  &  non-trawl, limited entry, minor slope rockfish and darkblotched rockfish and pacific ocean perch, 40000 lbs per 2 months of which no more than 1375 lbs may be blackgill rockfish \\ 
  1/1/2015 &  4010 North  &  non-trawl, open access, pacific ocean perch, 100 lbs per month \\ 
  1/1/2015 &  4010 South  &  non-trawl, open access, minor slope rockfish including darkblotched rockfishand pacific ocean perch, 10000 lbs per 2 months of which no more than 475 lbs may be blackgill rockfish \\ 
  7/1/2015 &  4010 South  &  non-trawl, limited entry, minor slope rockfish and darkblotched rockfish and pacific ocean perch, 40000 lbs per 2 months of which no more than 1600 lbs may be blackgill rockfish \\ 
  7/1/2015 &  4010 South  &  non-trawl, open access, minor slope rockfish including darkblotched rockfishand pacific ocean perch, 10000 lbs per 2 months of which no more than 550 lbs may be blackgill rockfish \\ 
  1/1/2016 &  4010 North  &  non-trawl, limited entry, pacific ocean perch, 1800 lbs per 2 months \\ 
  1/1/2016 &  4010 North  &  non-trawl, open access, pacific ocean perch, 100 lbs per month \\ 
  1/1/2016 &  4010 South  &  non-trawl, open access, minor slope rockfish including darkblotched rockfishand pacific ocean perch, 10000 lbs per 2 months of which no more than 475 lbs may be blackgill rockfish \\ 
  7/1/2016 &  4010 South  &  non-trawl, open access, minor slope rockfish including darkblotched rockfishand pacific ocean perch, 10000 lbs per 2 months of which no more than 550 lbs may be blackgill rockfish \\ 
   \hline
\end{tabular}
\endgroup
\end{table}

\begin{table}[ht]
\centering
\caption{Recent trend in estimated total catch relative to management guidelines.} 
\label{tab:mnmgt_perform_tables}
\begin{tabular}{>{\raggedleft}p{0.5in}>{\centering}p{1.0in}>{\centering}p{1.0in}>{\centering}p{1.0in}>{\centering}p{1.1in}>{\centering}p{1.1in}}
  \hline
Year & OFL (mt; ABC prior to 2011) & ABC (mt) & ACL (mt; OY prior to 2011) & Total landings (mt) & Estimated total catch (mt) \\ 
  \hline
\text{2007} & 900 &  & 150 & 134 & 159 \\ 
  \text{2008} & 911 &  & 150 & 92 & 135 \\ 
  \text{2009} & 1,160 &  & 189 & 97 & 194 \\ 
  \text{2010} & 1,173 &  & 200 & 99 & 183 \\ 
  \text{2011} & 1,026 & 981 & 180 & 61 & 62 \\ 
  \text{2012} & 1,007 & 962 & 183 & 59 & 60 \\ 
  \text{2013} & 844 & 807 & 150 & 57 & 58 \\ 
  \text{2014} & 838 & 801 & 153 & 54 & 56 \\ 
  \text{2015} & 842 & 805 & 158 & 60 & 61 \\ 
  \text{2016} & 850 & 813 & 164 & 68 & 68 \\ 
   \hline
\end{tabular}
\end{table}

\begin{table}[ht]
\centering
\caption{Description of the data used to create the indices, the modeling platform used to generate the estimates, and the model configuration.} 
\label{tab:strata}
\begin{tabular}{>{\raggedleft}p{1.10in}>{\centering}p{1.10in}>{\centering}p{1.10in}>{\centering}p{1.10in}>{\centering}p{1.10in}}
  \hline
 & Pacific ocean perch & AFSC Slope & NWFSC Slope & NWFS Shelf-Slope \\ 
  \hline
Depth & 155-500 & 183-549 & 183-549 & 55-549 \\ 
  Latitude & 44-48.5 & 42-49 & 42-49 & 42-49 \\ 
  Model & VAST & VAST & Bayesian Delta GLMM & VAST \\ 
  Error Structure & Lognormal & Lognormal & Gamma & Lognormal \\ 
  Knots & 1000 & 1000 & - & 1000 \\ 
  Spatial & Y & Y & N & Y \\ 
  Temporal & Y & Y & N & Y \\ 
  Vessel-Year & N & N & Y & Y \\ 
   \hline
\end{tabular}
\end{table}

\begin{table}[ht]
\centering
\caption{Summary of the fishery-independent biomass/abundance
                                         time-series used in the stock
                                         assessment.  The standard error includes the input annual standard error and model estimated added variance.} 
\label{tab:Index_Summary}
\begin{tabular}{>{\centering}p{.4in}>{\centering}p{.5in}>{\centering}p{.3in}>{\centering}p{.5in}>{\centering}p{.3in}>{\centering}p{.5in}>{\centering}p{.3in}>{\centering}p{.5in}>{\centering}p{.3in}}
  \hline
   & \multicolumn{2}{c}{POP} &  \multicolumn{2}{c}{AFSC Slope} & \multicolumn{2}{c}{NWFSC Slope} & \multicolumn{2}{c}{NWFSC Shelf-Slope} \\
 Year & Obs & SE & Obs & SE & Obs & SE & Obs & SE \\
 \hline
1979 & 56461 & 0.27 & - & - & - & - & - & - \\ 
  1985 & 34645 & 0.29 & - & - & - & - & - & - \\ 
  1996 & - & - & 7621 & 0.51 & - & - & - & - \\ 
  1997 & - & - & 3807 & 0.51 & - & - & - & - \\ 
  1999 & - & - & 4694 & 0.50 & 3643 & 0.63 & - & - \\ 
  2000 & - & - & 4243 & 0.53 & 4120 & 0.58 & - & - \\ 
  2001 & - & - & 4187 & 0.49 & 2325 & 0.59 & - & - \\ 
  2002 & - & - & - & - & 1903 & 0.60 & - & - \\ 
  2003 & - & - & - & - & - & - & 9646 & 0.36 \\ 
  2004 & - & - & - & - & - & - & 5284 & 0.39 \\ 
  2005 & - & - & - & - & - & - & 7528 & 0.39 \\ 
  2006 & - & - & - & - & - & - & 6010 & 0.41 \\ 
  2007 & - & - & - & - & - & - & 6268 & 0.36 \\ 
  2008 & - & - & - & - & - & - & 3867 & 0.39 \\ 
  2009 & - & - & - & - & - & - & 2745 & 0.36 \\ 
  2010 & - & - & - & - & - & - & 5404 & 0.34 \\ 
  2011 & - & - & - & - & - & - & 7533 & 0.34 \\ 
  2012 & - & - & - & - & - & - & 9289 & 0.34 \\ 
  2013 & - & - & - & - & - & - & 8093 & 0.34 \\ 
  2014 & - & - & - & - & - & - & 4914 & 0.34 \\ 
  2015 & - & - & - & - & - & - & 5752 & 0.31 \\ 
  2016 & - & - & - & - & - & - & 11770 & 0.36 \\ 
   \hline
\end{tabular}
\end{table}

\FloatBarrier

\begin{table}[ht]
\centering
\caption{Summary of the design-based estimates of fishery-independent biomass/abundance
                                         time-series.} 
\label{tab:Design_Based}
\begin{tabular}{>{\centering}p{.4in}>{\centering}p{.6in}>{\centering}p{.3in}>{\centering}p{.6in}>{\centering}p{.3in}>{\centering}p{.6in}>{\centering}p{.3in}>{\centering}p{.6in}>{\centering}p{.3in}}
  \hline
   & \multicolumn{2}{c}{POP} &  \multicolumn{2}{c}{AFSC Slope} & \multicolumn{2}{c}{NWFSC Slope} & \multicolumn{2}{c}{NWFSC Shelf-Slope} \\
 Year & Obs & SE & Obs & SE & Obs & SE & Obs & SE \\
 \hline
1979 & 34135 & 0.25 & - & - & - & - & - & - \\ 
  1985 & 16675 & 0.18 & - & - & - & - & - & - \\ 
  1996 & - & - & 6472 & 0.29 & - & - & - & - \\ 
  1997 & - & - & 2965 & 0.43 & - & - & - & - \\ 
  1999 & - & - & 19063 & 0.48 & 6472 & 0.45 & - & - \\ 
  2000 & - & - & 4438 & 0.50 & 2965 & 0.48 & - & - \\ 
  2001 & - & - & 14570 & 0.69 & 19063 & 0.40 & - & - \\ 
  2002 & - & - & - & - & 4438 & 0.45 & - & - \\ 
  2003 & - & - & - & - & - & - & 21055 & 0.36 \\ 
  2004 & - & - & - & - & - & - & 4623 & 0.55 \\ 
  2005 & - & - & - & - & - & - & 9674 & 0.60 \\ 
  2006 & - & - & - & - & - & - & 9609 & 0.53 \\ 
  2007 & - & - & - & - & - & - & 3769 & 0.57 \\ 
  2008 & - & - & - & - & - & - & 5723 & 0.59 \\ 
  2009 & - & - & - & - & - & - & 14790 & 0.78 \\ 
  2010 & - & - & - & - & - & - & 11133 & 0.47 \\ 
  2011 & - & - & - & - & - & - & 6186 & 0.46 \\ 
  2012 & - & - & - & - & - & - & 10208 & 0.46 \\ 
  2013 & - & - & - & - & - & - & 14306 & 0.58 \\ 
  2014 & - & - & - & - & - & - & 4040 & 0.29 \\ 
  2015 & - & - & - & - & - & - & 9766 & 0.56 \\ 
  2016 & - & - & - & - & - & - & 19859 & 0.52 \\ 
   \hline
\end{tabular}
\end{table}

\FloatBarrier

\begin{table}[ht]
\centering
\caption{Summary of NWFSC shelf-slope survey length samples used in the stock assessment.} 
\label{tab:NWcombo_Lengths}
\begin{tabular}{>{\centering}p{.75in}>{\centering}p{.75in}>{\centering}p{.75in}>{\centering}p{1in}}
  \hline
Year & Tows & Fish & Sample Size \\ 
  \hline
2003 & 46 & 1426 & 111 \\ 
  2004 & 34 & 565 & 82 \\ 
  2005 & 38 & 526 & 92 \\ 
  2006 & 33 & 659 & 80 \\ 
  2007 & 50 & 628 & 121 \\ 
  2008 & 39 & 539 & 94 \\ 
  2009 & 46 & 471 & 111 \\ 
  2010 & 53 & 907 & 128 \\ 
  2011 & 53 & 921 & 128 \\ 
  2012 & 50 & 1175 & 121 \\ 
  2013 & 45 & 732 & 109 \\ 
  2014 & 52 & 991 & 126 \\ 
  2015 & 69 & 1165 & 167 \\ 
  2016 & 50 & 1150 & 121 \\ 
   \hline
\end{tabular}
\end{table}

\begin{table}[ht]
\centering
\caption{Summary of NWFSC shelf-slope survey age samples used in the stock assessment.} 
\label{tab:NWcombo_Ages}
\begin{tabular}{>{\centering}p{.75in}>{\centering}p{.75in}>{\centering}p{.75in}>{\centering}p{1in}}
  \hline
Year & Tows & Fish & Sample Size \\ 
  \hline
2003 & 45 & 432 & 109 \\ 
  2004 & 34 & 219 & 82 \\ 
  2005 & 38 & 257 & 92 \\ 
  2006 & 33 & 254 & 80 \\ 
  2007 & 50 & 439 & 121 \\ 
  2008 & 39 & 328 & 94 \\ 
  2009 & 45 & 331 & 109 \\ 
  2010 & 53 & 579 & 128 \\ 
  2011 & 53 & 674 & 128 \\ 
  2012 & 49 & 699 & 119 \\ 
  2013 & 44 & 553 & 106 \\ 
  2014 & 52 & 626 & 126 \\ 
  2015 & 68 & 840 & 165 \\ 
  2016 & 44 & 703 & 106 \\ 
   \hline
\end{tabular}
\end{table}

\begin{table}[ht]
\centering
\caption{Summary of NWFSC slope survey length samples used in the stock assessment.} 
\label{tab:NWslope_Lengths}
\begin{tabular}{>{\centering}p{.75in}>{\centering}p{.75in}>{\centering}p{.75in}>{\centering}p{1in}}
  \hline
Year & Tows & Fish & Sample Size \\ 
  \hline
2001 & 18 & 173 & 43 \\ 
  2002 & 24 & 368 & 58 \\ 
   \hline
\end{tabular}
\end{table}

\begin{table}[ht]
\centering
\caption{Summary of NWFSC slope survey age samples used in the stock assessment.} 
\label{tab:NWslope_Ages}
\begin{tabular}{>{\centering}p{.75in}>{\centering}p{.75in}>{\centering}p{.75in}>{\centering}p{1in}}
  \hline
Year & Tows & Fish & Sample Size \\ 
  \hline
2001 & 17 & 172 & 41 \\ 
  2002 & 24 & 359 & 58 \\ 
   \hline
\end{tabular}
\end{table}

\begin{table}[ht]
\centering
\caption{Summary of AFSC slope survey length samples used in the stock assessment.} 
\label{tab:AFSC_Lengths}
\begin{tabular}{>{\centering}p{.75in}>{\centering}p{.75in}>{\centering}p{.75in}>{\centering}p{1in}}
  \hline
Year & Tows & Fish & Sample Size \\ 
  \hline
1996 & 48 & 1396 & 116 \\ 
  1997 & 21 & 347 & 51 \\ 
  1999 & 21 & 562 & 51 \\ 
  2000 & 19 & 353 & 46 \\ 
  2001 & 23 & 390 & 55 \\ 
   \hline
\end{tabular}
\end{table}

\begin{table}[ht]
\centering
\caption{Summary of Pacific ocean perch survey length samples used in the stock assessment.} 
\label{tab:POP_Lengths}
\begin{tabular}{>{\centering}p{.75in}>{\centering}p{.75in}>{\centering}p{.75in}>{\centering}p{1in}}
  \hline
Year & Tows & Fish & Sample Size \\ 
  \hline
1979 & 125 & 2375 & 303 \\ 
  1985 & 126 & 2558 & 306 \\ 
   \hline
\end{tabular}
\end{table}

\begin{table}[ht]
\centering
\caption{Summary of Pacific ocean perch survey age samples used in the stock assessment.} 
\label{tab:POP_Ages}
\begin{tabular}{>{\centering}p{.75in}>{\centering}p{.75in}>{\centering}p{.75in}>{\centering}p{1in}}
  \hline
Year & Tows & Fish & Sample Size \\ 
  \hline
1985 & 29 & 1635 & 70 \\ 
   \hline
\end{tabular}
\end{table}

\begin{table}[ht]
\centering
\caption{Summary of discard rates used in the model by each data source.} 
\label{tab:Discard}
\begin{tabular}{>{\centering}p{.75in}>{\centering}p{1.1in}>{\centering}p{.75in}>{\centering}p{1.1in}}
  \hline
Year & Source & Discard & Standard Error \\ 
  \hline
1985 & Pikitch & 0.027 & 0.068 \\ 
  1986 & Pikitch & 0.024 & 0.063 \\ 
  1987 & Pikitch & 0.039 & 0.083 \\ 
  1992 & Management Restrictions & 0.100 & 0.300 \\ 
  2002 & WCGOP & 0.150 & 0.164 \\ 
  2003 & WCGOP & 0.183 & 0.268 \\ 
  2004 & WCGOP & 0.203 & 0.206 \\ 
  2005 & WCGOP & 0.175 & 0.346 \\ 
  2006 & WCGOP & 0.148 & 0.243 \\ 
  2007 & WCGOP & 0.171 & 0.261 \\ 
  2008 & WCGOP & 0.362 & 0.172 \\ 
  2009 & WCGOP & 0.504 & 0.153 \\ 
  2010 & WCGOP & 0.487 & 0.195 \\ 
  2011 & WCGOP & 0.015 & 0.053 \\ 
  2012 & WCGOP & 0.028 & 0.054 \\ 
  2013 & WCGOP & 0.027 & 0.054 \\ 
  2014 & WCGOP & 0.035 & 0.050 \\ 
  2015 & WCGOP & 0.010 & 0.053 \\ 
   \hline
\end{tabular}
\end{table}

\begin{table}[ht]
\centering
\caption{Summary of commercial fishery length samples used in the stock assessment (continued on next page).} 
\label{tab:Comm_Lengths}
\begin{tabular}{>{\centering}p{.75in}>{\centering}p{.75in}>{\centering}p{.75in}>{\centering}p{1in}}
  \hline
Year & Trips & Fish & Sample Size \\ 
  \hline
1966 & 1 & 238 & 7 \\ 
  1967 & 5 & 1020 & 35 \\ 
  1968 & 3 & 912 & 21 \\ 
  1969 & 4 & 1213 & 28 \\ 
  1970 & 13 & 1830 & 92 \\ 
  1971 & 22 & 4698 & 155 \\ 
  1972 & 23 & 4561 & 162 \\ 
  1973 & 17 & 4134 & 120 \\ 
  1974 & 20 & 4806 & 141 \\ 
  1975 & 19 & 3637 & 134 \\ 
  1976 & 21 & 3677 & 148 \\ 
  1977 & 32 & 4846 & 226 \\ 
  1978 & 52 & 7715 & 367 \\ 
  1979 & 34 & 3414 & 240 \\ 
  1980 & 55 & 5425 & 388 \\ 
  1981 & 40 & 3921 & 282 \\ 
  1982 & 48 & 4824 & 339 \\ 
  1983 & 39 & 3944 & 275 \\ 
  1984 & 31 & 3102 & 219 \\ 
  1985 & 45 & 4508 & 318 \\ 
  1986 & 40 & 4002 & 282 \\ 
  1987 & 43 & 3053 & 304 \\ 
  1988 & 9 & 601 & 64 \\ 
  1989 & 16 & 798 & 113 \\ 
  1990 & 12 & 599 & 85 \\ 
  1991 & 8 & 216 & 38 \\ 
  1994 & 43 & 2608 & 304 \\ 
  1995 & 49 & 3161 & 346 \\ 
  1996 & 64 & 3085 & 452 \\ 
  1997 & 76 & 3570 & 537 \\ 
  1998 & 56 & 3450 & 395 \\ 
  1999 & 58 & 2812 & 409 \\ 
  2000 & 49 & 2004 & 326 \\ 
  2001 & 59 & 1696 & 293 \\ 
  2002 & 50 & 1666 & 280 \\ 
   \hline
\end{tabular}
\end{table}

\begin{table}[ht]
\centering
\begin{tabular}{>{\centering}p{.75in}>{\centering}p{.75in}>{\centering}p{.75in}>{\centering}p{1in}}
  \hline
Year & Trips & Fish & Sample Size \\ 
  \hline
2003 & 67 & 1661 & 296 \\ 
  2004 & 53 & 1202 & 219 \\ 
  2005 & 51 & 1277 & 227 \\ 
  2006 & 59 & 1486 & 264 \\ 
  2007 & 81 & 2248 & 391 \\ 
  2008 & 101 & 3058 & 523 \\ 
  2009 & 107 & 3207 & 550 \\ 
  2010 & 134 & 2872 & 530 \\ 
  2011 & 100 & 1943 & 368 \\ 
  2012 & 97 & 1873 & 355 \\ 
  2013 & 117 & 2167 & 416 \\ 
  2014 & 140 & 2850 & 533 \\ 
  2015 & 110 & 2504 & 456 \\ 
  2016 & 131 & 2158 & 429 \\ 
   \hline
\end{tabular}
\end{table}

\FloatBarrier

\begin{table}[ht]
\centering
\caption{Summary of commercial fishery age samples used in the stock assessment.} 
\label{tab:Comm_Ages}
\begin{tabular}{>{\centering}p{.75in}>{\centering}p{.75in}>{\centering}p{.75in}>{\centering}p{1in}}
  \hline
Year & Trips & Fish & Sample Size \\ 
  \hline
1981 & 20 & 1901 & 141 \\ 
  1982 & 40 & 2776 & 282 \\ 
  1983 & 33 & 3317 & 233 \\ 
  1984 & 27 & 2625 & 191 \\ 
  1985 & 21 & 2096 & 148 \\ 
  1986 & 17 & 1693 & 120 \\ 
  1987 & 24 & 1193 & 169 \\ 
  1988 & 4 & 199 & 28 \\ 
  1994 & 8 & 238 & 41 \\ 
  1999 & 18 & 863 & 127 \\ 
  2000 & 14 & 677 & 99 \\ 
  2001 & 40 & 1349 & 226 \\ 
  2002 & 38 & 1414 & 233 \\ 
  2003 & 40 & 1309 & 221 \\ 
  2004 & 30 & 854 & 148 \\ 
  2005 & 37 & 1018 & 177 \\ 
  2006 & 49 & 1258 & 223 \\ 
  2007 & 63 & 1825 & 315 \\ 
  2008 & 44 & 1129 & 200 \\ 
  2009 & 75 & 1548 & 289 \\ 
  2010 & 54 & 1264 & 228 \\ 
  2011 & 85 & 1230 & 255 \\ 
  2012 & 7 & 331 & 49 \\ 
  2013 & 10 & 265 & 47 \\ 
  2014 & 91 & 587 & 172 \\ 
  2015 & 78 & 513 & 149 \\ 
  2016 & 21 & 254 & 56 \\ 
   \hline
\end{tabular}
\end{table}

\FloatBarrier

\begin{table}[ht]
\centering
\caption{Summary of at-sea hake fishery length samples used in the stock assessment.} 
\label{tab:ASHOP_Lengths}
\begin{tabular}{>{\centering}p{.75in}>{\centering}p{.75in}>{\centering}p{.75in}>{\centering}p{1in}}
  \hline
Year & Trips & Fish & Sample Size \\ 
  \hline
2003 & 153 & 805 & 263 \\ 
  2004 & 128 & 329 & 172 \\ 
  2005 & 221 & 734 & 321 \\ 
  2006 & 210 & 751 & 312 \\ 
  2007 & 319 & 1119 & 470 \\ 
  2008 & 26 & 2491 & 162 \\ 
  2009 & 12 & 366 & 63 \\ 
  2010 & 22 & 1794 & 155 \\ 
  2011 & 36 & 1748 & 226 \\ 
  2012 & 26 & 881 & 148 \\ 
  2013 & 26 & 834 & 140 \\ 
  2014 & 31 & 532 & 103 \\ 
  2015 & 23 & 925 & 150 \\ 
  2016 & 35 & 1947 & 240 \\ 
   \hline
\end{tabular}
\end{table}

\begin{table}[ht]
\centering
\caption{Summary of at-sea hake fishery age samples used in the stock assessment.} 
\label{tab:ASHOP_Ages}
\begin{tabular}{>{\centering}p{.75in}>{\centering}p{.75in}>{\centering}p{.75in}>{\centering}p{1in}}
  \hline
Year & Trips & Fish & Sample Size \\ 
  \hline
2003 & 142 & 378 & 194 \\ 
  2006 & 198 & 410 & 255 \\ 
  2007 & 297 & 620 & 383 \\ 
  2014 & 22 & 101 & 36 \\ 
   \hline
\end{tabular}
\end{table}

\FloatBarrier

\begin{table}[ht]
\centering
\caption{Estimated ageing error from the CAPS lab used in the assessment model} 
\label{tab:Age_Error}
\begin{tabular}{>{\centering}p{1.2in}>{\centering}p{1.2in}>{\centering}p{1.2in}>{\centering}p{1.2in}}
  \hline
True Age (yr) & SD of Observed Age (yr) & True Age (yr) & SD of Observed Age (yr) \\ 
  \hline
0.5 & 0.156 & 31.5 & 2.772 \\ 
  1.5 & 0.156 & 32.5 & 2.854 \\ 
  2.5 & 0.249 & 33.5 & 2.935 \\ 
  3.5 & 0.341 & 34.5 & 3.016 \\ 
  4.5 & 0.433 & 35.5 & 3.097 \\ 
  5.5 & 0.524 & 36.5 & 3.177 \\ 
  6.5 & 0.615 & 37.5 & 3.257 \\ 
  7.5 & 0.706 & 38.5 & 3.337 \\ 
  8.5 & 0.796 & 39.5 & 3.416 \\ 
  9.5 & 0.886 & 40.5 & 3.495 \\ 
  10.5 & 0.976 & 41.5 & 3.574 \\ 
  11.5 & 1.065 & 42.5 & 3.652 \\ 
  12.5 & 1.154 & 43.5 & 3.73 \\ 
  13.5 & 1.242 & 44.5 & 3.808 \\ 
  14.5 & 1.33 & 45.5 & 3.885 \\ 
  15.5 & 1.418 & 46.5 & 3.962 \\ 
  16.5 & 1.505 & 47.5 & 4.039 \\ 
  17.5 & 1.592 & 48.5 & 4.115 \\ 
  18.5 & 1.679 & 49.5 & 4.191 \\ 
  19.5 & 1.765 & 50.5 & 4.267 \\ 
  20.5 & 1.851 & 51.5 & 4.342 \\ 
  21.5 & 1.937 & 52.5 & 4.417 \\ 
  22.5 & 2.022 & 53.5 & 4.492 \\ 
  23.5 & 2.107 & 54.5 & 4.566 \\ 
  24.5 & 2.191 & 55.5 & 4.641 \\ 
  25.5 & 2.275 & 56.5 & 4.714 \\ 
  26.5 & 2.359 & 57.5 & 4.788 \\ 
  27.5 & 2.442 & 58.5 & 4.861 \\ 
  28.5 & 2.525 & 59.5 & 4.934 \\ 
  29.5 & 2.608 & 60.5 & 5.007 \\ 
  30.5 & 2.69 &   &   \\ 
   \hline
\end{tabular}
\end{table}

\FloatBarrier

\begin{table}[ht]
\centering
\caption{Specifications of the base model for Pacific ocean perch.} 
\label{tab:Model_setup}
\scalebox{0.9}{
\begin{tabular}{>{\raggedright}p{3in}>{\centering}p{2in}}
  \hline
Model Specification & Base Model \\ 
  \hline
Starting year & 1918 \\ 
   &  \\ 
  \underline{Population characteristics} &  \\ 
  Maximum age & 60 \\ 
  Gender & 2 \\ 
  Population lengths & 5-50 cm by 1 cm bins \\ 
  Summary biomass (mt) & Age 3+ \\ 
   &  \\ 
  \underline{Data characteristics} &  \\ 
  Data lengths & 11-47 cm by 1 cm bins \\ 
  Data ages & 1-40 ages \\ 
  Minimun age for growth calculations & 3 \\ 
  Maximum age for growth calculations & 20 \\ 
  First mature age & 0 \\ 
  Starting year of estimated recruitment & 1940 \\ 
   &  \\ 
  \underline{Fishery characteristics} &  \\ 
  Fishing mortality method & Discrete \\ 
  Maximum F & 0.9 \\ 
  Catchability & Analytical estimate \\ 
  Fishery selectivity & Double Normal \\ 
  At-Sea Hake selectivity & Double Normal \\ 
  POP survey selectivity & Logistic \\ 
  Triennial survey & Double Normal \\ 
  AFSC slope survey & Double Normal \\ 
  NWFSC slope survey & Double Normal \\ 
  NWFSC shelf-slope survey & Double Normal \\ 
   &  \\ 
  \underline{Fishery time blocks} &  \\ 
  Fishery selectivity & 1918-1999, 2000-2016 \\ 
  Fishery retention & 1918-1991, 1992-2001, 2002-2007, 2008, 2009-2010, 2011-2016 \\ 
   \hline
\end{tabular}
}
\end{table}

\FloatBarrier

\begin{table}[ht]
\centering
\caption{Data weights applied when using Francis data weighting in the base model.} 
\label{tab:francis}
\begin{tabular}{>{\raggedright}p{2in}>{\centering}p{.7in}>{\centering}p{.7in}}
  \hline
Fleet & Lengths & Ages \\ 
  \hline
Fishery & 0.096 & 0.217 \\ 
  At-sea hake & 0.104 & 0.032 \\ 
  Pacific ocean perch survey  & 1.000 & 1 \\ 
  AFSC slope survey & 0.077 & - \\ 
  NWFSC slope survey & 0.565 & 0.304 \\ 
  NWFSC shelf-slope survey & 0.031 & 0.363 \\ 
   \hline
\end{tabular}
\end{table}

\FloatBarrier 

\begin{landscape}
\begin{longtable}{lrcccll}
\caption{List of parameters used in
                                          the base model, including estimated 
                                          values and standard deviations (SD), 
                                          bounds (minimum and maximum), 
                                          estimation phase (negative values indicate
                                          not estimated), status (indicates if 
                                          parameters are near bounds, and prior type
                                          information (mean, SD).} \\ 
  \hline
Parameter & Value & Phase & Bounds & Status & SD & Prior (Exp.Val, SD)  \\ 
  \hline 
\endhead 
\hline 
\multicolumn{3}{l}{\footnotesize Continued on next page} 
\endfoot 
\endlastfoot 
 \hline
NatM\_p\_1\_Fem\_GP\_1 & 0.054 & -5 & (0.02, 0.1) &  &  & Log\_Norm (-2.92, 0.44) \\ 
  L\_at\_Amin\_Fem\_GP\_1 & 20.7538 & 3 & (15, 25) & OK & 0.14 & None \\ 
  L\_at\_Amax\_Fem\_GP\_1 & 41.6011 & 2 & (35, 45) & OK & 0.14 & None \\ 
  VonBert\_K\_Fem\_GP\_1 & 0.166779 & 3 & (0.1, 0.4) & OK & 0.00 & None \\ 
  SD\_young\_Fem\_GP\_1 & 1.34872 & 4 & (0.03, 5) & OK & 0.06 & None \\ 
  SD\_old\_Fem\_GP\_1 & 2.56049 & 4 & (0.03, 5) & OK & 0.12 & None \\ 
  Wtlen\_1\_Fem & 1.003e-05 & -99 & (0, 3) &  &  & None \\ 
  Wtlen\_2\_Fem & 3.1026 & -99 & (2, 4) &  &  & None \\ 
  Mat50\%\_Fem & 32.1 & -99 & (20, 40) &  &  & None \\ 
  Mat\_slope\_Fem & -1 & -99 & (-2, 4) &  &  & None \\ 
  Eggs\_scalar\_Fem & 8.66e-10 & -99 & (0, 6) &  &  & None \\ 
  Eggs\_exp\_len\_Fem & 4.9767 & -99 & (-3, 5) &  &  & None \\ 
  NatM\_p\_1\_Mal\_GP\_1 & 0.054 & -5 & (0, 0.3) &  &  & Normal (0.05, 0.1) \\ 
  L\_at\_Amin\_Mal\_GP\_1 & 20.7538 & -2 & (6, 68) &  &  & None \\ 
  L\_at\_Amax\_Mal\_GP\_1 & 38.9253 & 2 & (13, 122) & OK & 0.00 & None \\ 
  VonBert\_K\_Mal\_GP\_1 & 0.198 & 3 & (0.04, 1.09) & OK & 0.03 & None \\ 
  SD\_young\_Mal\_GP\_1 & 1.34872 & -5 & (0, 742.07) &  &  & None \\ 
  SD\_old\_Mal\_GP\_1 & 2.28 & 5 & (0, 742.07) & OK & 0.06 & None \\ 
  Wtlen\_1\_Mal & 9.881e-06 & -99 & (0, 3) &  &  & None \\ 
  Wtlen\_2\_Mal & 3.1039 & -99 & (2, 4) &  &  & None \\ 
  CohortGrowDev & 1 & -1 & (1, 1) &  &  & None \\ 
  FracFemale\_GP\_1 & 0.5 & -99 & (0.01, 0.99) &  &  & None \\ 
  SR\_LN(R0) & 9.4018 & 1 & (5, 20) & OK & 0.15 & None \\ 
  SR\_BH\_steep & 0.5 & -2 & (0.2, 1) &  &  & Full\_Beta (0.72, 0.15) \\ 
  SR\_sigmaR & 0.7 & -6 & (0.5, 1.2) &  &  & None \\ 
  SR\_regime & 0 & -99 & (-5, 5) &  &  & None \\ 
  SR\_autocorr & 0 & -99 & (0, 2) &  &  & None \\ 
  Early\_InitAge\_18 & 0.00265625 & 3 & (-6, 6) & act & 0.70 & dev (NA, NA) \\ 
  Early\_InitAge\_17 & 0.00279222 & 3 & (-6, 6) & act & 0.70 & dev (NA, NA) \\ 
  Early\_InitAge\_16 & 0.00293308 & 3 & (-6, 6) & act & 0.70 & dev (NA, NA) \\ 
  Early\_InitAge\_15 & 0.00307856 & 3 & (-6, 6) & act & 0.70 & dev (NA, NA) \\ 
  Early\_InitAge\_14 & 0.00322832 & 3 & (-6, 6) & act & 0.70 & dev (NA, NA) \\ 
  Early\_InitAge\_13 & 0.00338192 & 3 & (-6, 6) & act & 0.70 & dev (NA, NA) \\ 
  Early\_InitAge\_12 & 0.00353873 & 3 & (-6, 6) & act & 0.70 & dev (NA, NA) \\ 
  Early\_InitAge\_11 & 0.00369798 & 3 & (-6, 6) & act & 0.70 & dev (NA, NA) \\ 
  Early\_InitAge\_10 & 0.0038583 & 3 & (-6, 6) & act & 0.70 & dev (NA, NA) \\ 
  Early\_InitAge\_9 & 0.00401862 & 3 & (-6, 6) & act & 0.70 & dev (NA, NA) \\ 
  Early\_InitAge\_8 & 0.00417628 & 3 & (-6, 6) & act & 0.70 & dev (NA, NA) \\ 
  Early\_InitAge\_7 & 0.00432953 & 3 & (-6, 6) & act & 0.70 & dev (NA, NA) \\ 
  Early\_InitAge\_6 & 0.00448031 & 3 & (-6, 6) & act & 0.70 & dev (NA, NA) \\ 
  Early\_InitAge\_5 & 0.00463308 & 3 & (-6, 6) & act & 0.70 & dev (NA, NA) \\ 
  Early\_InitAge\_4 & 0.00479034 & 3 & (-6, 6) & act & 0.70 & dev (NA, NA) \\ 
  Early\_InitAge\_3 & 0.00495221 & 3 & (-6, 6) & act & 0.70 & dev (NA, NA) \\ 
  Early\_InitAge\_2 & 0.00511809 & 3 & (-6, 6) & act & 0.70 & dev (NA, NA) \\ 
  Early\_InitAge\_1 & 0.00528801 & 3 & (-6, 6) & act & 0.70 & dev (NA, NA) \\ 
  LnQ\_base\_POP(4) & -0.217115 & -1 & (-15, 15) &  &  & None \\ 
  LnQ\_base\_AFSCSlope(6) & -2.67499 & -1 & (-15, 15) &  &  & None \\ 
  LnQ\_base\_NWFSCSlope(7) & -3.04717 & -1 & (-15, 15) &  &  & None \\ 
  LnQ\_base\_NWFSCcombo(8) & -2.73349 & -1 & (-15, 15) &  &  & None \\ 
  Q\_extraSD\_NWFSCcombo(8) & 0.01779 & 2 & (0, 0.5) & OK & 0.07 & None \\ 
  SizeSel\_P1\_Fishery(1) & 37.0908 & 1 & (20, 45) & OK & 0.13 & None \\ 
  SizeSel\_P2\_Fishery(1) & -5 & -2 & (-6, 4) &  &  & None \\ 
  SizeSel\_P3\_Fishery(1) & 3.47683 & 3 & (-1, 9) & OK & 0.13 & None \\ 
  SizeSel\_P4\_Fishery(1) & -1.65 & -3 & (-9, 9) &  &  & None \\ 
  SizeSel\_P5\_Fishery(1) & -3.2223 & 4 & (-5, 9) & OK & 0.20 & None \\ 
  SizeSel\_P6\_Fishery(1) & 0.00856061 & 4 & (-5, 9) & OK & 0.28 & None \\ 
  Retain\_P1\_Fishery(1) & 28.4526 & 1 & (15, 45) & OK & 0.36 & None \\ 
  Retain\_P2\_Fishery(1) & 0.985719 & 1 & (0.1, 10) & OK & 0.13 & None \\ 
  Retain\_P3\_Fishery(1) & 7.11797 & 1 & (-10, 10) & OK & 1.72 & None \\ 
  Retain\_P4\_Fishery(1) & 0 & -3 & (0, 0) &  &  & None \\ 
  SizeSel\_P1\_ASHOP(2) & 49.4956 & 1 & (20, 49.5) & HI & 0.14 & None \\ 
  SizeSel\_P2\_ASHOP(2) & -5 & -2 & (-6, 4) &  &  & None \\ 
  SizeSel\_P3\_ASHOP(2) & 5.15704 & 3 & (-1, 9) & OK & 0.18 & None \\ 
  SizeSel\_P4\_ASHOP(2) & 1 & -3 & (-1, 9) &  &  & None \\ 
  SizeSel\_P5\_ASHOP(2) & -4.35 & -4 & (-9, 9) &  &  & None \\ 
  SizeSel\_P6\_ASHOP(2) & 999 & -2 & (-5, 999) &  &  & None \\ 
  SizeSel\_P1\_POP(4) & 25.1237 & 1 & (20, 70) & OK & 2.28 & None \\ 
  SizeSel\_P2\_POP(4) & 11.654 & 3 & (0.001, 50) & OK & 4.14 & None \\ 
  SizeSel\_P1\_AFSCSlope(6) & 21.5056 & 1 & (20, 45) & OK & 6.26 & None \\ 
  SizeSel\_P2\_AFSCSlope(6) & -5 & -2 & (-6, 4) &  &  & None \\ 
  SizeSel\_P3\_AFSCSlope(6) & 1.14059 & 3 & (-1, 9) & OK & 6.75 & None \\ 
  SizeSel\_P4\_AFSCSlope(6) & 1 & -3 & (-1, 9) &  &  & None \\ 
  SizeSel\_P5\_AFSCSlope(6) & -9 & -4 & (-9, 9) &  &  & None \\ 
  SizeSel\_P6\_AFSCSlope(6) & 999 & -2 & (-5, 999) &  &  & None \\ 
  SizeSel\_P1\_NWFSCSlope(7) & 35.9371 & 1 & (20, 45) & OK & 2.36 & None \\ 
  SizeSel\_P2\_NWFSCSlope(7) & -5 & -2 & (-6, 4) &  &  & None \\ 
  SizeSel\_P3\_NWFSCSlope(7) & 1.84591 & 3 & (-1, 9) & OK & 1.93 & None \\ 
  SizeSel\_P4\_NWFSCSlope(7) & 1 & -3 & (-1, 9) &  &  & None \\ 
  SizeSel\_P5\_NWFSCSlope(7) & -9 & -4 & (-9, 9) &  &  & None \\ 
  SizeSel\_P6\_NWFSCSlope(7) & 999 & -2 & (-5, 999) &  &  & None \\ 
  SizeSel\_P1\_NWFSCcombo(8) & 21.1613 & 1 & (18, 49.5) & OK & 4.09 & None \\ 
  SizeSel\_P2\_NWFSCcombo(8) & -5 & -2 & (-6, 4) &  &  & None \\ 
  SizeSel\_P3\_NWFSCcombo(8) & 3.02794 & 3 & (-1, 9) & OK & 2.21 & None \\ 
  SizeSel\_P4\_NWFSCcombo(8) & 1 & -3 & (-1, 9) &  &  & None \\ 
  SizeSel\_P5\_NWFSCcombo(8) & -9 & -4 & (-9, 9) &  &  & None \\ 
  SizeSel\_P6\_NWFSCcombo(8) & 999 & -2 & (-5, 999) &  &  & None \\ 
  SizeSel\_P6\_Fishery(1)\_BLK4repl\_1918 & 1.50688 & 2 & (-5, 9) & OK & 0.69 & None \\ 
  Retain\_P2\_Fishery(1)\_BLK2add\_1918 & 1.26058 & 2 & (0.1, 10) & OK & 0.13 & None \\ 
  Retain\_P3\_Fishery(1)\_BLK1repl\_1918 & 9.58287 & 4 & (-10, 10) & OK & 11.06 & None \\ 
  Retain\_P3\_Fishery(1)\_BLK1repl\_1992 & 2.58069 & 4 & (-10, 10) & OK & 0.48 & None \\ 
  Retain\_P3\_Fishery(1)\_BLK1repl\_2002 & 1.91825 & 4 & (-10, 10) & OK & 0.15 & None \\ 
  Retain\_P3\_Fishery(1)\_BLK1repl\_2008 & 0.689664 & 4 & (-10, 10) & OK & 0.29 & None \\ 
  Retain\_P3\_Fishery(1)\_BLK1repl\_2009 & 0.0280968 & 4 & (-10, 10) & OK & 0.24 & None \\ 
   \hline
\hline
\label{tab:model_params}
\end{longtable}
\end{landscape}

\newpage

\FloatBarrier

\begin{table}[ht]
\centering
\caption{Results from 50 jitters from the base model.} 
\label{tab:jitter}
\begin{tabular}{>{\raggedright}p{2in}>{\centering}p{1in}}
  \hline
Status & Base.Model \\ 
  \hline
Returned to base case &  27 \\ 
  Found local minimum &  23 \\ 
  Found better solution &   0 \\ 
  Total &  50 \\ 
   \hline
\end{tabular}
\end{table}

\FloatBarrier  

\begin{table}[ht]
\centering
\caption{Likelihood components from the base model} 
\label{tab:like}
\begin{tabular}{>{\raggedright}p{2in}>{\centering}p{1.0in}}
  \hline
Likelihood Component & Value \\ 
  \hline
Total & 1639.13 \\ 
  Survey & -13.51 \\ 
  Discard & -34.57 \\ 
  Length-frequency data & 143.5 \\ 
  Age-frequency data & 1531.08 \\ 
  Recruitment & 11.62 \\ 
  Forecast Recruitment & 0 \\ 
  Parameter Priors & 1 \\ 
   \hline
\end{tabular}
\end{table}

\FloatBarrier

\begin{table}[ht]
\centering
\caption{Summary of reference 
                                        points and management quantities for the 
                                        base case.} 
\label{tab:Ref_pts}
\begin{tabular}{>{\raggedright}p{4.1in}>{\centering}p{.65in}>{\centering}p{1.4in}}
  \hline
\textbf{Quantity} & \textbf{Estimate} & \textbf{\~95\%  Confidence Interval} \\ 
  \hline
Unfished spawning output (million eggs) & 6889.2 &   4860.7 -   8917.6 \\ 
  Unfished age 3+ biomass (mt) & 147286 & 104000.8 - 190571.2 \\ 
  Unfished recruitment (R0, thousands) & 12110.2 &   9046.1 -  16212.1 \\ 
  Spawning output(2017 million eggs) & 5280.4 &   2407.4 -   8153.3 \\ 
  Depletion (2017) & 0.766 &    0.556 -    0.977 \\ 
  \textbf{$\text{Reference points based on } \mathbf{SB_{40\%}}$} &  &  \\ 
  Proxy spawning output ($B_{40\%}$) & 2755.7 &   1944.3 -     3567 \\ 
  SPR resulting in $B_{40\%}$ ($SPR_{B40\%}$) & 0.55 &     0.55 -     0.55 \\ 
  Exploitation rate resulting in $B_{40\%}$ & 0.028 &    0.028 -    0.029 \\ 
  Yield with $SPR_{B40\%}$ at $B_{40\%}$ (mt) & 1808.3 &   1278.2 -   2338.4 \\ 
  \textbf{\textit{Reference points based on SPR proxy for MSY}} &  &  \\ 
  Spawning output & 2296.4 &   1620.2 -   2972.5 \\ 
  $SPR_{proxy}$ & 0.5 &  \\ 
  Exploitation rate corresponding to $SPR_{proxy}$ & 0.033 &    0.033 -    0.034 \\ 
  Yield with $SPR_{proxy}$ at $SB_{SPR}$ (mt) & 1822.5 &   1288.5 -   2356.5 \\ 
  \textbf{\textit{Reference points based on estimated MSY values}} &  &  \\ 
  Spawning output at $MSY$ ($SB_{MSY}$) & 2425 &   1708.1 -   3141.8 \\ 
  $SPR_{MSY}$ & 0.514 &    0.512 -    0.516 \\ 
  Exploitation rate at $MSY$ & 0.032 &    0.031 -    0.032 \\ 
  $MSY$ (mt)  & 1825.3 &   1290.4 -   2360.2 \\ 
   \hline
\end{tabular}
\end{table}

\newpage

\FloatBarrier

\pagebreak

\begingroup\fontsize{11pt}{11pt}\selectfont

\begin{longtable}{c>{\centering}p{.5in}>{\centering}p{.65in}>{\centering}p{.6in}>{\centering}p{.6in}>{\centering}p{.5in}>{\centering}p{.60in}>{\centering}p{.45in}c}
\caption{Time-series of population estimates from the base model.} \\ 
  \hline
Year & Total biomass (mt) & Spawning output (million eggs) & Summary biomass 3+ & Relative biomass & Age-0 recruits & Estimated total catch (mt) & 1-SPR & Exploit. rate \\ 
  \hline \endhead  \hline
1918 & 147,517 & 6,896 & 146,768 & 1.00 &  12,180 & 1 & 0 & 0 \\ 
  1919 & 147,536 & 6,897 & 146,788 & 1.00 &  12,182 & 0 & 0 & 0 \\ 
  1920 & 147,557 & 6,898 & 146,808 & 1.00 &  12,185 & 0 & 0 & 0 \\ 
  1921 & 147,578 & 6,899 & 146,829 & 1.00 &  12,188 & 0 & 0 & 0 \\ 
  1922 & 147,600 & 6,900 & 146,851 & 1.00 &  12,190 & 0 & 0 & 0 \\ 
  1923 & 147,622 & 6,901 & 146,873 & 1.00 &  12,193 & 0 & 0 & 0 \\ 
  1924 & 147,645 & 6,902 & 146,896 & 1.00 &  12,195 & 1 & 0 & 0 \\ 
  1925 & 147,668 & 6,903 & 146,918 & 1.00 &  12,197 & 1 & 0 & 0 \\ 
  1926 & 147,691 & 6,904 & 146,941 & 1.00 &  12,200 & 1 & 0 & 0 \\ 
  1927 & 147,714 & 6,905 & 146,964 & 1.00 &  12,202 & 1 & 0 & 0 \\ 
  1928 & 147,737 & 6,906 & 146,987 & 1.00 &  12,203 & 1 & 0 & 0 \\ 
  1929 & 147,761 & 6,907 & 147,011 & 1.00 &  12,205 & 1 & 0 & 0 \\ 
  1930 & 147,785 & 6,908 & 147,035 & 1.00 &  12,206 & 1 & 0 & 0 \\ 
  1931 & 147,809 & 6,909 & 147,059 & 1.00 &  12,207 & 1 & 0 & 0 \\ 
  1932 & 147,834 & 6,910 & 147,084 & 1.00 &  12,208 & 1 & 0 & 0 \\ 
  1933 & 147,859 & 6,911 & 147,109 & 1.00 &  12,209 & 1 & 0 & 0 \\ 
  1934 & 147,883 & 6,913 & 147,133 & 1.00 &  12,212 & 3 & 0 & 0 \\ 
  1935 & 147,906 & 6,914 & 147,155 & 1.00 &  12,216 & 8 & 0 & 0 \\ 
  1936 & 147,923 & 6,914 & 147,172 & 1.00 &  12,225 & 2 & 0 & 0 \\ 
  1937 & 147,946 & 6,916 & 147,195 & 1.00 &  12,239 & 3 & 0 & 0 \\ 
  1938 & 147,969 & 6,917 & 147,218 & 1.00 &  12,263 & 6 & 0 & 0 \\ 
  1939 & 147,991 & 6,918 & 147,238 & 1.00 &  12,297 & 10 & 0 & 0 \\ 
  1940 & 148,011 & 6,918 & 147,256 & 1.00 &  12,394 & 23 & 0.005 & 0 \\ 
  1941 & 148,023 & 6,918 & 147,265 & 1.00 &  12,454 & 30 & 0.005 & 0 \\ 
  1942 & 148,037 & 6,918 & 147,274 & 1.00 &  12,526 & 47 & 0.01 & 0 \\ 
  1943 & 148,047 & 6,917 & 147,281 & 1.00 &  12,609 & 561 & 0.09 & 0.004 \\ 
  1944 & 147,567 & 6,891 & 146,796 & 1.00 &  12,689 & 927 & 0.14 & 0.006 \\ 
  1945 & 146,760 & 6,847 & 145,984 & 0.99 &  12,772 & 2188 & 0.285 & 0.015 \\ 
  1946 & 144,758 & 6,743 & 143,977 & 0.98 &  12,839 & 1070 & 0.16 & 0.007 \\ 
  1947 & 143,949 & 6,695 & 143,163 & 0.97 &  12,960 & 568 & 0.09 & 0.004 \\ 
  1948 & 143,701 & 6,675 & 142,910 & 0.97 &  13,128 & 688 & 0.11 & 0.005 \\ 
  1949 & 143,387 & 6,651 & 142,588 & 0.96 &  13,323 & 905 & 0.14 & 0.006 \\ 
  1950 & 142,922 & 6,619 & 142,112 & 0.96 &  13,520 & 1399 & 0.205 & 0.01 \\ 
  1951 & 142,041 & 6,565 & 141,219 & 0.95 &  13,651 & 1616 & 0.23 & 0.011 \\ 
  1952 & 141,037 & 6,504 & 140,204 & 0.94 &  13,628 & 2394 & 0.315 & 0.017 \\ 
  1953 & 139,363 & 6,408 & 138,525 & 0.93 &  13,345 & 1772 & 0.255 & 0.013 \\ 
  1954 & 138,415 & 6,346 & 137,584 & 0.92 &  12,835 & 2559 & 0.335 & 0.019 \\ 
  1955 & 136,769 & 6,251 & 135,959 & 0.91 &  12,144 & 2000 & 0.285 & 0.015 \\ 
  1956 & 135,740 & 6,188 & 134,964 & 0.90 &  11,352 & 3192 & 0.4 & 0.024 \\ 
  1957 & 133,540 & 6,072 & 132,809 & 0.88 &  10,644 & 2734 & 0.365 & 0.021 \\ 
  1958 & 131,783 & 5,985 & 131,099 & 0.87 &  10,134 & 2151 & 0.31 & 0.016 \\ 
  1959 & 130,539 & 5,932 & 129,894 & 0.86 &   9,914 & 1262 & 0.205 & 0.01 \\ 
  1960 & 130,066 & 5,925 & 129,446 & 0.86 &  10,125 & 2364 & 0.33 & 0.018 \\ 
  1961 & 128,365 & 5,864 & 127,751 & 0.85 &  10,850 & 3321 & 0.42 & 0.026 \\ 
  1962 & 125,624 & 5,755 & 124,989 & 0.83 &  11,567 & 4414 & 0.505 & 0.035 \\ 
  1963 & 121,776 & 5,588 & 121,100 & 0.81 &  10,839 & 5869 & 0.6 & 0.048 \\ 
  1964 & 116,537 & 5,343 & 115,843 & 0.77 &   9,191 & 6223 & 0.63 & 0.054 \\ 
  1965 & 111,044 & 5,076 & 110,409 & 0.74 &   8,103 & 7818 & 0.705 & 0.071 \\ 
  1966 & 104,043 & 4,729 & 103,498 & 0.69 &   7,551 & 18964 & 0.9 & 0.183 \\ 
  1967 & 86,087 & 3,840 & 85,602 & 0.56 &   7,055 & 14650 & 0.89 & 0.171 \\ 
  1968 & 72,717 & 3,175 & 72,263 & 0.46 &   7,372 & 9717 & 0.855 & 0.134 \\ 
  1969 & 64,452 & 2,766 & 64,009 & 0.40 &  10,218 & 2188 & 0.51 & 0.034 \\ 
  1970 & 63,791 & 2,732 & 63,276 & 0.40 &  16,959 & 2307 & 0.525 & 0.036 \\ 
  1971 & 63,133 & 2,698 & 62,420 & 0.39 &   7,817 & 1909 & 0.475 & 0.031 \\ 
  1972 & 63,119 & 2,683 & 62,231 & 0.39 &   5,659 & 1892 & 0.47 & 0.03 \\ 
  1973 & 63,334 & 2,665 & 62,888 & 0.39 &   5,700 & 2646 & 0.57 & 0.042 \\ 
  1974 & 62,677 & 2,608 & 62,329 & 0.38 &   5,770 & 2277 & 0.53 & 0.037 \\ 
  1975 & 62,265 & 2,569 & 61,910 & 0.37 &   7,357 & 1185 & 0.355 & 0.019 \\ 
  1976 & 62,796 & 2,595 & 62,421 & 0.38 &   5,666 & 1514 & 0.415 & 0.024 \\ 
  1977 & 62,849 & 2,629 & 62,421 & 0.38 &   7,200 & 1282 & 0.365 & 0.021 \\ 
  1978 & 62,998 & 2,678 & 62,631 & 0.39 &   5,072 & 2008 & 0.48 & 0.032 \\ 
  1979 & 62,265 & 2,680 & 61,856 & 0.39 &   5,649 & 1546 & 0.41 & 0.025 \\ 
  1980 & 61,874 & 2,684 & 61,553 & 0.39 &   5,969 & 1731 & 0.445 & 0.028 \\ 
  1981 & 61,152 & 2,668 & 60,796 & 0.39 &   7,753 & 1382 & 0.385 & 0.023 \\ 
  1982 & 60,714 & 2,662 & 60,309 & 0.39 &  11,662 & 1058 & 0.32 & 0.018 \\ 
  1983 & 60,622 & 2,667 & 60,085 & 0.39 &  10,769 & 1629 & 0.435 & 0.027 \\ 
  1984 & 60,117 & 2,641 & 59,422 & 0.38 &   8,042 & 1659 & 0.44 & 0.028 \\ 
  1985 & 59,818 & 2,608 & 59,202 & 0.38 &   7,756 & 1425 & 0.405 & 0.024 \\ 
  1986 & 59,898 & 2,581 & 59,408 & 0.37 &   8,037 & 1376 & 0.4 & 0.023 \\ 
  1987 & 60,094 & 2,555 & 59,614 & 0.37 &   7,574 & 1107 & 0.345 & 0.019 \\ 
  1988 & 60,616 & 2,551 & 60,125 & 0.37 &   9,387 & 1382 & 0.4 & 0.023 \\ 
  1989 & 60,922 & 2,550 & 60,411 & 0.37 &  16,275 & 1478 & 0.415 & 0.024 \\ 
  1990 & 61,254 & 2,561 & 60,568 & 0.37 &  15,636 & 1127 & 0.345 & 0.019 \\ 
  1991 & 62,209 & 2,592 & 61,241 & 0.38 &   6,924 & 1483 & 0.41 & 0.024 \\ 
  1992 & 63,156 & 2,604 & 62,343 & 0.38 &   4,464 & 1571 & 0.425 & 0.025 \\ 
  1993 & 64,118 & 2,608 & 63,732 & 0.38 &   4,778 & 1417 & 0.395 & 0.022 \\ 
  1994 & 65,023 & 2,621 & 64,732 & 0.38 &   9,705 & 1180 & 0.345 & 0.018 \\ 
  1995 & 65,959 & 2,656 & 65,585 & 0.39 &   9,946 & 956 & 0.29 & 0.015 \\ 
  1996 & 66,969 & 2,725 & 66,381 & 0.40 &   5,164 & 883 & 0.265 & 0.013 \\ 
  1997 & 67,979 & 2,819 & 67,446 & 0.41 &   4,736 & 718 & 0.22 & 0.011 \\ 
  1998 & 68,964 & 2,913 & 68,656 & 0.42 &   3,507 & 725 & 0.22 & 0.011 \\ 
  1999 & 69,666 & 2,982 & 69,351 & 0.43 &  21,662 & 563 & 0.175 & 0.008 \\ 
  2000 & 70,446 & 3,037 & 69,912 & 0.44 &  32,360 & 161 & 0.05 & 0.002 \\ 
  2001 & 71,921 & 3,107 & 70,473 & 0.45 &   9,819 & 297 & 0.09 & 0.004 \\ 
  2002 & 74,097 & 3,171 & 72,483 & 0.46 &   5,377 & 179 & 0.055 & 0.002 \\ 
  2003 & 76,945 & 3,230 & 76,420 & 0.47 &   2,676 & 158 & 0.05 & 0.002 \\ 
  2004 & 79,589 & 3,274 & 79,292 & 0.47 &   6,757 & 149 & 0.045 & 0.002 \\ 
  2005 & 81,950 & 3,318 & 81,728 & 0.48 &   3,265 & 78 & 0.025 & 0.001 \\ 
  2006 & 83,973 & 3,412 & 83,613 & 0.49 &   3,592 & 86 & 0.025 & 0.001 \\ 
  2007 & 85,564 & 3,571 & 85,358 & 0.52 &   3,462 & 159 & 0.045 & 0.002 \\ 
  2008 & 86,802 & 3,745 & 86,308 & 0.54 & 116,128 & 135 & 0.035 & 0.002 \\ 
  2009 & 88,561 & 3,885 & 86,803 & 0.56 &   4,731 & 194 & 0.05 & 0.002 \\ 
  2010 & 92,115 & 3,976 & 86,769 & 0.58 &   7,499 & 183 & 0.045 & 0.002 \\ 
  2011 & 98,527 & 4,032 & 98,173 & 0.58 &  15,198 & 62 & 0.015 & 0.001 \\ 
  2012 & 104,262 & 4,067 & 103,709 & 0.59 &   2,101 & 60 & 0.015 & 0.001 \\ 
  2013 & 110,043 & 4,091 & 109,254 & 0.59 &  29,027 & 58 & 0.015 & 0.001 \\ 
  2014 & 115,579 & 4,197 & 115,075 & 0.61 &   4,630 & 56 & 0.015 & 0 \\ 
  2015 & 120,592 & 4,516 & 119,187 & 0.65 &  10,661 & 61 & 0.015 & 0.001 \\ 
  2016 & 125,377 & 4,931 & 124,995 & 0.72 &  11,016 & 68 & 0.015 & 0.001 \\ 
  2017 & 129,191 & 5,280 & 128,529 & 0.77 &  11,253 & - & - & - \\ 
   \hline
\hline
\label{tab:Timeseries_mod1}
\end{longtable}

\endgroup

\FloatBarrier

\begin{sidewaystable}[ht]
\centering
\caption{Sensitivity of the base model} 
\label{tab:Sensitivity1}
\scalebox{0.9}{
\begin{tabular}{l>{\centering}p{.8in}>{\centering}p{.8in}>{\centering}p{.8in}>{\centering}p{.8in}>{\centering}p{.8in}>{\centering}p{.8in}>{\centering}p{.8in}}
  \hline
Label & Base & Harmonic weights & Steepness = 0.40 & Steepness = 0.72 & Old Maturity & Old Fecundity & 2008 Recruitment \\ 
  \hline
Total Likelihood & 1639.130 & 2441.720 & 1639.950 & 1638.250 & 1639.140 & 1639.130 & 1877.740 \\ 
  Survey Likelihood & -13.514 & -13.870 & -13.676 & -13.421 & -13.515 & -13.509 & -12.863 \\ 
  Discard Likelihood & -34.574 & -17.102 & -34.425 & -34.744 & -34.578 & -34.578 & 56.929 \\ 
  Length Likelihood & 143.504 & 742.387 & 143.129 & 143.932 & 143.501 & 143.516 & 191.232 \\ 
  Age Likelihood & 1531.080 & 1711.000 & 1531.410 & 1530.680 & 1531.100 & 1531.070 & 1636.830 \\ 
  Recruitment Likelihood & 11.618 & 18.273 & 11.623 & 11.661 & 11.620 & 11.616 & 4.595 \\ 
  Forecast Recruitment Likelihood & 0.000 & 0.000 & 0.000 & 0.000 & 0.000 & 0.000 & 0.000 \\ 
  Parameter Priors Likelihood & 1.000 & 1.000 & 1.870 & 0.125 & 1.000 & 1.000 & 1.000 \\ 
  Parameter Deviation Likelihood & 0.000 & 0.000 & 0.000 & 0.000 & 0.000 & 0.000 & 0.000 \\ 
  log(R0) & 9.402 & 9.270 & 9.341 & 9.450 & 9.401 & 9.403 & 9.392 \\ 
  SB Virgin & 6889.170 & 6155.290 & 6475.990 & 7239.700 & 6758.200 & 8213.080 & 6908.240 \\ 
  SB 2017 & 5280.380 & 3723.450 & 3585.220 & 7002.320 & 5280.670 & 6473.350 & 4074.950 \\ 
  Depletion 2017 & 0.766 & 0.605 & 0.554 & 0.967 & 0.781 & 0.788 & 0.590 \\ 
  Total Yield - SPR 50 & 1822.490 & 1620.140 & 1028.650 & 2560.050 & 1818.760 & 1844.750 & 1823.380 \\ 
  Steepness & 0.500 & 0.500 & 0.400 & 0.720 & 0.500 & 0.500 & 0.500 \\ 
  Natural Mortality - Female & 0.054 & 0.054 & 0.054 & 0.054 & 0.054 & 0.054 & 0.054 \\ 
  Length at Amin - Female & 20.754 & 20.649 & 20.753 & 20.756 & 20.753 & 20.754 & 20.373 \\ 
  Length at Amax - Female & 41.601 & 41.726 & 41.596 & 41.611 & 41.601 & 41.601 & 41.727 \\ 
  Von Bert. k - Female & 0.167 & 0.169 & 0.167 & 0.167 & 0.167 & 0.167 & 0.175 \\ 
  SD young - Female & 1.349 & 1.336 & 1.349 & 1.348 & 1.349 & 1.349 & 1.397 \\ 
  SD old - Female & 2.560 & 2.772 & 2.562 & 2.558 & 2.561 & 2.560 & 2.516 \\ 
  Natural Mortality - Male & 0.054 & 0.054 & 0.054 & 0.054 & 0.054 & 0.054 & 0.054 \\ 
  Length at Amin - Male & 20.754 & 20.649 & 20.753 & 20.756 & 20.753 & 20.754 & 20.373 \\ 
  Length at Amax - Male & 38.925 & 38.933 & 38.917 & 38.938 & 38.926 & 38.925 & 39.087 \\ 
  Von Bert. k - Male & 0.198 & 0.201 & 0.198 & 0.197 & 0.198 & 0.198 & 0.204 \\ 
  SD young - Male & 1.349 & 1.336 & 1.349 & 1.348 & 1.349 & 1.349 & 1.397 \\ 
  SD old - Male & 2.280 & 2.587 & 2.281 & 2.279 & 2.280 & 2.280 & 2.203 \\ 
   \hline
\end{tabular}
}
\end{sidewaystable}

\FloatBarrier 

\begin{sidewaystable}[ht]
\centering
\caption{Sensitivity of the base model} 
\label{tab:Sensitivity2}
\scalebox{0.9}{
\begin{tabular}{l>{\centering}p{.8in}>{\centering}p{.8in}>{\centering}p{.8in}>{\centering}p{.8in}>{\centering}p{.8in}>{\centering}p{.8in}>{\centering}p{.8in}}
  \hline
Label & Base & Include Triennial & Only Triennial & Include CPUE & Canadian Data & WA Research Lengths & OR Special Projects \\ 
  \hline
Total Likelihood & 1639.13 & 1665.44 & 164.52 & 1639.13 & 1732.43 & 1661.35 & 1704.70 \\ 
  Survey Likelihood & -13.51 & -12.94 & -4.72 & -13.51 & -13.75 & -13.52 & -13.56 \\ 
  Discard Likelihood & -34.57 & -34.44 & -41.41 & -34.57 & -34.34 & -34.53 & -33.98 \\ 
  Length Likelihood & 143.50 & 149.40 & 103.50 & 143.50 & 183.12 & 164.57 & 171.18 \\ 
  Age Likelihood & 1531.08 & 1550.29 & 98.85 & 1531.08 & 1583.84 & 1532.08 & 1566.68 \\ 
  Recruitment Likelihood & 11.62 & 12.11 & 5.13 & 11.62 & 12.53 & 11.73 & 13.36 \\ 
  Forecast Recruitment Likelihood & 0.00 & 0.00 & 0.00 & 0.00 & 0.00 & 0.00 & 0.00 \\ 
  Parameter Priors Likelihood & 1.00 & 1.00 & 3.16 & 1.00 & 1.00 & 1.00 & 1.00 \\ 
  Parameter Deviation Likelihood & 0.00 & 0.00 & 0.00 & 0.00 & 0.00 & 0.00 & 0.00 \\ 
  log(R0) & 9.40 & 9.34 & 9.18 & 9.40 & 9.40 & 9.38 & 9.33 \\ 
  SB Virgin & 6889.17 & 6509.19 & 5494.23 & 6889.17 & 6932.47 & 6705.96 & 6447.45 \\ 
  SB 2017 & 5280.38 & 4763.81 & 614.09 & 5280.38 & 5046.25 & 5015.16 & 4700.25 \\ 
  Depletion 2017 & 0.77 & 0.73 & 0.11 & 0.77 & 0.73 & 0.75 & 0.73 \\ 
  Total Yield - SPR 50 & 1822.49 & 1721.47 & 26.65 & 1822.49 & 1856.88 & 1788.49 & 1699.53 \\ 
  Steepness & 0.50 & 0.50 & 0.33 & 0.50 & 0.50 & 0.50 & 0.50 \\ 
  Natural Mortality - Female & 0.05 & 0.05 & 0.05 & 0.05 & 0.05 & 0.05 & 0.05 \\ 
  Length at Amin - Female & 20.75 & 20.76 & 20.75 & 20.75 & 20.74 & 20.74 & 20.78 \\ 
  Length at Amax - Female & 41.60 & 41.60 & 41.60 & 41.60 & 41.66 & 41.53 & 41.64 \\ 
  Von Bert. k - Female & 0.17 & 0.17 & 0.17 & 0.17 & 0.17 & 0.17 & 0.17 \\ 
  SD young - Female & 1.35 & 1.35 & 1.35 & 1.35 & 1.35 & 1.35 & 1.34 \\ 
  SD old - Female & 2.56 & 2.56 & 2.56 & 2.56 & 2.55 & 2.56 & 2.58 \\ 
  Natural Mortality - Male & 0.05 & 0.05 & 0.05 & 0.05 & 0.05 & 0.05 & 0.05 \\ 
  Length at Amin - Male & 20.75 & 20.76 & 20.75 & 20.75 & 20.74 & 20.74 & 20.78 \\ 
  Length at Amax - Male & 38.93 & 38.92 & 38.92 & 38.93 & 38.95 & 38.89 & 38.98 \\ 
  Von Bert. k - Male & 0.20 & 0.20 & 0.20 & 0.20 & 0.20 & 0.20 & 0.20 \\ 
  SD young - Male & 1.35 & 1.35 & 1.35 & 1.35 & 1.35 & 1.35 & 1.34 \\ 
  SD old - Male & 2.28 & 2.28 & 2.28 & 2.28 & 2.28 & 2.29 & 2.34 \\ 
   \hline
\end{tabular}
}
\end{sidewaystable}

\FloatBarrier 

\begin{table}[ht]
\centering
\caption{Data weights applied when using harmonic data weighting.} 
\label{tab:harm}
\begin{tabular}{>{\raggedright}p{2in}>{\centering}p{.7in}>{\centering}p{.7in}}
  \hline
Fleet & Lengths & Ages \\ 
  \hline
Fishery & 0.361 & 0.77 \\ 
  At-sea hake & 0.621 & 0.14 \\ 
  Pacific ocean perch survey  & 1.000 & 1 \\ 
  AFSC slope survey & 0.696 & 1 \\ 
  NWFSC slope survey & 0.463 & - \\ 
  NWFSC shelf-slope survey & 0.549 & 0.348 \\ 
   \hline
\end{tabular}
\end{table}

\FloatBarrier 

\newpage

\begin{table}[ht]
\centering
\caption{Projection of potential
                                         OFL, spawning biomass, and depletion for the
                                         base case model.} 
\label{tab:Forecast_mod1}
\begin{tabular}{c>{\centering}p{1in}>{\centering}p{1in}>{\centering}p{1in}>{\centering}p{1in}}
  \hline
Year & OFL (mt) & ACL (mt) & Spawning Output & Depletion (\%) \\ 
  \hline
2019 & 4753 & 4340 & 5741 & 83.3 \\ 
  2020 & 4632 & 4229 & 5745 & 83.4 \\ 
  2021 & 4499 & 4108 & 5723 & 83.1 \\ 
  2022 & 4364 & 3984 & 5666 & 82.2 \\ 
  2023 & 4230 & 3862 & 5586 & 81.1 \\ 
  2024 & 4105 & 3748 & 5494 & 79.8 \\ 
  2025 & 3991 & 3644 & 5395 & 78.3 \\ 
  2026 & 3889 & 3551 & 5292 & 76.8 \\ 
  2027 & 3797 & 3467 & 5188 & 75.3 \\ 
  2028 & 3712 & 3389 & 5084 & 73.8 \\ 
   \hline
\end{tabular}
\end{table}

\FloatBarrier

\begin{table}[ht]
\centering
\caption{Summary of 10-year 
                                             projections beginning in 2019 
                                             for alternate states of nature based on 
                                             an axis of uncertainty for the base model. 
                                             Columns range over low, mid, and high
                                             states of nature over natural mortality, and rows range over different 
                                             assumptions of catch levels. An entry of "--" 
                                             indicates that the stock is driven to very low 
                                             abundance under the particular scenario.} 
\label{tab:Decision_table_mod1_back}
\scalebox{0.85}{
\begin{tabular}{l|cc|>{\centering}p{.7in}c|>{\centering}p{.7in}c|>{\centering}p{.7in}c}
   \multicolumn{3}{c}{}  &  \multicolumn{2}{c}{} 
                               & \multicolumn{2}{c}{\textbf{States of nature}} 
                               & \multicolumn{2}{c}{} \\
  \multicolumn{3}{c}{}  &  \multicolumn{2}{c}{M = 0.04725} 
                               & \multicolumn{2}{c}{M = 0.054} 
                               &  \multicolumn{2}{c}{M = 0.0595} \\
 \hline
 & Year & Catch & Spawning Output & Depletion (\%) & Spawning Output & Depletion (\%) & Spawning Output & Depletion (\%) \\ 
  \hline
 & 2019 & 4340 & 3944 & 62.9 & 5741 & 83.3 & 7505 & 96.8 \\ 
   & 2020 & 4229 & 3909 & 62.4 & 5745 & 83.4 & 7542 & 97.3 \\ 
   & 2021 & 4108 & 3858 & 61.6 & 5723 & 83.1 & 7546 & 97.3 \\ 
  ABC & 2022 & 3984 & 3784 & 60.4 & 5666 & 82.2 & 7503 & 96.8 \\ 
   & 2023 & 3862 & 3695 & 59.0 & 5586 & 81.1 & 7427 & 95.8 \\ 
   & 2024 & 3748 & 3600 & 57.4 & 5494 & 79.7 & 7332 & 94.6 \\ 
   & 2025 & 3644 & 3502 & 55.9 & 5395 & 78.3 & 7226 & 93.2 \\ 
   & 2026 & 3551 & 3404 & 54.3 & 5292 & 76.8 & 7113 & 91.8 \\ 
   & 2027 & 3467 & 3308 & 52.8 & 5188 & 75.3 & 6996 & 90.3 \\ 
   & 2028 & 3389 & 3213 & 51.3 & 5084 & 73.8 & 6879 & 88.7 \\ 
   \hline
 & 2019 & 1822 & 3944 & 62.9 & 5741 & 83.3 & 7505 & 96.8 \\ 
   & 2020 & 1822 & 4022 & 64.2 & 5857 & 85.0 & 7654 & 98.7 \\ 
   & 2021 & 1822 & 4083 & 65.1 & 5946 & 86.3 & 7768 & 100.2 \\ 
  SPR50 & 2022 & 1822 & 4117 & 65.7 & 5996 & 87.0 & 7830 & 101.0 \\ 
   & 2023 & 1822 & 4131 & 65.9 & 6016 & 87.3 & 7852 & 101.3 \\ 
   & 2024 & 1822 & 4133 & 65.9 & 6017 & 87.3 & 7848 & 101.2 \\ 
   & 2025 & 1822 & 4125 & 65.8 & 6004 & 87.1 & 7824 & 100.9 \\ 
   & 2026 & 1822 & 4110 & 65.6 & 5979 & 86.8 & 7786 & 100.4 \\ 
   & 2027 & 1822 & 4090 & 65.3 & 5947 & 86.3 & 7736 & 99.8 \\ 
   & 2028 & 1822 & 4067 & 64.9 & 5908 & 85.8 & 7679 & 99.1 \\ 
   \hline
\end{tabular}
}
\end{table}

\clearpage

\section{Figures}\label{figures}

\FloatBarrier

\begin{figure}
\centering
\includegraphics{r4ss/plots_mod1/catch2 landings stacked.png}
\caption{Total catches Pacific ocean perch through 2016.
\label{fig:Catch}}
\end{figure}

\FloatBarrier

\begin{figure}
\centering
\includegraphics{Figures/data_plot_wo_tri.png}
\caption{Summary of data sources used in the base model.
\label{fig:data_plot}}
\end{figure}

\FloatBarrier

\begin{figure}
\centering
\includegraphics{Figures/Index_Data_4.png}
\caption{Fishery-dependent and fishery-independent indices for Pacific
ocean perch. \label{fig:indices}}
\end{figure}

\FloatBarrier

\begin{figure}
\centering
\includegraphics{Figures/Q-Q_plot_combo.jpg}
\caption{Q-Q plots for the VAST lognormal distribution for the NWFSC
shelf-slope survey. \label{fig:nw_qq}}
\end{figure}

\FloatBarrier

\begin{figure}
\centering
\includegraphics{r4ss/plots_mod1/comp_lendat_bubflt8mkt0.png}
\caption{NWFSC shelf-slope survey length frequency distributions for
Pacific ocean perch. \label{fig:nw_Length}}
\end{figure}

\FloatBarrier

\begin{figure}
\centering
\includegraphics{r4ss/plots_mod1/comp_gstagedat_bubflt8mkt0.png}
\caption{NWFSC shelf-slope survey age frequency distributions for
Pacific ocean perch. \label{fig:nw_Age}}
\end{figure}

\FloatBarrier

\begin{figure}
\centering
\includegraphics{Figures/Q-Q_plot_nw_slope_gammaECE.jpg}
\caption{Q-Q plots for the VAST lognormal distribution for the NWFSC
slope survey. \label{fig:nw_slope_qq}}
\end{figure}

\FloatBarrier

\begin{figure}
\centering
\includegraphics{r4ss/plots_mod1/comp_lendat_bubflt7mkt0.png}
\caption{NWFSC slope survey length frequency distributions for Pacific
ocean perch. \label{fig:nw_slope_Length}}
\end{figure}

\FloatBarrier

\begin{figure}
\centering
\includegraphics{r4ss/plots_mod1/comp_agedat_bubflt7mkt0.png}
\caption{NWFSC slope survey age frequency distributions for Pacific
ocean perch. \label{fig:nw_slope_Age}}
\end{figure}

\FloatBarrier

\begin{figure}
\centering
\includegraphics{Figures/Q-Q_plot_afsc.jpg}
\caption{Q-Q plots for the VAST lognormal distribution for the AFSC
slope survey. \label{fig:afsc_qq}}
\end{figure}

\FloatBarrier

\begin{figure}
\centering
\includegraphics{r4ss/plots_mod1/comp_lendat_bubflt6mkt0.png}
\caption{AFSC slope survey length frequency distributions for Pacific
ocean perch. \label{fig:afsc_Length}}
\end{figure}

\FloatBarrier

\begin{figure}
\centering
\includegraphics{Figures/Q-Q_plot_pop.jpg}
\caption{Q-Q plots for the VAST lognormal distribution for the Pacific
ocean perch survey. \label{fig:pop_qq}}
\end{figure}

\FloatBarrier

\begin{figure}
\centering
\includegraphics{r4ss/plots_mod1/comp_lendat_bubflt4mkt0.png}
\caption{Pacific ocean perch survey length frequency distributions for
Pacific ocean perch. \label{fig:POP_Length}}
\end{figure}

\FloatBarrier

\begin{figure}
\centering
\includegraphics{r4ss/plots_mod1/comp_agedat_bubflt4mkt0.png}
\caption{Pacific ocean perch survey age frequency distributions for
Pacific ocean perch. \label{fig:POP_Age}}
\end{figure}

\FloatBarrier

\begin{figure}
\centering
\includegraphics{r4ss/plots_mod1/comp_lendat_bubflt1mkt1.png}
\caption{Discard length frequency distributions from WCGOP for Pacific
ocean perch. \label{fig:WCGOP_discard}}
\end{figure}

\FloatBarrier

\begin{figure}
\centering
\includegraphics{r4ss/plots_mod1/comp_lendat_bubflt1mkt2_page4.png}
\caption{Commercial fishery length frequency distributions for Pacific
ocean perch. \label{fig:Comm_Length}}
\end{figure}

\FloatBarrier

\begin{figure}
\centering
\includegraphics{r4ss/plots_mod1/comp_agedat_bubflt1mkt2_page2.png}
\caption{Commercial fishery age frequency distributions for Pacific
ocean perch. \label{fig:Comm_Age}}
\end{figure}

\FloatBarrier

\begin{figure}
\centering
\includegraphics{r4ss/plots_mod1/comp_lendat_bubflt2mkt0.png}
\caption{At-sea hake fishery length frequency distributions for Pacific
ocean perch. \label{fig:ASHOP_Length}}
\end{figure}

\FloatBarrier

\begin{figure}
\centering
\includegraphics{r4ss/plots_mod1/comp_agedat_bubflt2mkt0.png}
\caption{At-sea hake fishery age frequency distributions for Pacific
ocean perch. \label{fig:ASHOP_Age}}
\end{figure}

\FloatBarrier

\begin{figure}
\centering
\includegraphics{Figures/allSexRatios.png}
\caption{The estimated sex ratio of Pacific ocean perch at length from
all biological data sources. \label{fig:sexratio}}
\end{figure}

\begin{figure}
\centering
\includegraphics{Figures/allSexRatiosAge.png}
\caption{The estimated sex ratio of Pacific ocean perch at age from all
biological data sources. \label{fig:sexratio_Age}}
\end{figure}

\begin{figure}
\centering
\includegraphics{Figures/Functional_Maturity.png}
\caption{The estimated functional maturity of Pacific ocean perch at
length. \label{fig:mat}}
\end{figure}

\begin{figure}
\centering
\includegraphics{Figures/Maturity_Comparison.png}
\caption{Comparison between estimated maturity-at-length used in this
assessment and maturity-at-age applied in the 2011 assessment of Pacific
ocean perch. \label{fig:mat_compare}}
\end{figure}

\begin{figure}
\centering
\includegraphics{Figures/Fecundity_Comparison.png}
\caption{Fecundity at length of Pacific ocean perch in the base model
and a comparison of the fecundity in the 2011 assessment.
\label{fig:fecundity}}
\end{figure}

\FloatBarrier 

\begin{figure}
\centering
\includegraphics{Figures/weightAtLengthBySource.png}
\caption{Weight-at-length for Pacific ocean perch from all data sources.
\label{fig:Wt_len}}
\end{figure}

\FloatBarrier 

\begin{figure}
\centering
\includegraphics{Figures/weightAtLengthPred.png}
\caption{Estimated weight-at-length for Pacific ocean perch from all
data sources. \label{fig:Wt_len_pred}}
\end{figure}

\FloatBarrier 

\begin{figure}
\centering
\includegraphics{Figures/LengthAgeAll.png}
\caption{Estimated length-at-age for Pacific ocean perch from all data
sources. \label{fig:Len_Age}}
\end{figure}

\FloatBarrier 

\begin{figure}
\centering
\includegraphics{Figures/Ageing_Error.png}
\caption{The estimated ageing error used in this assessment compared to
the ageing error assumed in the previous assessment for Pacific ocean
perch. \label{fig:Age_Error}}
\end{figure}

\FloatBarrier 

\begin{figure}
\centering
\includegraphics{r4ss/plots_mod1/recruit_fit_bias_adjust.png}
\caption{Recruitment bias ramp applied in the base model.
\label{fig:bias_ramp}}
\end{figure}

\begin{figure}
\centering
\includegraphics{Figures/Catch_Comparison.png}
\caption{Comparison of the catches assumed by this assessment and the
previous assessment for Pacific ocean perch. \label{fig:Catch_Compare}}
\end{figure}

\FloatBarrier 

\begin{figure}
\centering
\includegraphics{Figures/bridging.png}
\caption{Comparison of model bridging estimates from Stock Synthesis
version 3.30 and 3.24 for Pacific ocean perch for the 2011 assessment.
\label{fig:bridge}}
\end{figure}

\FloatBarrier 

\begin{figure}
\centering
\includegraphics{Figures/Data_Bratio_uncertainty.png}
\caption{Each of the data sets used in the current assessment was added
to the 2011 model without updating model assumptions. Each data source
was included in an additive fashion where the final model ``+ Age'' is
the 2011 model with all data sources updated. \label{fig:data_update}}
\end{figure}

\FloatBarrier

\begin{figure}
\centering
\includegraphics{r4ss/plots_mod1/bio1_sizeatage.png}
\caption{Estimated length-at-age for male and female for Pacific ocean
perch with estimated CV. \label{fig:sizeatage}}
\end{figure}

\FloatBarrier 

\begin{figure}
\centering
\includegraphics{Figures/Growth_Estimate_Comparison.png}
\caption{Comparison between the estimated length-at-age for male and
female (solid lines) for Pacific ocean perch with estimated CV to the
external estimates based on the data (dashed lines).
\label{fig:length_compare}}
\end{figure}

\FloatBarrier 

\begin{figure}
\centering
\includegraphics{r4ss/plots_mod1/POP_selectivity_with_Tri.png}
\caption{Estimated selectivity by length by each fishery and survey for
Pacific ocean perch. The Triennial selectivity was fixed at the
estimated selectivity from preliminary models using the Triennial data.
The final selectivity was only used to remove Triennial catch from the
population. \label{fig:selex}}
\end{figure}

\FloatBarrier 

\begin{figure}
\centering
\includegraphics{r4ss/plots_mod1/POP_retention.png}
\caption{Estimated retention by length by the fishery fleet for Pacific
ocean perch. \label{fig:retention}}
\end{figure}

\FloatBarrier 

\begin{figure}
\centering
\includegraphics{r4ss/plots_mod1/ts11_Age-0_recruits_(1000s)_with_95_asymptotic_intervals.png}
\caption{Estimated time-series of recruitment for Pacific ocean perch.
\label{fig:recruits}}
\end{figure}

\FloatBarrier

\begin{figure}
\centering
\includegraphics{r4ss/plots_mod1/recdevs2_withbars.png}
\caption{Estimated time-series of recruitment deviations for Pacific
ocean perch. \label{fig:recdevs}}
\end{figure}

\FloatBarrier

\begin{figure}
\centering
\includegraphics{r4ss/plots_mod1/POP_index_fits.png}
\caption{Estimated fits to the survey indices for Pacific ocean perch.
\label{fig:index_fits}}
\end{figure}

\FloatBarrier 

\begin{figure}
\centering
\includegraphics{r4ss/plots_mod1/POP_discard_fits.png}
\caption{Estimated fits to the discard rates for Pacific ocean perch.
\label{fig:discard_fits}}
\end{figure}

\FloatBarrier 

\begin{figure}
\centering
\includegraphics{r4ss/plots_mod1/catch7 discards stacked plot (depends on multiple fleets).png}
\caption{Estimated total discards for Pacific ocean perch. Estimated
discard contributes less than 3.5 percent of the total morality across
all years from the fishery. \label{fig:total_discard}}
\end{figure}

\begin{figure}
\centering
\includegraphics{r4ss/plots_mod1/comp_lenfit__aggregated_across_time.png}
\caption{Length compositions aggregated across time by fleet. Labels
`retained' and `discard' indicate retained or discarded samples for each
fleet. Panels without this designation represent the whole catch. The
Triennial shelf survey length data were not used in the final model, but
the implied model fits are shown. \label{fig:length_agg}}
\end{figure}

\begin{figure}
\centering
\includegraphics{r4ss/plots_mod1/comp_lenfit_residsflt1mkt1.png}
\caption{Pearson residuals, discard, Fishery (max=4.14)\\
Closed bubbles are positive residuals (observed \textgreater{} expected)
and open bubbles are negative residuals (observed \textless{} expected).
\label{fig:discard_len_pearson}}
\end{figure}

\begin{figure}
\centering
\includegraphics{r4ss/plots_mod1/comp_lenfit_residsflt1mkt2_page4.png}
\caption{Pearson residuals, retained, Fishery (max=3.41)\\
Closed bubbles are positive residuals (observed \textgreater{} expected)
and open bubbles are negative residuals (observed \textless{} expected).
\label{fig:fishery_len_pearson}}
\end{figure}

\begin{figure}
\centering
\includegraphics{r4ss/plots_mod1/comp_lenfit_residsflt2mkt0.png}
\caption{Pearson residuals, whole catch, At\_sea hake (max=2.41)\\
Closed bubbles are positive residuals (observed \textgreater{} expected)
and open bubbles are negative residuals (observed \textless{} expected).
\label{fig:ashop_len_pearson}}
\end{figure}

\begin{figure}
\centering
\includegraphics{r4ss/plots_mod1/comp_lenfit_residsflt4mkt0.png}
\caption{Pearson residuals, whole catch, Pacific ocean perch survey
(max=1.82)\\
Closed bubbles are positive residuals (observed \textgreater{} expected)
and open bubbles are negative residuals (observed \textless{} expected).
\label{fig:pop_len_pearson}}
\end{figure}

\begin{figure}
\centering
\includegraphics{r4ss/plots_mod1/comp_lenfit_residsflt6mkt0.png}
\caption{Pearson residuals, whole catch, AFSC slope survey (max=2.88)\\
Closed bubbles are positive residuals (observed \textgreater{} expected)
and open bubbles are negative residuals (observed \textless{} expected).
\label{fig:afsc_len_pearson}}
\end{figure}

\begin{figure}
\centering
\includegraphics{r4ss/plots_mod1/comp_lenfit_residsflt7mkt0.png}
\caption{Pearson residuals, whole catch, NWFSC slope survey (max=3.38)\\
Closed bubbles are positive residuals (observed \textgreater{} expected)
and open bubbles are negative residuals (observed \textless{} expected).
\label{fig:nwfsc_len_pearson}}
\end{figure}

\begin{figure}
\centering
\includegraphics{r4ss/plots_mod1/comp_lenfit_residsflt8mkt0.png}
\caption{Pearson residuals, whole catch, NWFSC shelf\_slope survey
(max=2.85)\\
Closed bubbles are positive residuals (observed \textgreater{} expected)
and open bubbles are negative residuals (observed \textless{} expected).
\label{fig:nwfsc_combo_len_pearson}}
\end{figure}

\begin{figure}
\centering
\includegraphics{r4ss/plots_mod1/comp_lenfit_data_weighting_TA1.8_Fishery.png}
\caption{Mean length for Fishery with 95\% confidence intervals based on
current samples sizes. Francis data weighting method TA1.8: thinner
intervals (with capped ends) show result of further adjusting sample
sizes based on suggested multiplier (with 95\% interval) for len data
from Fishery: 0.9903 (0.6743\_1.745) For more info, see Francis,
R.I.C.C. (2011). Data weighting in statistical fisheries stock
assessment models. Can. J. Fish. Aquat. Sci. 68: 1124\_1138.
\label{fig:weighting_len_fishery}}
\end{figure}

\begin{figure}
\centering
\includegraphics{r4ss/plots_mod1/comp_lenfit_data_weighting_TA1.8_At-sea hake.png}
\caption{Mean length for At\_sea hake with 95\% confidence intervals
based on current samples sizes. Francis data weighting method TA1.8:
thinner intervals (with capped ends) show result of further adjusting
sample sizes based on suggested multiplier (with 95\% interval) for len
data from At\_sea hake: 0.9939 (0.4994\_5.6181) For more info, see
Francis, R.I.C.C. (2011). Data weighting in statistical fisheries stock
assessment models. Can. J. Fish. Aquat. Sci. 68: 1124\_1138.
\label{fig:weighting_len_ashop}}
\end{figure}

\begin{figure}
\centering
\includegraphics{r4ss/plots_mod1/comp_lenfit_data_weighting_TA1.8_Pacific ocean perch survey.png}
\caption{Mean length for Pacific ocean perch survey with 95\% confidence
intervals based on current samples sizes. Francis data weighting method
TA1.8: thinner intervals (with capped ends) show result of further
adjusting sample sizes based on suggested multiplier (with 95\%
interval) for len data from Pacific ocean perch survey: 9.0018
(9.0018\_Inf) For more info, see Francis, R.I.C.C. (2011). Data
weighting in statistical fisheries stock assessment models. Can. J.
Fish. Aquat. Sci. 68: 1124\_1138. \label{fig:weighting_len_pop}}
\end{figure}

\begin{figure}
\centering
\includegraphics{r4ss/plots_mod1/comp_lenfit_data_weighting_TA1.8_AFSC slope survey.png}
\caption{Mean length for AFSC slope survey with 95\% confidence
intervals based on current samples sizes. Francis data weighting method
TA1.8: thinner intervals (with capped ends) show result of further
adjusting sample sizes based on suggested multiplier (with 95\%
interval) for len data from AFSC slope survey: 0.9963 (0.5782\_16.165)
For more info, see Francis, R.I.C.C. (2011). Data weighting in
statistical fisheries stock assessment models. Can. J. Fish. Aquat. Sci.
68: 1124\_1138. \label{fig:weighting_len_afsc}}
\end{figure}

\begin{figure}
\centering
\includegraphics{r4ss/plots_mod1/comp_lenfit_data_weighting_TA1.8_NWFSC slope survey.png}
\caption{Mean length for NWFSC slope survey with 95\% confidence
intervals based on current samples sizes. Francis data weighting method
TA1.8: thinner intervals (with capped ends) show result of further
adjusting sample sizes based on suggested multiplier (with 95\%
interval) for len data from NWFSC slope survey: 0.9971 (0.9971\_Inf) For
more info, see Francis, R.I.C.C. (2011). Data weighting in statistical
fisheries stock assessment models. Can. J. Fish. Aquat. Sci. 68:
1124\_1138. \label{fig:weighting_len_nwfsc}}
\end{figure}

\begin{figure}
\centering
\includegraphics{r4ss/plots_mod1/comp_lenfit_data_weighting_TA1.8_NWFSC shelf-slope survey.png}
\caption{Mean length for NWFSC shelf\_slope survey with 95\% confidence
intervals based on current samples sizes. Francis data weighting method
TA1.8: thinner intervals (with capped ends) show result of further
adjusting sample sizes based on suggested multiplier (with 95\%
interval) for len data from NWFSC shelf\_slope survey: 1.0149
(0.594\_4.0526) For more info, see Francis, R.I.C.C. (2011). Data
weighting in statistical fisheries stock assessment models. Can. J.
Fish. Aquat. Sci. 68: 1124\_1138. \label{fig:weighting_len_nwfsccombo}}
\end{figure}

\FloatBarrier

\begin{figure}
\centering
\includegraphics{r4ss/plots_mod1/comp_agefit__aggregated_across_time.png}
\caption{Age compositions aggregated across time by fleet. The Triennial
shelf survey age data were not used in the final model, but the implied
model fits are shown. \label{fig:age_agg}}
\end{figure}

\begin{figure}
\centering
\includegraphics{r4ss/plots_mod1/comp_agefit_residsflt1mkt2_page2.png}
\caption{Pearson residuals, retained, Fishery (max=5.41)\\
Closed bubbles are positive residuals (observed \textgreater{} expected)
and open bubbles are negative residuals (observed \textless{} expected).
\label{fig:fishery_age_pearson}}
\end{figure}

\begin{figure}
\centering
\includegraphics{r4ss/plots_mod1/comp_agefit_residsflt2mkt0.png}
\caption{Pearson residuals, whole catch, At\_sea hake (max=3.91)\\
Closed bubbles are positive residuals (observed \textgreater{} expected)
and open bubbles are negative residuals (observed \textless{} expected).
\label{fig:ashop_age_pearson}}
\end{figure}

\begin{figure}
\centering
\includegraphics{r4ss/plots_mod1/comp_agefit_residsflt4mkt0.png}
\caption{Pearson residuals, whole catch, Pacific ocean perch survey
(max=2.62)\\
Closed bubbles are positive residuals (observed \textgreater{} expected)
and open bubbles are negative residuals (observed \textless{} expected).
\label{fig:pop_age_pearson}}
\end{figure}

\begin{figure}
\centering
\includegraphics{r4ss/plots_mod1/comp_agefit_residsflt7mkt0.png}
\caption{Pearson residuals, whole catch, NWFSC slope survey (max=2.28)\\
Closed bubbles are positive residuals (observed \textgreater{} expected)
and open bubbles are negative residuals (observed \textless{} expected).
\label{fig:nwfsc_age_pearson}}
\end{figure}

\begin{figure}
\centering
\includegraphics{r4ss/plots_mod1/comp_condAALfit_Andre_plotsflt8mkt0_page1.png}
\caption{Conditional AAL plot, whole catch, NWFSC shelf\_slope survey
(plot 1 of 5) These plots show mean age and std. dev. in conditional
AAL. Left plots are mean AAL by size\_class (obs. and pred.) with 90\%
CIs based on adding 1.64 SE of mean to the data. Right plots in each
pair are SE of mean AAL (obs. and pred.) with 90\% CIs based on the
chi\_square distribution. \label{fig:nwfsc_combo_andre_1}}
\end{figure}

\begin{figure}
\centering
\includegraphics{r4ss/plots_mod1/comp_condAALfit_Andre_plotsflt8mkt0_page2.png}
\caption{Conditional AAL plot, whole catch, NWFSC shelf\_slope survey
(plot 2 of 5) \label{fig:nwfsc_combo_andre_2}}
\end{figure}

\begin{figure}
\centering
\includegraphics{r4ss/plots_mod1/comp_condAALfit_Andre_plotsflt8mkt0_page3.png}
\caption{Conditional AAL plot, whole catch, NWFSC shelf\_slope survey
(plot 3 of 5) \label{fig:nwfsc_combo_andre_3}}
\end{figure}

\begin{figure}
\centering
\includegraphics{r4ss/plots_mod1/comp_condAALfit_Andre_plotsflt8mkt0_page4.png}
\caption{Conditional AAL plot, whole catch, NWFSC shelf\_slope survey
(plot 4 of 5) \label{fig:nwfsc_combo_andre_4}}
\end{figure}

\begin{figure}
\centering
\includegraphics{r4ss/plots_mod1/comp_condAALfit_Andre_plotsflt8mkt0_page5.png}
\caption{Conditional AAL plot, whole catch, NWFSC shelf\_slope survey
(plot 5 of 5) \label{fig:nwfsc_combo_andre_5}}
\end{figure}

\begin{figure}
\centering
\includegraphics{r4ss/plots_mod1/comp_agefit_data_weighting_TA1.8_Fishery.png}
\caption{Mean age for Fishery with 95\% confidence intervals based on
current samples sizes. Francis data weighting method TA1.8: thinner
intervals (with capped ends) show result of further adjusting sample
sizes based on suggested multiplier (with 95\% interval) for age data
from Fishery: 0.9999 (0.6705\_2.0064) For more info, see Francis,
R.I.C.C. (2011). Data weighting in statistical fisheries stock
assessment models. Can. J. Fish. Aquat. Sci. 68: 1124\_1138.
\label{fig:weighting_fishery}}
\end{figure}

\begin{figure}
\centering
\includegraphics{r4ss/plots_mod1/comp_agefit_data_weighting_TA1.8_At-sea hake.png}
\caption{Mean age for At\_sea hake with 95\% confidence intervals based
on current samples sizes. Francis data weighting method TA1.8: thinner
intervals (with capped ends) show result of further adjusting sample
sizes based on suggested multiplier (with 95\% interval) for age data
from At\_sea hake: 1.0068 (0.6598\_2756.7898) For more info, see
Francis, R.I.C.C. (2011). Data weighting in statistical fisheries stock
assessment models. Can. J. Fish. Aquat. Sci. 68: 1124\_1138.
\label{fig:weighting_ashop}}
\end{figure}

\begin{figure}
\centering
\includegraphics{r4ss/plots_mod1/comp_agefit_data_weighting_TA1.8_NWFSC slope survey.png}
\caption{Mean age for NWFSC slope survey with 95\% confidence intervals
based on current samples sizes. Francis data weighting method TA1.8:
thinner intervals (with capped ends) show result of further adjusting
sample sizes based on suggested multiplier (with 95\% interval) for age
data from NWFSC slope survey: 1.0004 (1.0004\_Inf) For more info, see
Francis, R.I.C.C. (2011). Data weighting in statistical fisheries stock
assessment models. Can. J. Fish. Aquat. Sci. 68: 1124\_1138.
\label{fig:weighting_nwfscslope}}
\end{figure}

\begin{figure}
\centering
\includegraphics{r4ss/plots_mod1/comp_condAALfit_data_weighting_TA1.8_condAgeNWFSC shelf-slope survey.png}
\caption{Mean age from conditional data (aggregated across length bins)
for NWFSC shelf\_slope survey with 95\% confidence intervals based on
current samples sizes. Francis data weighting method TA1.8: thinner
intervals (with capped ends) show result of further adjusting sample
sizes based on suggested multiplier (with 95\% interval) for conditional
age\_at\_length data from NWFSC shelf\_slope survey: 1.0037
(0.5733\_3.5119) For more info, see Francis, R.I.C.C. (2011). Data
weighting in statistical fisheries stock assessment models. Can. J.
Fish. Aquat. Sci. 68: 1124\_1138. \label{fig:weighting_nwfsccombo}}
\end{figure}

\FloatBarrier

\begin{figure}
\centering
\includegraphics{r4ss/plots_mod1/ts7_Spawning_output_with_95_asymptotic_intervals_intervals.png}
\caption{Estimated time-series of spawning output trajectory (circles
and line: median; light broken lines: 95\% credibility intervals) for
Pacific ocean perch. \label{fig:ssb}}
\end{figure}

\FloatBarrier

\begin{figure}
\centering
\includegraphics{r4ss/plots_mod1/ts1_Total_biomass_(mt).png}
\caption{Estimated time-series of total biomass for Pacific ocean perch.
\label{fig:total_bio}}
\end{figure}

\FloatBarrier

\begin{figure}
\centering
\includegraphics{r4ss/plots_mod1/ts9_Spawning_depletion_with_95_asymptotic_intervals_intervals.png}
\caption{Estimated time-series of relative spawning output (depletion)
(circles and line: median; light broken lines: 95\% credibility
intervals) for Pacific ocean perch. \label{fig:depl}}
\end{figure}

\FloatBarrier

\begin{figure}
\centering
\includegraphics{r4ss/plots_mod1/SR_curve2.png}
\caption{Estimated recruitment (red circles) and the assumed
stock-recruit relationship (black line). The green line shows the effect
of the bias correction for the lognormal distribution
\label{fig:stock_recruit_curve}}
\end{figure}

\begin{figure}
\centering
\includegraphics{Figures/ssb_sens1.png}
\caption{Time-series of spawning output for model sensitivities for
Pacific ocean perch. \label{fig:sens1_ssb}}
\end{figure}

\FloatBarrier

\begin{figure}
\centering
\includegraphics{Figures/depl_sens1.png}
\caption{Time-series of relative spawning output (depletion) for model
sensitivities for Pacific ocean perch. \label{fig:sens1_depl}}
\end{figure}

\FloatBarrier

\begin{figure}
\centering
\includegraphics{Figures/ssb_sens2.png}
\caption{Time-series of spawning output for model sensitivities for
Pacific ocean perch. \label{fig:sens2_ssb}}
\end{figure}

\FloatBarrier

\begin{figure}
\centering
\includegraphics{Figures/depl_sens2.png}
\caption{Time-series of relative spawning outptut (depletion) for model
sensitivities for Pacific ocean perch. \label{fig:sens2_depl}}
\end{figure}

\FloatBarrier

\begin{figure}
\centering
\includegraphics{Figures/compare2_spawnbio_uncertainty.png}
\caption{Retrospective pattern for spawning output.
\label{fig:retro_sb}}
\end{figure}

\FloatBarrier

\begin{figure}
\centering
\includegraphics{Figures/compare10_recdevs_uncertainty.png}
\caption{Retrospective pattern for estimated recruitment deviations.
\label{fig:retro_recdev}}
\end{figure}

\FloatBarrier

\begin{figure}
\centering
\includegraphics{Figures/piner_panel_h.png}
\caption{Likelihood profile across steepness values.
\label{fig:piner_h}}
\end{figure}

\FloatBarrier

\begin{figure}
\centering
\includegraphics{Figures/h_trajectories.png}
\caption{Trajectories of relative spawning output (depletion) across
values of steepness. \label{fig:h_trajectory}}
\end{figure}

\FloatBarrier

\begin{figure}
\centering
\includegraphics{Figures/piner_panel_m.png}
\caption{Likelihood profile across natural mortality values.
\label{fig:m_like}}
\end{figure}

\FloatBarrier

\begin{figure}
\centering
\includegraphics{Figures/m_trajectories.png}
\caption{Trajectories of relative spawning output (depletion) across
values of natural mortality. \label{fig:m_trajectory}}
\end{figure}

\FloatBarrier

\begin{figure}
\centering
\includegraphics{Figures/piner_panel_R0.png}
\caption{Likelihood profile across R\textsubscript{0} values.
\label{fig:piner_R0}}
\end{figure}

\FloatBarrier

\begin{figure}
\centering
\includegraphics{r4ss/plots_mod1/SPR3_ratiointerval.png}
\caption{Estimated spawning potential ratio (1-SPR)/(1-SPR50\%) for the
base-case model. One minus SPR is plotted so that higher exploitation
rates occur on the upper portion of the y-axis. The management target is
plotted as a red horizontal line and values above this reflect harvests
in excess of the overfishing proxy based on the SPR50\% harvest rate.
The last year in the time series is 2016. \label{fig:SPR}}
\end{figure}

\FloatBarrier

\begin{figure}
\centering
\includegraphics{r4ss/plots_mod1/yield1_yield_curve.png}
\caption{Equilibrium yield curve for the base case model. Values are
based on the 2016 fishery selectivity and with steepness fixed at 0.50.
\label{fig:yield}}
\end{figure}

\FloatBarrier

\newpage

\FloatBarrier
\newpage

\section{Appendix A. Detailed Fit to Length Composition
Data}\label{appendix-a.-detailed-fit-to-length-composition-data}

\begin{figure}
\centering
\includegraphics{r4ss/plots_mod1/comp_lenfit_flt1mkt1.png}
\caption{Length comps, discard, Fishery \label{fig:length_fits}}
\end{figure}

\begin{figure}
\centering
\includegraphics{r4ss/plots_mod1/comp_lenfit_flt1mkt2_page1.png}
\caption{Length comps, retained, Fishery (plot 1 of 4)
\label{fig:length_fits}}
\end{figure}

\begin{figure}
\centering
\includegraphics{r4ss/plots_mod1/comp_lenfit_flt1mkt2_page2.png}
\caption{Length comps, retained, Fishery (plot 2 of 4)
\label{fig:length_fits}}
\end{figure}

\begin{figure}
\centering
\includegraphics{r4ss/plots_mod1/comp_lenfit_flt1mkt2_page3.png}
\caption{Length comps, retained, Fishery (plot 3 of 4)
\label{fig:length_fits}}
\end{figure}

\begin{figure}
\centering
\includegraphics{r4ss/plots_mod1/comp_lenfit_flt1mkt2_page4.png}
\caption{Length comps, retained, Fishery (plot 4 of 4)
\label{fig:length_fits}}
\end{figure}

\begin{figure}
\centering
\includegraphics{r4ss/plots_mod1/comp_lenfit_flt2mkt0.png}
\caption{Length comps, whole catch, At\_sea hake
\label{fig:length_fits}}
\end{figure}

\begin{figure}
\centering
\includegraphics{r4ss/plots_mod1/comp_lenfit_flt4mkt0.png}
\caption{Length comps, whole catch, Pacific ocean perch survey
\label{fig:length_fits}}
\end{figure}

\begin{figure}
\centering
\includegraphics{r4ss/plots_mod1/comp_lenfit_flt6mkt0.png}
\caption{Length comps, whole catch, AFSC slope survey
\label{fig:length_fits}}
\end{figure}

\begin{figure}
\centering
\includegraphics{r4ss/plots_mod1/comp_lenfit_flt7mkt0.png}
\caption{Length comps, whole catch, NWFSC slope survey
\label{fig:length_fits}}
\end{figure}

\begin{figure}
\centering
\includegraphics{r4ss/plots_mod1/comp_lenfit_flt8mkt0.png}
\caption{Length comps, whole catch, NWFSC shelf\_slope survey
\label{fig:length_fits}}
\end{figure}

\FloatBarrier

\section{Appendix B. Detailed Fit to Age Composition
Data}\label{appendix-b.-detailed-fit-to-age-composition-data}

\begin{figure}
\centering
\includegraphics{r4ss/plots_mod1/comp_agefit_flt1mkt2_page1.png}
\caption{Age comps, retained, Fishery (plot 1 of 2)
\label{fig:age_fits}}
\end{figure}

\begin{figure}
\centering
\includegraphics{r4ss/plots_mod1/comp_agefit_flt1mkt2_page2.png}
\caption{Age comps, retained, Fishery (plot 2 of 2)
\label{fig:age_fits}}
\end{figure}

\begin{figure}
\centering
\includegraphics{r4ss/plots_mod1/comp_agefit_flt2mkt0.png}
\caption{Age comps, whole catch, At\_sea hake \label{fig:age_fits}}
\end{figure}

\begin{figure}
\centering
\includegraphics{r4ss/plots_mod1/comp_agefit_flt4mkt0.png}
\caption{Age comps, whole catch, Pacific ocean perch survey
\label{fig:age_fits}}
\end{figure}

\begin{figure}
\centering
\includegraphics{r4ss/plots_mod1/comp_gstagefit_flt8mkt0.png}
\caption{Ghost age comps, whole catch, NWFSC shelf\_slope survey
\label{fig:age_fits}}
\end{figure}

\FloatBarrier

\section{Appendix C. Description of CPUE and Triennial
Data}\label{appendix-c.-description-of-cpue-and-triennial-data}

Data on catch-per-unit-effort (CPUE) in mt/hr from the domestic fishery
were combined for the INPFC Vancouver and Columbia areas Gunderson
(\protect\hyperlink{ref-gunderson_population_1977}{1977})). Although
these data reflect catch rates for the US fleet, the highest catch rates
coincided with the beginning of removals by the foreign fleet. This
suggests that, barring unaccounted changes in fishing efficiency during
this period, the level of abundance was high at that time. The estimated
index of abundance is shown in Table \ref{tab:CPUE_Summary} and Figure
\ref{fig:Excluded_Indices}.

The Triennial shelf survey index of abundance was estimated based on the
data using the VAST delta-GLMM model. The estimated index of abundance
is shown in Table \ref{tab:CPUE_Summary} and Figure
\ref{fig:Excluded_Indices}. The lognormal distribution with random
strata-year had the lowest AIC and was chosen as the final model. The
index shows a decline in abundance in the early years of the time-series
and abundance remaining flat for the latter years.

Triennial shelf survey length and age compositions were expanded based
upon the survey stratification. The number of tows with length data
ranged from 17 in 1986 to 81 in 1998 (Table \ref{tab:TriennialLengths}).
Ages were read using surface reading methods until 1989 when the
break-and-burn method replaced surface reads as the best method to age
Pacific ocean perch. Unfortunately, surface reading of Pacific ocean
perch otoliths results in significant underestimates of age. Due to
this, these otoliths were excluded from analysis. The available ages
from the Triennial shelf survey and the number of tows where otoliths
were collected are shown in Table \ref{tab:Triennial_Ages}. The expanded
length and age frequencies from this survey are shown in Figures
\ref{fig:Tri_Length} and \ref{fig:Tri_Age}, respectively.

Including the fishery CPUE or the Triennial survey data in the final
base model had only negligible changes in the stock size and status
(Figures \ref{fig:Excluded_Data_ssb} and \ref{fig:Excluded_Data_depl}).

\begin{table}[ht]
\centering
\caption{Summary of the fishery CPUE
                                         and the Triennial shelf survey indices not used in the stock
                                         assessment.} 
\label{tab:CPUE_Summary}
\begin{tabular}{>{\centering}p{.5in}>{\centering}p{.7in}>{\centering}p{.7in}>{\centering}p{.7in}>{\centering}p{.7in}}
  \hline
   & \multicolumn{2}{c}{Fishery CPUE} &  \multicolumn{2}{c}{Triennial} \\
 Year & Obs & SE & Obs & SE \\
 \hline
1956 & 0.40 & 0.40 & - & - \\ 
  1957 & 0.30 & 0.40 & - & - \\ 
  1958 & 0.32 & 0.40 & - & - \\ 
  1959 & 0.29 & 0.40 & - & - \\ 
  1960 & 0.28 & 0.40 & - & - \\ 
  1961 & 0.31 & 0.40 & - & - \\ 
  1962 & 0.29 & 0.40 & - & - \\ 
  1963 & 0.34 & 0.40 & - & - \\ 
  1964 & 0.35 & 0.40 & - & - \\ 
  1965 & 0.55 & 0.40 & - & - \\ 
  1966 & 0.47 & 0.40 & - & - \\ 
  1967 & 0.30 & 0.40 & - & - \\ 
  1968 & 0.17 & 0.40 & - & - \\ 
  1969 & 0.18 & 0.40 & - & - \\ 
  1970 & 0.17 & 0.40 & - & - \\ 
  1971 & 0.20 & 0.40 & - & - \\ 
  1972 & 0.20 & 0.40 & - & - \\ 
  1973 & 0.11 & 0.40 & - & - \\ 
  1980 & - & - & 10384 & 0.64 \\ 
  1983 & - & - & 8974 & 0.59 \\ 
  1986 & - & - & 2977 & 0.65 \\ 
  1989 & - & - & 4873 & 0.65 \\ 
  1992 & - & - & 3207 & 0.64 \\ 
  1995 & - & - & 2724 & 0.62 \\ 
  1998 & - & - & 4163 & 0.63 \\ 
  2001 & - & - & 1494 & 0.63 \\ 
  2004 & - & - & 2922 & 0.67 \\ 
   \hline
\end{tabular}
\end{table}

\FloatBarrier

\begin{table}[ht]
\centering
\caption{Summary of Triennial shelf survey length samples.} 
\label{tab:TriennialLengths}
\begin{tabular}{>{\centering}p{.75in}>{\centering}p{.75in}>{\centering}p{.75in}>{\centering}p{1in}}
  \hline
Year & Tows & Fish & Sample Size \\ 
  \hline
1980 & 18 & 1315 & 43 \\ 
  1983 & 40 & 2820 & 97 \\ 
  1986 & 17 & 877 & 41 \\ 
  1989 & 42 & 1851 & 102 \\ 
  1992 & 33 & 1182 & 80 \\ 
  1995 & 71 & 1136 & 172 \\ 
  1998 & 81 & 1482 & 196 \\ 
  2001 & 74 & 669 & 179 \\ 
  2004 & 63 & 1240 & 153 \\ 
   \hline
\end{tabular}
\end{table}

\FloatBarrier

\begin{table}[ht]
\centering
\caption{Summary of Triennial shelf survey age samples.} 
\label{tab:Triennial_Ages}
\begin{tabular}{>{\centering}p{.75in}>{\centering}p{.75in}>{\centering}p{.75in}>{\centering}p{1in}}
  \hline
Year & Tows & Fish & Sample Size \\ 
  \hline
1989 & 15 & 577 & 36 \\ 
  1992 & 10 & 373 & 24 \\ 
  1995 & 12 & 275 & 29 \\ 
  1998 & 28 & 352 & 68 \\ 
  2001 & 43 & 342 & 104 \\ 
  2004 & 57 & 416 & 138 \\ 
   \hline
\end{tabular}
\end{table}

\FloatBarrier

\begin{figure}
\centering
\includegraphics{Figures/Index_Data_2.png}
\caption{Fishery CPUE and Triennial shelf survey indices of abundance
for Pacific ocean perch. The fishery CPUE was based on Gunderson 1977
and the Triennial shelf survey index was estimated using VAST.
\label{fig:Excluded_Indices}}
\end{figure}

\FloatBarrier

\begin{figure}
\centering
\includegraphics{r4ss/plots_mod1/comp_lendat_bubflt5mkt0.png}
\caption{Triennial shelf survey length frequency distributions for
Pacific ocean perch. \label{fig:Tri_Length}}
\end{figure}

\FloatBarrier

\begin{figure}
\centering
\includegraphics{r4ss/plots_mod1/comp_agedat_bubflt5mkt0.png}
\caption{Triennial shelf survey age frequency distributions for Pacific
ocean perch. \label{fig:Tri_Age}}
\end{figure}

\FloatBarrier

\begin{figure}
\centering
\includegraphics{Figures/ExcludedData_ssb.png}
\caption{Plot comparison of spawning output when either the fishery CPUE
or the Triennial shelf survey data are included in the base model for
Pacific ocean perch. \label{fig:Excluded_Data_ssb}}
\end{figure}

\FloatBarrier

\begin{figure}
\centering
\includegraphics{Figures/ExcludedData_depl.png}
\caption{Plot comparison of relative spawning output (depletion) when
either the fishery CPUE or the Triennial shelf survey data are included
in the base model for Pacific ocean perch.
\label{fig:Excluded_Data_depl}}
\end{figure}

\FloatBarrier

\newpage

\section{Appendix D. SSC Groundfish Subcommittee Discussion Regarding
Steepness}\label{appendix-d.-ssc-groundfish-subcommittee-discussion-regarding-steepness}

The Pacific ocean perch base model is highly sensitive to steepness. In
the final base model the profile over steepness is flat across a wide
range of potential values. The flat profile over steepness was in
contrast to the previous assessment model from 2011 where the likelihood
was minimized at a value of 0.40. The change in perceived information
regarding steepness between this and the last assessment is due to the
new data since 2011, the updated data weighting approach, and minor
changes in model structure. Given the lack of information regarding
steepness, preliminary models explored using the mean of the 2017
steepness prior, the approach endorsed by the SSC when there is not
information regarding steepness for a specific stock. However, using the
steepness prior of 0.72 resulted in Pacific ocean perch being estimated
near unfished conditions, a result that was in strong contrast to the
previous assessment which estimated the stock size at 19.1\% of unfished
stock size in 2011. Due to concerns of plausibility, the STAT team
presented an initial model to the STAR panel for review using an
intermediate steepness value of 0.50. This value was selected because
the resulting spawning output was encapsulated within the uncertainty
from when steepness was assumed to be 0.40, the previous assessment
value, and 0.72, the current mean of the steepness prior. Over the
course of the week of the STAR panel after many discussions both the
STAR panel and the STAT team agreed that in the absence of information
regarding steepness the base model should use the mean of the prior.
However, upon review by the SSC, it was concluded that the results of
the assessment model were implausible when the steepness prior of 0.72
was used. In particular, the value of catchability for Pacific ocean
perch from the NWFSC shelf-slope survey was 0.05, far below that for
other rockfish species observed off the US west coast. The SSC requested
additional model exploration be done and reviewed at the SSC Groundfish
Subcommittee (GFSC) September 28, 2017 meeting regarding steepness and
re-examining the information provided by the Triennial shelf survey.

Preliminary models included the Triennial shelf survey which was removed
from the final base model during the STAR panel due to model lack of fit
to this data-set which was in contrast to all other available data and
concerns that this survey did not sample a representative subset of the
population off the US west coast. Additionally, the estimates of
spawning output and depletion with and without the Triennial survey
data, given the value of steepness, were negligible indicating that the
other sources of information were the more influential data in the
model. However, profiles over steepness from preliminary models which
included the Triennial shelf survey indicated that the index of
abundance supported low steepness values (there was no information from
the length or age composition data). The perceived information regarding
steepness from this index of abundance is due to a change in the
abundance index between the first two data points of the survey, 1980
and 1983, which are higher than the subsequent years that drop to lower
abundance levels from 1986 to 2004 (final year of the survey). A profile
over steepness values when the Triennial shelf survey was used as a
single time-series resulted in a profile that had a local minimum at
0.75 with global minimum occurring at a steepness value of 0.27 (Figure
\ref{fig:Tri_Profile}). The estimated stock status when assuming a
steepness value of 0.27, a value that is far lower than any other
estimated steepness value for a US west coast groundfish stock, was less
that 10\% of the unfished spawning output in 2017. The STAT team and the
SSC GFSC agreed that this was not a plausible based upon other estimated
steepness values from US west coast groundfish.

The models explored regarding steepness either using the mean of the
prior or the value supported by the Triennial shelf survey index led to
quite different estimates of depletion for Pacific ocean perch in 2017.
Given the insufficient information to estimate steepness within the
model an alternative approach for determining steepness was proposed by
Dr.~Owen Hamel during the SSC GFSC webinar held on September 28, 2017.
The subcommittee notes state:

\emph{The GFSC therefore concluded that the available data are
insufficient to estimate steepness. It is usual in this situation to
base the assessment on the mean of the prior for steepness (0.72), but
this value leads to an unrealistically low estimate of survey
catchability (i.e.~model A {[}fixing steepness at 0.27{]}), and the
prior is rather diffuse with comparable support for values anywhere
between 0.4 and 1.0. Dr.~Hamel provided a way to account for uncertainty
in steepness that the GFSC recommends be adopted. This involves
calculating current ending spawning output biomass for steepness values
ranging from 0.25 to 0.95 in increments of 0.05 and assuming each value
to be equally plausible). The expected (i.e., arithmetic mean) ending
spawning output is 5,364 million eggs, which corresponds most closely to
a steepness value of 0.5 (5,296 million eggs for the run in the
profile). Thus, the model in which steepness is set to 0.5 represents
the expected ending spawning output given steepness values between 0.225
and 0.975 are considered equally likely.}

\emph{The GFSC therefore recommends that the base model be revised to
fix steepness to 0.5. The final base model should be retuned, checked by
jittering, and presented to the SSC for final approval and adoption.}

The STAT team agreed with the recommendation to fix steepness at 0.50
for the Pacific ocean perch base model.

\begin{figure}
\centering
\includegraphics{Figures/triennial_piner_panel_h.png}
\caption{Profile over steepness with the inclusion of the Triennial
shelf survey when treated as a single time-series for Pacific ocean
perch. \label{fig:Tri_Profile}}
\end{figure}

\FloatBarrier

\section{Appendix E. List of Auxiliary Files
Available}\label{appendix-e.-list-of-auxiliary-files-available}

The listed files are also available as auxiliary files to accompany the
assessment document:

\begin{enumerate}
  \item Numbers at age for female and male Pacific ocean perch (POPnatagef.csv and POPnatagem.csv)
  \item The Pacific ocean perch Stock Synthesis 3.30 model files
  
  \begin{enumerate}
    \item 2017pop.dat
    \item 2017pop.ctl
    \item forecast.ss
    \item starter.ss
  \end{enumerate}
\end{enumerate}

\newpage

--\textgreater{} \color{black}

\section{References}\label{references}

\renewcommand{\thepage}{}

\hypertarget{refs}{}
\hypertarget{ref-bradburn_2003_2011}{}
Bradburn, M., Keller, A., and Horness, B. 2011. The 2003 to 2008 US West
Coast bottom trawl surveys of groundfish resources off Washington,
Oregon, and California: Estimates of distribution, abundance, length,
and age composition. US Department of Commerce, National Oceanic;
Atmospheric Administration, National Marine Fisheries Service.

\hypertarget{ref-chilton_age_1982}{}
Chilton, D.E., and Beamish, R.J. 1982. Age determination methods for
fishes studied by the Groundfish Program at the Pacific Biological
Station. {[}Ottawa:{]} Minister of Supply; Services Canada.

\hypertarget{ref-dick_meta-analysis_2017}{}
Dick, E., Beyer, S., Mangel, M., and Ralston, S. 2017. A meta-analysis
of fecundity in rockfishes (genus \emph{Sebastes}). Fisheries Research
\textbf{187}: 73--85. doi:
\href{https://doi.org/10.1016/j.fishres.2016.11.009}{10.1016/j.fishres.2016.11.009}.

\hypertarget{ref-dick_modeling_2009}{}
Dick, E.J. 2009. Modeling the Reproductive Potential of Rockfishes
(\emph{Sebastes} Spp.). ProQuest. Available from
\url{http://books.google.com/books?hl=en\&lr=\&id=0d6-3rhfynkC\&oi=fnd\&pg=PR7\&dq=\%22Synthesis+of+findings+regarding+the+reproductive\%22+\%22C:+Linear+interpolation+algorithms\%22+\%22for+yellowtail+rockfish+(S.+flavidus)\%22+\%22greater+than+zero,+based+on+the+2-level+relative+fecundity\%22+\%22A:+Methods+for+data+recovery+from+published\%22+\&ots=NR0UylgymD\&sig=58IaN_a3pJeYTPYVmJ1NYMABmvE}
{[}accessed 27 February 2017{]}.

\hypertarget{ref-field_status_2007}{}
Field, J.C. 2007. Status of the Chilipepper rockfish, \emph{Sebastes
goodei}, in 2007. Pacific Fishery Management Council, 7700 Ambassador
Place NE, Suite 200, Portland, OR 97220.

\hypertarget{ref-francis_data_2011}{}
Francis, R.C., and Hilborn, R. 2011. Data weighting in statistical
fisheries stock assessment models. Canadian Journal of Fisheries and
Aquatic Sciences \textbf{68}(6): 1124--1138. doi:
\href{https://doi.org/10.1139/f2011-025}{10.1139/f2011-025}.

\hypertarget{ref-gertseva_status_2015}{}
Gertseva, V., Matson, S., and Councill, E. 2015. Status of the
darkblotched rockfish resource off the continental U.S. Pacific Coast in
2015. Pacific Fishery Management Council, 7700 Ambassador Place NE,
Suite 200, Portland, OR 97220.

\hypertarget{ref-gunderson_population_1977}{}
Gunderson, D.R. 1977. Population biology of Pacific ocean perch,
\emph{Sebastes alutus}, stocks in the WashingtonQueen Charlotte Sound
region and their response to fishing. Fishery Bulletin \textbf{75}:
369--403. Available from
\url{http://fishbull.noaa.gov/75-2/gunderson.pdf} {[}accessed 27
February 2017{]}.

\hypertarget{ref-gunderson_results_1978}{}
Gunderson, D.R. 1978. Results of cohort analysis for Pacific ocean perch
stocks off British Columbia, Washington, and Oregon and an evaluation of
alternative rebuilding strategies for these stocks. Pacific Fishery
Management Council, 7700 Ambassador Place NE, Suite 200, Portland, OR
97220.

\hypertarget{ref-gunderson_updated_1981}{}
Gunderson, D.R. 1981. An updated cohort analysis for Pacific ocean perch
stocks off Washington and Oregon. Unpublished report, Pacific Fishery
Management Council, 7700 Ambassador Place NE, Suite 200, Portland, OR
97220.

\hypertarget{ref-gunderson_trade-off_1997}{}
Gunderson, D.R. 1997. Trade-off between reproductive effort and adult
survival in oviparous and viviparous fishes. Canadian Journal of
Fisheries and Aquatic Sciences \textbf{54}(5): 990--998. Available from
\url{http://www.nrcresearchpress.com/doi/abs/10.1139/f97-019}
{[}accessed 27 February 2017{]}.

\hypertarget{ref-gunderson_distribution_1980}{}
Gunderson, D.R., and Sample, T.M. 1980. Distribution and abundance of
rockfish off Washington, Oregon and California during 1977. Northwest;
Alaska Fisheries Center, National Marine Fisheries Service. Available
from \url{http://spo.nmfs.noaa.gov/mfr423-4/mfr423-42.pdf} {[}accessed
28 February 2017{]}.

\hypertarget{ref-gunderson_status_1977}{}
Gunderson, D.R., Westrheim, S., Demory, R., and Fraidenburg, M. 1977.
The status of Pacific ocean perch (\emph{Sebastes alutus}) stocks off
British Columbia, Washington, and Oregon in 1974.

\hypertarget{ref-hamel_method_2015}{}
Hamel, O.S. 2015. A method for calculating a meta-analytical prior for
the natural mortality rate using multiple life history correlates. ICES
Journal of Marine Science: Journal du Conseil \textbf{72}(1): 62--69.
doi:
\href{https://doi.org/10.1093/icesjms/fsu131}{10.1093/icesjms/fsu131}.

\hypertarget{ref-hamel_stock_2011}{}
Hamel, O.S., and Ono, K. 2011. Stock Assessment of Pacific Ocean Perch
in Waters off of the U.S. West Coast in 2011. Pacific Fishery Management
Council, 7700 Ambassador Place NE, Suite 200, Portland, OR 97220.

\hypertarget{ref-hannah_age-modulated_2007}{}
Hannah, R., and Parker, S. 2007. Age-modulated variation in reproductive
development of female Pacific Ocean perch (\emph{Sebastes alutus}) in
waters off Oregon. Alaska Sea Grant, University of Alaska Fairbanks. pp.
1--20. doi:
\href{https://doi.org/10.4027/bamnpr.2007.01}{10.4027/bamnpr.2007.01}.

\hypertarget{ref-helser_generalized_2004}{}
Helser, T., Punt, A.E., and Methot, R.D. 2004. A generalized linear
mixed model analysis of a multi-vessel fishery resouce survey.
\textbf{70}: 251--264.

\hypertarget{ref-hicks_status_2015}{}
Hicks, A.C., and Wetzel, C.R. 2015. The status of Widow Rockfish
(\emph{Sebastes entomelas}) along the U.S. west coast in 2015. Pacific
Fishery Management Council, 7700 Ambassador Place NE, Suite 200,
Portland, OR 97220.

\hypertarget{ref-hicks_status_2009}{}
Hicks, A.C., Haltuch, M.A., and Wetzel, C.R. 2009. Status of
greenstriped rockfish (\emph{Sebastes elongatus}) along the outer coast
of California, Oregon, and Washington. Pacific Fishery Management
Council, 7700 Ambassador Place NE, Suite 200, Portland, OR 97220.

\hypertarget{ref-hoenig_empirical_1983}{}
Hoenig, J.M. 1983. Empirical use of longevity data to estimate mortality
rates. Fishery Bulletin \textbf{82}: 898--903. Available from
\url{http://fishbull.noaa.gov/81-4/hoenig.pdf} {[}accessed 28 February
2017{]}.

\hypertarget{ref-ianelli_status_1998}{}
Ianelli, J.N., and Zimmermann, M. 1998. Status and future prospects for
the Pacific ocean perch resource in waters off Washington and Oregon as
assessed in 1998. Pacific Fishery Management Council, 7700 Ambassador
Place NE, Suite 200, Portland, OR 97220.

\hypertarget{ref-ianelli_status_1992}{}
Ianelli, J.N., Ito, D.H., and Wilkins, M. 1992. Status and future
prospects for the Pacific ocean perch resource in waters off Washington
and Oregon as assessed in 1992. Pacific Fishery Management Council, 7700
Ambassador Place NE, Suite 200, Portland, OR 97220.

\hypertarget{ref-karnowski_historical_2014}{}
Karnowski, M., Gertseva, V., and Stephens, A. 2014. Historical
Reconstruction of Oregon's Commercial Fisheries Landings. Oregon
Department of Fish; Wildlife, Salem, OR.

\hypertarget{ref-kristensen_tmb:_2016}{}
Kristensen, K., Nielsen, A., Berg, C.W., Skaug, H.J., and Bell, B. 2016.
TMB: Automatic Differentiation and Laplace Approximation. Journal of
Statistical Software \textbf{70}: 1--21.

\hypertarget{ref-mcallister_bayesian_1997}{}
McAllister, M.K., and Ianelli, J.N. 1997. Bayesian stock assessment
using catch-age data and the sampling - importance resampling algorithm.
Canadian Journal of Fisheries and Aquatic Sciences \textbf{54}:
284--300. Available from
\url{http://www.nrcresearchpress.com/doi/pdf/10.1139/f96-285}
{[}accessed 10 March 2017{]}.

\hypertarget{ref-mccoy_predicting_2008}{}
McCoy, M.W., and Gillooly, J.F. 2008. Predicting natural mortality rates
of plants and animals. Ecology Letters \textbf{11}(7): 710--716. doi:
\href{https://doi.org/10.1111/j.1461-0248.2008.01190.x}{10.1111/j.1461-0248.2008.01190.x}.

\hypertarget{ref-methot_stock_2013}{}
Methot, R.D., and Wetzel, C.R. 2013. Stock synthesis: A biological and
statistical framework for fish stock assessment and fishery management.
Fisheries Research \textbf{142}: 86--99. doi:
\href{https://doi.org/10.1016/j.fishres.2012.10.012}{10.1016/j.fishres.2012.10.012}.

\hypertarget{ref-methot_adjusting_2011}{}
Methot, R.D., Taylor, I.G., and Chen, Y. 2011. Adjusting for bias due to
variability of estimated recruitments in fishery assessment models.
Canadian Journal of Fisheries and Aquatic Sciences \textbf{68}(10):
1744--1760. doi:
\href{https://doi.org/10.1139/f2011-092}{10.1139/f2011-092}.

\hypertarget{ref-pikitch_evaluation_1988}{}
Pikitch, E.K., Erickson, D.L., and Wallace, J.R. 1988. An evaluation of
the effectiveness of trip limits as a management tool. Northwest; Alaska
Fisheries Center, National Marine Fisheries Service NWAFC Processed
Report. Available from
\url{https://www.afsc.noaa.gov/Publications/ProcRpt/PR1988-27.pdf}
{[}accessed 28 February 2017{]}.

\hypertarget{ref-punt_quantifying_2008}{}
Punt, A.E., Smith, D.C., KrusicGolub, K., and Robertson, S. 2008.
Quantifying age-reading error for use in fisheries stock assessments,
with application to species in Australia's southern and eastern
scalefish and shark fishery. Canadian Journal of Fisheries and Aquatic
Sciences \textbf{65}(9): 1991--2005. doi:
\href{https://doi.org/10.1139/F08-111}{10.1139/F08-111}.

\hypertarget{ref-ralston_documentation_2010}{}
Ralston, S., Pearson, D.E., Field, J.C., and Key, M. 2010. Documentation
of the California catch reconstruction project. US Department of
Commerce, National Oceanic; Atmospheric Adminstration, National Marine.

\hypertarget{ref-rogers_species_2003}{}
Rogers, J. 2003. Species allocation of \emph{Sebastes} and
\emph{Sebastolobus} species caught by foreign countries off Washington,
Oregon, and California, U.S.A. in 1965-1976. Unpublished document.

\hypertarget{ref-rogers_numerical_1992}{}
Rogers, J.B., and Pikitch, E.K. 1992. Numerical definition of groundfish
assemblages caught off the coasts of Oregon and Washington using
commercial fishing strategies. Canadian Journal of Fisheries and Aquatic
Sciences \textbf{49}(12): 2648--2656. Available from
\url{http://www.nrcresearchpress.com/doi/abs/10.1139/f92-293}
{[}accessed 9 March 2017{]}.

\hypertarget{ref-seeb_genetic_1988}{}
Seeb, L.W., and Gunderson, D.R. 1988. Genetic variation and population
structure of Pacific ocean perch (\emph{Sebastes alutus}). Canadian
Journal of Fisheries and Aquatic Sciences \textbf{45}(1): 78--88.
Available from
\url{http://www.nrcresearchpress.com/doi/abs/10.1139/f88-010}
{[}accessed 28 February 2017{]}.

\hypertarget{ref-stewart_bootstrapping_2014}{}
Stewart, I.J., and Hamel, O.S. 2014. Bootstrapping of sample sizes for
length- or age-composition data used in stock assessments. Canadian
Journal of Fisheries and Aquatic Sciences \textbf{71}(4): 581--588. doi:
\href{https://doi.org/10.1139/cjfas-2013-0289}{10.1139/cjfas-2013-0289}.

\hypertarget{ref-then_evaluating_2015}{}
Then, A.Y., Hoenig, J.M., Hall, N.G., and Hewitt, D.A. 2015. Evaluating
the predictive performance of empirical estimators of natural mortality
rate using information on over 200 fish species. ICES Journal of Marine
Science \textbf{72}(1): 82--92. doi:
\href{https://doi.org/10.1093/icesjms/fsu136}{10.1093/icesjms/fsu136}.

\hypertarget{ref-thorson_comparing_2017}{}
Thorson, J.T., and Barnett, L.A.K. 2017. Comparing estimates of
abundance trends and distribution shifts using single- and multispecies
models of fishes and biogenic habitat. ICES Journal of Marine Science:
Journal du Conseil: fsw193. doi:
\href{https://doi.org/10.1093/icesjms/fsw193}{10.1093/icesjms/fsw193}.

\hypertarget{ref-thorson_implementing_2016}{}
Thorson, J.T., and Kristensen, K. 2016. Implementing a generic method
for bias correction in statistical models using random effects, with
spatial and population dynamics examples. Fisheries Research
\textbf{175}: 66--74. doi:
\href{https://doi.org/10.1016/j.fishres.2015.11.016}{10.1016/j.fishres.2015.11.016}.

\hypertarget{ref-thorson_accounting_2014}{}
Thorson, J.T., and Ward, E.J. 2014. Accounting for vessel effects when
standardizing catch rates from cooperative surveys. Fisheries Research
\textbf{155}: 168--176. doi:
\href{https://doi.org/10.1016/j.fishres.2014.02.036}{10.1016/j.fishres.2014.02.036}.

\hypertarget{ref-thorson_geostatistical_2015}{}
Thorson, J.T., Shelton, A.O., Ward, E.J., and Skaug, H.J. 2015.
Geostatistical delta-generalized linear mixed models improve precision
for estimated abundance indices for West Coast groundfishes. ICES
Journal of Marine Science \textbf{72}(5): 1297--1310. doi:
\href{https://doi.org/10.1093/icesjms/fsu243}{10.1093/icesjms/fsu243}.

\hypertarget{ref-thorson_nwfscageingerror:_2012}{}
Thorson, J.T., Stewart, I.J., and Punt, A.E. 2012. nwfscAgeingError: A
user interface in R for the Punt et al. (2008) method for calculating
ageing error and imprecision. Available from:
http://github.com/nwfsc-assess/nwfscAgeingError/.

\hypertarget{ref-weinberg_estimation_2002}{}
Weinberg, J.R., Rago, P.J., Wakefield, W.W., and Keith, C. 2002.
Estimation of tow distance and spatial heterogeneity using data from
inclinometer sensors: An example using a clam survey dredge. Fisheries
Research \textbf{55}(1--3): 49--61. doi:
\href{https://doi.org/10.1016/S0165-7836(01)00292-2}{10.1016/S0165-7836(01)00292-2}.

\hypertarget{ref-wilkins_condition_1983}{}
Wilkins, M., and Golden, J. 1983. Condition of the Pacific ocean perch
resource off Washington and Oregon during 1979: Results of a cooperative
trawl survey. North American Journal of Fisheries Management \textbf{3}:
103--122.

\hypertarget{ref-withler_co-existing_2001}{}
Withler, R., Beacham, T., Schulze, A., Richards, L., and Miller, K.
2001. Co-existing populations of Pacific ocean perch, \emph{Sebastes
alutus} , in Queen Charlotte Sound, British Columbia. Marine Biology
\textbf{139}(1): 1--12. doi:
\href{https://doi.org/10.1007/s002270100560}{10.1007/s002270100560}.

\end{document}
